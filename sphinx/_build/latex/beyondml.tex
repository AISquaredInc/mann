%% Generated by Sphinx.
\def\sphinxdocclass{report}
\documentclass[letterpaper,10pt,english]{sphinxmanual}
\ifdefined\pdfpxdimen
   \let\sphinxpxdimen\pdfpxdimen\else\newdimen\sphinxpxdimen
\fi \sphinxpxdimen=.75bp\relax
\ifdefined\pdfimageresolution
    \pdfimageresolution= \numexpr \dimexpr1in\relax/\sphinxpxdimen\relax
\fi
%% let collapsible pdf bookmarks panel have high depth per default
\PassOptionsToPackage{bookmarksdepth=5}{hyperref}


\PassOptionsToPackage{warn}{textcomp}
\usepackage[utf8]{inputenc}
\ifdefined\DeclareUnicodeCharacter
% support both utf8 and utf8x syntaxes
  \ifdefined\DeclareUnicodeCharacterAsOptional
    \def\sphinxDUC#1{\DeclareUnicodeCharacter{"#1}}
  \else
    \let\sphinxDUC\DeclareUnicodeCharacter
  \fi
  \sphinxDUC{00A0}{\nobreakspace}
  \sphinxDUC{2500}{\sphinxunichar{2500}}
  \sphinxDUC{2502}{\sphinxunichar{2502}}
  \sphinxDUC{2514}{\sphinxunichar{2514}}
  \sphinxDUC{251C}{\sphinxunichar{251C}}
  \sphinxDUC{2572}{\textbackslash}
\fi
\usepackage{cmap}
\usepackage[T1]{fontenc}
\usepackage{amsmath,amssymb,amstext}
\usepackage{babel}



\usepackage{tgtermes}
\usepackage{tgheros}
\renewcommand{\ttdefault}{txtt}



\usepackage[Bjarne]{fncychap}
\usepackage{sphinx}

\fvset{fontsize=auto}
\usepackage{geometry}


% Include hyperref last.
\usepackage{hyperref}
% Fix anchor placement for figures with captions.
\usepackage{hypcap}% it must be loaded after hyperref.
% Set up styles of URL: it should be placed after hyperref.
\urlstyle{same}

\addto\captionsenglish{\renewcommand{\contentsname}{Documentation:}}

\usepackage{sphinxmessages}
\setcounter{tocdepth}{1}



\title{BeyondML}
\date{Jan 11, 2023}
\release{}
\author{BeyondML Labs}
\newcommand{\sphinxlogo}{\vbox{}}
\renewcommand{\releasename}{}
\makeindex
\begin{document}

\ifdefined\shorthandoff
  \ifnum\catcode`\=\string=\active\shorthandoff{=}\fi
  \ifnum\catcode`\"=\active\shorthandoff{"}\fi
\fi

\pagestyle{empty}
\sphinxmaketitle
\pagestyle{plain}
\sphinxtableofcontents
\pagestyle{normal}
\phantomsection\label{\detokenize{index::doc}}
\noindent{\hspace*{\fill}\sphinxincludegraphics[width=400\sphinxpxdimen]{{BeyondML_horizontal-color}.png}\hspace*{\fill}}

\begin{DUlineblock}{0em}
\item[] 
\end{DUlineblock}



\sphinxAtStartPar
BeyondML is a Python package which enables creating sparse multitask artificial neural networks (MANNs)
compatible with \sphinxhref{https://tensorflow.org}{TensorFlow} and \sphinxhref{https://pytorch.org}{PyTorch}.
This package contains custom layers and utilities to facilitate the training and optimization of models
using the Reduction of Sub\sphinxhyphen{}Network Neuroplasticity (RSN2) training procedure developed by \sphinxhref{https://squared.ai}{AI Squared, Inc}.

\sphinxAtStartPar
\sphinxcode{\sphinxupquote{View this Documentation in PDF Format}}


\chapter{Installation}
\label{\detokenize{index:installation}}
\sphinxAtStartPar
This package is available through \sphinxhref{https://pypi.org}{Pypi} and can be installed by running the following command:

\begin{sphinxVerbatim}[commandchars=\\\{\}]
pip\PYG{+w}{ }install\PYG{+w}{ }beyondml
\end{sphinxVerbatim}

\sphinxAtStartPar
Alternatively, the latest version of the software can be installed directly from GitHub using the following command:

\begin{sphinxVerbatim}[commandchars=\\\{\}]
pip\PYG{+w}{ }install\PYG{+w}{ }git+https://github.com/beyond\PYGZhy{}ml\PYGZhy{}labs/beyondml
\end{sphinxVerbatim}

\sphinxstepscope


\section{beyondml}
\label{\detokenize{modules:beyondml}}\label{\detokenize{modules::doc}}
\sphinxstepscope


\subsection{beyondml package}
\label{\detokenize{beyondml:beyondml-package}}\label{\detokenize{beyondml::doc}}

\subsubsection{Subpackages}
\label{\detokenize{beyondml:subpackages}}
\sphinxstepscope


\paragraph{beyondml.pt package}
\label{\detokenize{beyondml.pt:beyondml-pt-package}}\label{\detokenize{beyondml.pt::doc}}

\subparagraph{Subpackages}
\label{\detokenize{beyondml.pt:subpackages}}
\sphinxstepscope


\subparagraph{beyondml.pt.layers package}
\label{\detokenize{beyondml.pt.layers:beyondml-pt-layers-package}}\label{\detokenize{beyondml.pt.layers::doc}}

\subparagraph{Submodules}
\label{\detokenize{beyondml.pt.layers:submodules}}

\subparagraph{beyondml.pt.layers.Conv2D module}
\label{\detokenize{beyondml.pt.layers:module-beyondml.pt.layers.Conv2D}}\label{\detokenize{beyondml.pt.layers:beyondml-pt-layers-conv2d-module}}\index{module@\spxentry{module}!beyondml.pt.layers.Conv2D@\spxentry{beyondml.pt.layers.Conv2D}}\index{beyondml.pt.layers.Conv2D@\spxentry{beyondml.pt.layers.Conv2D}!module@\spxentry{module}}\index{Conv2D (class in beyondml.pt.layers.Conv2D)@\spxentry{Conv2D}\spxextra{class in beyondml.pt.layers.Conv2D}}

\begin{fulllineitems}
\phantomsection\label{\detokenize{beyondml.pt.layers:beyondml.pt.layers.Conv2D.Conv2D}}
\pysigstartsignatures
\pysiglinewithargsret{\sphinxbfcode{\sphinxupquote{class\DUrole{w}{  }}}\sphinxcode{\sphinxupquote{beyondml.pt.layers.Conv2D.}}\sphinxbfcode{\sphinxupquote{Conv2D}}}{\emph{\DUrole{n}{kernel}}, \emph{\DUrole{n}{bias}}, \emph{\DUrole{n}{padding}\DUrole{o}{=}\DUrole{default_value}{\textquotesingle{}same\textquotesingle{}}}, \emph{\DUrole{n}{strides}\DUrole{o}{=}\DUrole{default_value}{1}}, \emph{\DUrole{n}{device}\DUrole{o}{=}\DUrole{default_value}{None}}, \emph{\DUrole{n}{dtype}\DUrole{o}{=}\DUrole{default_value}{None}}}{}
\pysigstopsignatures
\sphinxAtStartPar
Bases: \sphinxcode{\sphinxupquote{Module}}

\sphinxAtStartPar
Convolutional 2D layer initialized directly with weights, rather than with hyperparameters
\index{forward() (beyondml.pt.layers.Conv2D.Conv2D method)@\spxentry{forward()}\spxextra{beyondml.pt.layers.Conv2D.Conv2D method}}

\begin{fulllineitems}
\phantomsection\label{\detokenize{beyondml.pt.layers:beyondml.pt.layers.Conv2D.Conv2D.forward}}
\pysigstartsignatures
\pysiglinewithargsret{\sphinxbfcode{\sphinxupquote{forward}}}{\emph{\DUrole{n}{inputs}}}{}
\pysigstopsignatures
\sphinxAtStartPar
Call the layer on input data
\begin{quote}\begin{description}
\sphinxlineitem{Parameters}
\sphinxAtStartPar
\sphinxstyleliteralstrong{\sphinxupquote{inputs}} (\sphinxstyleliteralemphasis{\sphinxupquote{torch.Tensor}}) \textendash{} Inputs to call the layer’s logic on

\sphinxlineitem{Returns}
\sphinxAtStartPar
\sphinxstylestrong{results} \textendash{} The results of the layer’s logic

\sphinxlineitem{Return type}
\sphinxAtStartPar
torch.Tensor

\end{description}\end{quote}

\end{fulllineitems}


\end{fulllineitems}



\subparagraph{beyondml.pt.layers.Conv3D module}
\label{\detokenize{beyondml.pt.layers:module-beyondml.pt.layers.Conv3D}}\label{\detokenize{beyondml.pt.layers:beyondml-pt-layers-conv3d-module}}\index{module@\spxentry{module}!beyondml.pt.layers.Conv3D@\spxentry{beyondml.pt.layers.Conv3D}}\index{beyondml.pt.layers.Conv3D@\spxentry{beyondml.pt.layers.Conv3D}!module@\spxentry{module}}\index{Conv3D (class in beyondml.pt.layers.Conv3D)@\spxentry{Conv3D}\spxextra{class in beyondml.pt.layers.Conv3D}}

\begin{fulllineitems}
\phantomsection\label{\detokenize{beyondml.pt.layers:beyondml.pt.layers.Conv3D.Conv3D}}
\pysigstartsignatures
\pysiglinewithargsret{\sphinxbfcode{\sphinxupquote{class\DUrole{w}{  }}}\sphinxcode{\sphinxupquote{beyondml.pt.layers.Conv3D.}}\sphinxbfcode{\sphinxupquote{Conv3D}}}{\emph{\DUrole{n}{kernel}}, \emph{\DUrole{n}{bias}}, \emph{\DUrole{n}{padding}\DUrole{o}{=}\DUrole{default_value}{\textquotesingle{}same\textquotesingle{}}}, \emph{\DUrole{n}{strides}\DUrole{o}{=}\DUrole{default_value}{1}}, \emph{\DUrole{n}{device}\DUrole{o}{=}\DUrole{default_value}{None}}, \emph{\DUrole{n}{dtype}\DUrole{o}{=}\DUrole{default_value}{None}}}{}
\pysigstopsignatures
\sphinxAtStartPar
Bases: \sphinxcode{\sphinxupquote{Module}}

\sphinxAtStartPar
Convolutional 3D layer initialized directly with weights, rather than with hyperparameters
\index{forward() (beyondml.pt.layers.Conv3D.Conv3D method)@\spxentry{forward()}\spxextra{beyondml.pt.layers.Conv3D.Conv3D method}}

\begin{fulllineitems}
\phantomsection\label{\detokenize{beyondml.pt.layers:beyondml.pt.layers.Conv3D.Conv3D.forward}}
\pysigstartsignatures
\pysiglinewithargsret{\sphinxbfcode{\sphinxupquote{forward}}}{\emph{\DUrole{n}{inputs}}}{}
\pysigstopsignatures
\sphinxAtStartPar
Call the layer on input data
\begin{quote}\begin{description}
\sphinxlineitem{Parameters}
\sphinxAtStartPar
\sphinxstyleliteralstrong{\sphinxupquote{inputs}} (\sphinxstyleliteralemphasis{\sphinxupquote{torch.Tensor}}) \textendash{} Inputs to call the layer’s logic on

\sphinxlineitem{Returns}
\sphinxAtStartPar
\sphinxstylestrong{results} \textendash{} The results of the layer’s logic

\sphinxlineitem{Return type}
\sphinxAtStartPar
torch.Tensor

\end{description}\end{quote}

\end{fulllineitems}


\end{fulllineitems}



\subparagraph{beyondml.pt.layers.Dense module}
\label{\detokenize{beyondml.pt.layers:module-beyondml.pt.layers.Dense}}\label{\detokenize{beyondml.pt.layers:beyondml-pt-layers-dense-module}}\index{module@\spxentry{module}!beyondml.pt.layers.Dense@\spxentry{beyondml.pt.layers.Dense}}\index{beyondml.pt.layers.Dense@\spxentry{beyondml.pt.layers.Dense}!module@\spxentry{module}}\index{Dense (class in beyondml.pt.layers.Dense)@\spxentry{Dense}\spxextra{class in beyondml.pt.layers.Dense}}

\begin{fulllineitems}
\phantomsection\label{\detokenize{beyondml.pt.layers:beyondml.pt.layers.Dense.Dense}}
\pysigstartsignatures
\pysiglinewithargsret{\sphinxbfcode{\sphinxupquote{class\DUrole{w}{  }}}\sphinxcode{\sphinxupquote{beyondml.pt.layers.Dense.}}\sphinxbfcode{\sphinxupquote{Dense}}}{\emph{\DUrole{n}{weight}}, \emph{\DUrole{n}{bias}}, \emph{\DUrole{n}{device}\DUrole{o}{=}\DUrole{default_value}{None}}, \emph{\DUrole{n}{dtype}\DUrole{o}{=}\DUrole{default_value}{None}}}{}
\pysigstopsignatures
\sphinxAtStartPar
Bases: \sphinxcode{\sphinxupquote{Module}}

\sphinxAtStartPar
Fully\sphinxhyphen{}connected layer initialized directly with weights, rather than hyperparameters
\index{forward() (beyondml.pt.layers.Dense.Dense method)@\spxentry{forward()}\spxextra{beyondml.pt.layers.Dense.Dense method}}

\begin{fulllineitems}
\phantomsection\label{\detokenize{beyondml.pt.layers:beyondml.pt.layers.Dense.Dense.forward}}
\pysigstartsignatures
\pysiglinewithargsret{\sphinxbfcode{\sphinxupquote{forward}}}{\emph{\DUrole{n}{inputs}}}{}
\pysigstopsignatures
\sphinxAtStartPar
Call the layer on input data
\begin{quote}\begin{description}
\sphinxlineitem{Parameters}
\sphinxAtStartPar
\sphinxstyleliteralstrong{\sphinxupquote{inputs}} (\sphinxstyleliteralemphasis{\sphinxupquote{torch.Tensor}}) \textendash{} Inputs to call the layer’s logic on

\sphinxlineitem{Returns}
\sphinxAtStartPar
\sphinxstylestrong{results} \textendash{} The results of the layer’s logic

\sphinxlineitem{Return type}
\sphinxAtStartPar
torch.Tensor

\end{description}\end{quote}

\end{fulllineitems}


\end{fulllineitems}



\subparagraph{beyondml.pt.layers.FilterLayer module}
\label{\detokenize{beyondml.pt.layers:module-beyondml.pt.layers.FilterLayer}}\label{\detokenize{beyondml.pt.layers:beyondml-pt-layers-filterlayer-module}}\index{module@\spxentry{module}!beyondml.pt.layers.FilterLayer@\spxentry{beyondml.pt.layers.FilterLayer}}\index{beyondml.pt.layers.FilterLayer@\spxentry{beyondml.pt.layers.FilterLayer}!module@\spxentry{module}}\index{FilterLayer (class in beyondml.pt.layers.FilterLayer)@\spxentry{FilterLayer}\spxextra{class in beyondml.pt.layers.FilterLayer}}

\begin{fulllineitems}
\phantomsection\label{\detokenize{beyondml.pt.layers:beyondml.pt.layers.FilterLayer.FilterLayer}}
\pysigstartsignatures
\pysiglinewithargsret{\sphinxbfcode{\sphinxupquote{class\DUrole{w}{  }}}\sphinxcode{\sphinxupquote{beyondml.pt.layers.FilterLayer.}}\sphinxbfcode{\sphinxupquote{FilterLayer}}}{\emph{\DUrole{n}{is\_on}\DUrole{o}{=}\DUrole{default_value}{True}}, \emph{\DUrole{n}{device}\DUrole{o}{=}\DUrole{default_value}{None}}, \emph{\DUrole{n}{dtype}\DUrole{o}{=}\DUrole{default_value}{None}}}{}
\pysigstopsignatures
\sphinxAtStartPar
Bases: \sphinxcode{\sphinxupquote{Module}}

\sphinxAtStartPar
Layer which filters input data, either returning values or all zeros depending on state
\index{forward() (beyondml.pt.layers.FilterLayer.FilterLayer method)@\spxentry{forward()}\spxextra{beyondml.pt.layers.FilterLayer.FilterLayer method}}

\begin{fulllineitems}
\phantomsection\label{\detokenize{beyondml.pt.layers:beyondml.pt.layers.FilterLayer.FilterLayer.forward}}
\pysigstartsignatures
\pysiglinewithargsret{\sphinxbfcode{\sphinxupquote{forward}}}{\emph{\DUrole{n}{inputs}}}{}
\pysigstopsignatures
\sphinxAtStartPar
Call the layer on input data
\begin{quote}\begin{description}
\sphinxlineitem{Parameters}
\sphinxAtStartPar
\sphinxstyleliteralstrong{\sphinxupquote{inputs}} (\sphinxstyleliteralemphasis{\sphinxupquote{torch.Tensor}}) \textendash{} Inputs to call the layer’s logic on

\sphinxlineitem{Returns}
\sphinxAtStartPar
\sphinxstylestrong{results} \textendash{} The results of the layer’s logic

\sphinxlineitem{Return type}
\sphinxAtStartPar
torch.Tensor

\end{description}\end{quote}

\end{fulllineitems}

\index{is\_on (beyondml.pt.layers.FilterLayer.FilterLayer property)@\spxentry{is\_on}\spxextra{beyondml.pt.layers.FilterLayer.FilterLayer property}}

\begin{fulllineitems}
\phantomsection\label{\detokenize{beyondml.pt.layers:beyondml.pt.layers.FilterLayer.FilterLayer.is_on}}
\pysigstartsignatures
\pysigline{\sphinxbfcode{\sphinxupquote{property\DUrole{w}{  }}}\sphinxbfcode{\sphinxupquote{is\_on}}}
\pysigstopsignatures
\end{fulllineitems}

\index{turn\_off() (beyondml.pt.layers.FilterLayer.FilterLayer method)@\spxentry{turn\_off()}\spxextra{beyondml.pt.layers.FilterLayer.FilterLayer method}}

\begin{fulllineitems}
\phantomsection\label{\detokenize{beyondml.pt.layers:beyondml.pt.layers.FilterLayer.FilterLayer.turn_off}}
\pysigstartsignatures
\pysiglinewithargsret{\sphinxbfcode{\sphinxupquote{turn\_off}}}{}{}
\pysigstopsignatures
\sphinxAtStartPar
Turn off the layer

\end{fulllineitems}

\index{turn\_on() (beyondml.pt.layers.FilterLayer.FilterLayer method)@\spxentry{turn\_on()}\spxextra{beyondml.pt.layers.FilterLayer.FilterLayer method}}

\begin{fulllineitems}
\phantomsection\label{\detokenize{beyondml.pt.layers:beyondml.pt.layers.FilterLayer.FilterLayer.turn_on}}
\pysigstartsignatures
\pysiglinewithargsret{\sphinxbfcode{\sphinxupquote{turn\_on}}}{}{}
\pysigstopsignatures
\sphinxAtStartPar
Turn on the layer

\end{fulllineitems}


\end{fulllineitems}



\subparagraph{beyondml.pt.layers.MaskedConv2D module}
\label{\detokenize{beyondml.pt.layers:module-beyondml.pt.layers.MaskedConv2D}}\label{\detokenize{beyondml.pt.layers:beyondml-pt-layers-maskedconv2d-module}}\index{module@\spxentry{module}!beyondml.pt.layers.MaskedConv2D@\spxentry{beyondml.pt.layers.MaskedConv2D}}\index{beyondml.pt.layers.MaskedConv2D@\spxentry{beyondml.pt.layers.MaskedConv2D}!module@\spxentry{module}}\index{MaskedConv2D (class in beyondml.pt.layers.MaskedConv2D)@\spxentry{MaskedConv2D}\spxextra{class in beyondml.pt.layers.MaskedConv2D}}

\begin{fulllineitems}
\phantomsection\label{\detokenize{beyondml.pt.layers:beyondml.pt.layers.MaskedConv2D.MaskedConv2D}}
\pysigstartsignatures
\pysiglinewithargsret{\sphinxbfcode{\sphinxupquote{class\DUrole{w}{  }}}\sphinxcode{\sphinxupquote{beyondml.pt.layers.MaskedConv2D.}}\sphinxbfcode{\sphinxupquote{MaskedConv2D}}}{\emph{\DUrole{n}{in\_channels}}, \emph{\DUrole{n}{out\_channels}}, \emph{\DUrole{n}{kernel\_size}\DUrole{o}{=}\DUrole{default_value}{3}}, \emph{\DUrole{n}{padding}\DUrole{o}{=}\DUrole{default_value}{\textquotesingle{}same\textquotesingle{}}}, \emph{\DUrole{n}{strides}\DUrole{o}{=}\DUrole{default_value}{1}}, \emph{\DUrole{n}{device}\DUrole{o}{=}\DUrole{default_value}{None}}, \emph{\DUrole{n}{dtype}\DUrole{o}{=}\DUrole{default_value}{None}}}{}
\pysigstopsignatures
\sphinxAtStartPar
Bases: \sphinxcode{\sphinxupquote{Module}}

\sphinxAtStartPar
Masked 2D Convolutional layer
\index{forward() (beyondml.pt.layers.MaskedConv2D.MaskedConv2D method)@\spxentry{forward()}\spxextra{beyondml.pt.layers.MaskedConv2D.MaskedConv2D method}}

\begin{fulllineitems}
\phantomsection\label{\detokenize{beyondml.pt.layers:beyondml.pt.layers.MaskedConv2D.MaskedConv2D.forward}}
\pysigstartsignatures
\pysiglinewithargsret{\sphinxbfcode{\sphinxupquote{forward}}}{\emph{\DUrole{n}{inputs}}}{}
\pysigstopsignatures
\sphinxAtStartPar
Call the layer on input data
\begin{quote}\begin{description}
\sphinxlineitem{Parameters}
\sphinxAtStartPar
\sphinxstyleliteralstrong{\sphinxupquote{inputs}} (\sphinxstyleliteralemphasis{\sphinxupquote{torch.Tensor}}) \textendash{} Inputs to call the layer’s logic on

\sphinxlineitem{Returns}
\sphinxAtStartPar
\sphinxstylestrong{results} \textendash{} The results of the layer’s logic

\sphinxlineitem{Return type}
\sphinxAtStartPar
torch.Tensor

\end{description}\end{quote}

\end{fulllineitems}

\index{in\_channels (beyondml.pt.layers.MaskedConv2D.MaskedConv2D property)@\spxentry{in\_channels}\spxextra{beyondml.pt.layers.MaskedConv2D.MaskedConv2D property}}

\begin{fulllineitems}
\phantomsection\label{\detokenize{beyondml.pt.layers:beyondml.pt.layers.MaskedConv2D.MaskedConv2D.in_channels}}
\pysigstartsignatures
\pysigline{\sphinxbfcode{\sphinxupquote{property\DUrole{w}{  }}}\sphinxbfcode{\sphinxupquote{in\_channels}}}
\pysigstopsignatures
\end{fulllineitems}

\index{kernel\_size (beyondml.pt.layers.MaskedConv2D.MaskedConv2D property)@\spxentry{kernel\_size}\spxextra{beyondml.pt.layers.MaskedConv2D.MaskedConv2D property}}

\begin{fulllineitems}
\phantomsection\label{\detokenize{beyondml.pt.layers:beyondml.pt.layers.MaskedConv2D.MaskedConv2D.kernel_size}}
\pysigstartsignatures
\pysigline{\sphinxbfcode{\sphinxupquote{property\DUrole{w}{  }}}\sphinxbfcode{\sphinxupquote{kernel\_size}}}
\pysigstopsignatures
\end{fulllineitems}

\index{out\_channels (beyondml.pt.layers.MaskedConv2D.MaskedConv2D property)@\spxentry{out\_channels}\spxextra{beyondml.pt.layers.MaskedConv2D.MaskedConv2D property}}

\begin{fulllineitems}
\phantomsection\label{\detokenize{beyondml.pt.layers:beyondml.pt.layers.MaskedConv2D.MaskedConv2D.out_channels}}
\pysigstartsignatures
\pysigline{\sphinxbfcode{\sphinxupquote{property\DUrole{w}{  }}}\sphinxbfcode{\sphinxupquote{out\_channels}}}
\pysigstopsignatures
\end{fulllineitems}

\index{prune() (beyondml.pt.layers.MaskedConv2D.MaskedConv2D method)@\spxentry{prune()}\spxextra{beyondml.pt.layers.MaskedConv2D.MaskedConv2D method}}

\begin{fulllineitems}
\phantomsection\label{\detokenize{beyondml.pt.layers:beyondml.pt.layers.MaskedConv2D.MaskedConv2D.prune}}
\pysigstartsignatures
\pysiglinewithargsret{\sphinxbfcode{\sphinxupquote{prune}}}{\emph{\DUrole{n}{percentile}}}{}
\pysigstopsignatures
\sphinxAtStartPar
Prune the layer by updating the layer’s mask
\begin{quote}\begin{description}
\sphinxlineitem{Parameters}
\sphinxAtStartPar
\sphinxstyleliteralstrong{\sphinxupquote{percentile}} (\sphinxstyleliteralemphasis{\sphinxupquote{int}}) \textendash{} Integer between 0 and 99 which represents the proportion of weights to be inactive

\end{description}\end{quote}
\subsubsection*{Notes}

\sphinxAtStartPar
Acts on the layer in place

\end{fulllineitems}


\end{fulllineitems}



\subparagraph{beyondml.pt.layers.MaskedConv3D module}
\label{\detokenize{beyondml.pt.layers:module-beyondml.pt.layers.MaskedConv3D}}\label{\detokenize{beyondml.pt.layers:beyondml-pt-layers-maskedconv3d-module}}\index{module@\spxentry{module}!beyondml.pt.layers.MaskedConv3D@\spxentry{beyondml.pt.layers.MaskedConv3D}}\index{beyondml.pt.layers.MaskedConv3D@\spxentry{beyondml.pt.layers.MaskedConv3D}!module@\spxentry{module}}\index{MaskedConv3D (class in beyondml.pt.layers.MaskedConv3D)@\spxentry{MaskedConv3D}\spxextra{class in beyondml.pt.layers.MaskedConv3D}}

\begin{fulllineitems}
\phantomsection\label{\detokenize{beyondml.pt.layers:beyondml.pt.layers.MaskedConv3D.MaskedConv3D}}
\pysigstartsignatures
\pysiglinewithargsret{\sphinxbfcode{\sphinxupquote{class\DUrole{w}{  }}}\sphinxcode{\sphinxupquote{beyondml.pt.layers.MaskedConv3D.}}\sphinxbfcode{\sphinxupquote{MaskedConv3D}}}{\emph{\DUrole{n}{in\_channels}}, \emph{\DUrole{n}{out\_channels}}, \emph{\DUrole{n}{kernel\_size}\DUrole{o}{=}\DUrole{default_value}{3}}, \emph{\DUrole{n}{padding}\DUrole{o}{=}\DUrole{default_value}{\textquotesingle{}same\textquotesingle{}}}, \emph{\DUrole{n}{strides}\DUrole{o}{=}\DUrole{default_value}{1}}, \emph{\DUrole{n}{device}\DUrole{o}{=}\DUrole{default_value}{None}}, \emph{\DUrole{n}{dtype}\DUrole{o}{=}\DUrole{default_value}{None}}}{}
\pysigstopsignatures
\sphinxAtStartPar
Bases: \sphinxcode{\sphinxupquote{Module}}

\sphinxAtStartPar
Masked 3D Convolutional layer
\index{forward() (beyondml.pt.layers.MaskedConv3D.MaskedConv3D method)@\spxentry{forward()}\spxextra{beyondml.pt.layers.MaskedConv3D.MaskedConv3D method}}

\begin{fulllineitems}
\phantomsection\label{\detokenize{beyondml.pt.layers:beyondml.pt.layers.MaskedConv3D.MaskedConv3D.forward}}
\pysigstartsignatures
\pysiglinewithargsret{\sphinxbfcode{\sphinxupquote{forward}}}{\emph{\DUrole{n}{inputs}}}{}
\pysigstopsignatures
\sphinxAtStartPar
Call the layer on input data
\begin{quote}\begin{description}
\sphinxlineitem{Parameters}
\sphinxAtStartPar
\sphinxstyleliteralstrong{\sphinxupquote{inputs}} (\sphinxstyleliteralemphasis{\sphinxupquote{torch.Tensor}}) \textendash{} Inputs to call the layer’s logic on

\sphinxlineitem{Returns}
\sphinxAtStartPar
\sphinxstylestrong{results} \textendash{} The results of the layer’s logic

\sphinxlineitem{Return type}
\sphinxAtStartPar
torch.Tensor

\end{description}\end{quote}

\end{fulllineitems}

\index{in\_channels (beyondml.pt.layers.MaskedConv3D.MaskedConv3D property)@\spxentry{in\_channels}\spxextra{beyondml.pt.layers.MaskedConv3D.MaskedConv3D property}}

\begin{fulllineitems}
\phantomsection\label{\detokenize{beyondml.pt.layers:beyondml.pt.layers.MaskedConv3D.MaskedConv3D.in_channels}}
\pysigstartsignatures
\pysigline{\sphinxbfcode{\sphinxupquote{property\DUrole{w}{  }}}\sphinxbfcode{\sphinxupquote{in\_channels}}}
\pysigstopsignatures
\end{fulllineitems}

\index{kernel\_size (beyondml.pt.layers.MaskedConv3D.MaskedConv3D property)@\spxentry{kernel\_size}\spxextra{beyondml.pt.layers.MaskedConv3D.MaskedConv3D property}}

\begin{fulllineitems}
\phantomsection\label{\detokenize{beyondml.pt.layers:beyondml.pt.layers.MaskedConv3D.MaskedConv3D.kernel_size}}
\pysigstartsignatures
\pysigline{\sphinxbfcode{\sphinxupquote{property\DUrole{w}{  }}}\sphinxbfcode{\sphinxupquote{kernel\_size}}}
\pysigstopsignatures
\end{fulllineitems}

\index{out\_channels (beyondml.pt.layers.MaskedConv3D.MaskedConv3D property)@\spxentry{out\_channels}\spxextra{beyondml.pt.layers.MaskedConv3D.MaskedConv3D property}}

\begin{fulllineitems}
\phantomsection\label{\detokenize{beyondml.pt.layers:beyondml.pt.layers.MaskedConv3D.MaskedConv3D.out_channels}}
\pysigstartsignatures
\pysigline{\sphinxbfcode{\sphinxupquote{property\DUrole{w}{  }}}\sphinxbfcode{\sphinxupquote{out\_channels}}}
\pysigstopsignatures
\end{fulllineitems}

\index{prune() (beyondml.pt.layers.MaskedConv3D.MaskedConv3D method)@\spxentry{prune()}\spxextra{beyondml.pt.layers.MaskedConv3D.MaskedConv3D method}}

\begin{fulllineitems}
\phantomsection\label{\detokenize{beyondml.pt.layers:beyondml.pt.layers.MaskedConv3D.MaskedConv3D.prune}}
\pysigstartsignatures
\pysiglinewithargsret{\sphinxbfcode{\sphinxupquote{prune}}}{\emph{\DUrole{n}{percentile}}}{}
\pysigstopsignatures
\sphinxAtStartPar
Prune the layer by updating the layer’s masks
\begin{quote}\begin{description}
\sphinxlineitem{Parameters}
\sphinxAtStartPar
\sphinxstyleliteralstrong{\sphinxupquote{percentile}} (\sphinxstyleliteralemphasis{\sphinxupquote{int}}) \textendash{} Integer between 0 and 99 which represents the proportion of weights to be inactive

\end{description}\end{quote}
\subsubsection*{Notes}

\sphinxAtStartPar
Acts on the layer in place

\end{fulllineitems}


\end{fulllineitems}



\subparagraph{beyondml.pt.layers.MaskedDense module}
\label{\detokenize{beyondml.pt.layers:module-beyondml.pt.layers.MaskedDense}}\label{\detokenize{beyondml.pt.layers:beyondml-pt-layers-maskeddense-module}}\index{module@\spxentry{module}!beyondml.pt.layers.MaskedDense@\spxentry{beyondml.pt.layers.MaskedDense}}\index{beyondml.pt.layers.MaskedDense@\spxentry{beyondml.pt.layers.MaskedDense}!module@\spxentry{module}}\index{MaskedDense (class in beyondml.pt.layers.MaskedDense)@\spxentry{MaskedDense}\spxextra{class in beyondml.pt.layers.MaskedDense}}

\begin{fulllineitems}
\phantomsection\label{\detokenize{beyondml.pt.layers:beyondml.pt.layers.MaskedDense.MaskedDense}}
\pysigstartsignatures
\pysiglinewithargsret{\sphinxbfcode{\sphinxupquote{class\DUrole{w}{  }}}\sphinxcode{\sphinxupquote{beyondml.pt.layers.MaskedDense.}}\sphinxbfcode{\sphinxupquote{MaskedDense}}}{\emph{\DUrole{n}{in\_features}}, \emph{\DUrole{n}{out\_features}}, \emph{\DUrole{n}{device}\DUrole{o}{=}\DUrole{default_value}{None}}, \emph{\DUrole{n}{dtype}\DUrole{o}{=}\DUrole{default_value}{None}}}{}
\pysigstopsignatures
\sphinxAtStartPar
Bases: \sphinxcode{\sphinxupquote{Module}}

\sphinxAtStartPar
Masked fully\sphinxhyphen{}connected layer
\index{forward() (beyondml.pt.layers.MaskedDense.MaskedDense method)@\spxentry{forward()}\spxextra{beyondml.pt.layers.MaskedDense.MaskedDense method}}

\begin{fulllineitems}
\phantomsection\label{\detokenize{beyondml.pt.layers:beyondml.pt.layers.MaskedDense.MaskedDense.forward}}
\pysigstartsignatures
\pysiglinewithargsret{\sphinxbfcode{\sphinxupquote{forward}}}{\emph{\DUrole{n}{inputs}}}{}
\pysigstopsignatures
\sphinxAtStartPar
Call the layer on input data
\begin{quote}\begin{description}
\sphinxlineitem{Parameters}
\sphinxAtStartPar
\sphinxstyleliteralstrong{\sphinxupquote{inputs}} (\sphinxstyleliteralemphasis{\sphinxupquote{torch.Tensor}}) \textendash{} Inputs to call the layer’s logic on

\sphinxlineitem{Returns}
\sphinxAtStartPar
\sphinxstylestrong{results} \textendash{} The results of the layer’s logic

\sphinxlineitem{Return type}
\sphinxAtStartPar
torch.Tensor

\end{description}\end{quote}

\end{fulllineitems}

\index{prune() (beyondml.pt.layers.MaskedDense.MaskedDense method)@\spxentry{prune()}\spxextra{beyondml.pt.layers.MaskedDense.MaskedDense method}}

\begin{fulllineitems}
\phantomsection\label{\detokenize{beyondml.pt.layers:beyondml.pt.layers.MaskedDense.MaskedDense.prune}}
\pysigstartsignatures
\pysiglinewithargsret{\sphinxbfcode{\sphinxupquote{prune}}}{\emph{\DUrole{n}{percentile}}}{}
\pysigstopsignatures
\sphinxAtStartPar
Prune the layer by updating the layer’s mask
\begin{quote}\begin{description}
\sphinxlineitem{Parameters}
\sphinxAtStartPar
\sphinxstyleliteralstrong{\sphinxupquote{percentile}} (\sphinxstyleliteralemphasis{\sphinxupquote{int}}) \textendash{} Integer between 0 and 99 which represents the proportion of weights to be inactive

\end{description}\end{quote}
\subsubsection*{Notes}

\sphinxAtStartPar
Acts on the layer in place

\end{fulllineitems}


\end{fulllineitems}



\subparagraph{beyondml.pt.layers.MaskedMultiHeadAttention module}
\label{\detokenize{beyondml.pt.layers:module-beyondml.pt.layers.MaskedMultiHeadAttention}}\label{\detokenize{beyondml.pt.layers:beyondml-pt-layers-maskedmultiheadattention-module}}\index{module@\spxentry{module}!beyondml.pt.layers.MaskedMultiHeadAttention@\spxentry{beyondml.pt.layers.MaskedMultiHeadAttention}}\index{beyondml.pt.layers.MaskedMultiHeadAttention@\spxentry{beyondml.pt.layers.MaskedMultiHeadAttention}!module@\spxentry{module}}\index{MaskedMultiHeadAttention (class in beyondml.pt.layers.MaskedMultiHeadAttention)@\spxentry{MaskedMultiHeadAttention}\spxextra{class in beyondml.pt.layers.MaskedMultiHeadAttention}}

\begin{fulllineitems}
\phantomsection\label{\detokenize{beyondml.pt.layers:beyondml.pt.layers.MaskedMultiHeadAttention.MaskedMultiHeadAttention}}
\pysigstartsignatures
\pysiglinewithargsret{\sphinxbfcode{\sphinxupquote{class\DUrole{w}{  }}}\sphinxcode{\sphinxupquote{beyondml.pt.layers.MaskedMultiHeadAttention.}}\sphinxbfcode{\sphinxupquote{MaskedMultiHeadAttention}}}{\emph{\DUrole{n}{embed\_dim}}, \emph{\DUrole{n}{num\_heads}}, \emph{\DUrole{n}{dropout}\DUrole{o}{=}\DUrole{default_value}{0}}, \emph{\DUrole{n}{batch\_first}\DUrole{o}{=}\DUrole{default_value}{False}}, \emph{\DUrole{n}{device}\DUrole{o}{=}\DUrole{default_value}{None}}, \emph{\DUrole{n}{dtype}\DUrole{o}{=}\DUrole{default_value}{None}}}{}
\pysigstopsignatures
\sphinxAtStartPar
Bases: \sphinxcode{\sphinxupquote{Module}}

\sphinxAtStartPar
Masked Multi\sphinxhyphen{}Headed Attention Layer
\index{forward() (beyondml.pt.layers.MaskedMultiHeadAttention.MaskedMultiHeadAttention method)@\spxentry{forward()}\spxextra{beyondml.pt.layers.MaskedMultiHeadAttention.MaskedMultiHeadAttention method}}

\begin{fulllineitems}
\phantomsection\label{\detokenize{beyondml.pt.layers:beyondml.pt.layers.MaskedMultiHeadAttention.MaskedMultiHeadAttention.forward}}
\pysigstartsignatures
\pysiglinewithargsret{\sphinxbfcode{\sphinxupquote{forward}}}{\emph{\DUrole{n}{query}}, \emph{\DUrole{n}{key}}, \emph{\DUrole{n}{value}}, \emph{\DUrole{n}{key\_padding\_mask}\DUrole{o}{=}\DUrole{default_value}{None}}, \emph{\DUrole{n}{need\_weights}\DUrole{o}{=}\DUrole{default_value}{True}}, \emph{\DUrole{n}{attn\_mask}\DUrole{o}{=}\DUrole{default_value}{None}}, \emph{\DUrole{n}{average\_attn\_weights}\DUrole{o}{=}\DUrole{default_value}{True}}}{}
\pysigstopsignatures
\sphinxAtStartPar
Call the layer on input data
\begin{quote}\begin{description}
\sphinxlineitem{Parameters}\begin{itemize}
\item {} 
\sphinxAtStartPar
\sphinxstyleliteralstrong{\sphinxupquote{query}} (\sphinxstyleliteralemphasis{\sphinxupquote{torch Tensor}}) \textendash{} Query tensor

\item {} 
\sphinxAtStartPar
\sphinxstyleliteralstrong{\sphinxupquote{key}} (\sphinxstyleliteralemphasis{\sphinxupquote{torch Tensor}}) \textendash{} Key tensor

\item {} 
\sphinxAtStartPar
\sphinxstyleliteralstrong{\sphinxupquote{value}} (\sphinxstyleliteralemphasis{\sphinxupquote{torch Tensor}}) \textendash{} Value tensor

\item {} 
\sphinxAtStartPar
\sphinxstyleliteralstrong{\sphinxupquote{key\_padding\_mask}} (\sphinxstyleliteralemphasis{\sphinxupquote{None}}\sphinxstyleliteralemphasis{\sphinxupquote{ or }}\sphinxstyleliteralemphasis{\sphinxupquote{torch Tensor}}\sphinxstyleliteralemphasis{\sphinxupquote{ (}}\sphinxstyleliteralemphasis{\sphinxupquote{default None}}\sphinxstyleliteralemphasis{\sphinxupquote{)}}) \textendash{} If specified, a mask indicating which elements in \sphinxcode{\sphinxupquote{key}} to ignore

\item {} 
\sphinxAtStartPar
\sphinxstyleliteralstrong{\sphinxupquote{need\_weights}} (\sphinxstyleliteralemphasis{\sphinxupquote{Bool}}\sphinxstyleliteralemphasis{\sphinxupquote{ (}}\sphinxstyleliteralemphasis{\sphinxupquote{default True}}\sphinxstyleliteralemphasis{\sphinxupquote{)}}) \textendash{} If specified, returns \sphinxcode{\sphinxupquote{attn\_output\_weights}} as well as \sphinxcode{\sphinxupquote{attn\_outputs}}

\item {} 
\sphinxAtStartPar
\sphinxstyleliteralstrong{\sphinxupquote{attn\_mask}} (\sphinxstyleliteralemphasis{\sphinxupquote{None}}\sphinxstyleliteralemphasis{\sphinxupquote{ or }}\sphinxstyleliteralemphasis{\sphinxupquote{torch Tensor}}\sphinxstyleliteralemphasis{\sphinxupquote{ (}}\sphinxstyleliteralemphasis{\sphinxupquote{default None}}\sphinxstyleliteralemphasis{\sphinxupquote{)}}) \textendash{} If specified, a 2D or 3D mask preventing attention

\item {} 
\sphinxAtStartPar
\sphinxstyleliteralstrong{\sphinxupquote{average\_attn\_weights}} (\sphinxstyleliteralemphasis{\sphinxupquote{Bool}}\sphinxstyleliteralemphasis{\sphinxupquote{ (}}\sphinxstyleliteralemphasis{\sphinxupquote{default True}}\sphinxstyleliteralemphasis{\sphinxupquote{)}}) \textendash{} If True, indicates that returned \sphinxcode{\sphinxupquote{attn\_weights}} should be averaged across heads

\end{itemize}

\end{description}\end{quote}

\end{fulllineitems}

\index{prune() (beyondml.pt.layers.MaskedMultiHeadAttention.MaskedMultiHeadAttention method)@\spxentry{prune()}\spxextra{beyondml.pt.layers.MaskedMultiHeadAttention.MaskedMultiHeadAttention method}}

\begin{fulllineitems}
\phantomsection\label{\detokenize{beyondml.pt.layers:beyondml.pt.layers.MaskedMultiHeadAttention.MaskedMultiHeadAttention.prune}}
\pysigstartsignatures
\pysiglinewithargsret{\sphinxbfcode{\sphinxupquote{prune}}}{\emph{\DUrole{n}{percentile}}}{}
\pysigstopsignatures
\sphinxAtStartPar
Prune the layer by updating the layer’s mask
\begin{quote}\begin{description}
\sphinxlineitem{Parameters}
\sphinxAtStartPar
\sphinxstyleliteralstrong{\sphinxupquote{percentile}} (\sphinxstyleliteralemphasis{\sphinxupquote{int}}) \textendash{} Integer between 0 and 99 which represents the proportion of weights to be made inactive

\end{description}\end{quote}
\subsubsection*{Notes}

\sphinxAtStartPar
Acts on the layer in place

\end{fulllineitems}


\end{fulllineitems}



\subparagraph{beyondml.pt.layers.MaskedTransformerDecoderLayer module}
\label{\detokenize{beyondml.pt.layers:module-beyondml.pt.layers.MaskedTransformerDecoderLayer}}\label{\detokenize{beyondml.pt.layers:beyondml-pt-layers-maskedtransformerdecoderlayer-module}}\index{module@\spxentry{module}!beyondml.pt.layers.MaskedTransformerDecoderLayer@\spxentry{beyondml.pt.layers.MaskedTransformerDecoderLayer}}\index{beyondml.pt.layers.MaskedTransformerDecoderLayer@\spxentry{beyondml.pt.layers.MaskedTransformerDecoderLayer}!module@\spxentry{module}}\index{MaskedTransformerDecoderLayer (class in beyondml.pt.layers.MaskedTransformerDecoderLayer)@\spxentry{MaskedTransformerDecoderLayer}\spxextra{class in beyondml.pt.layers.MaskedTransformerDecoderLayer}}

\begin{fulllineitems}
\phantomsection\label{\detokenize{beyondml.pt.layers:beyondml.pt.layers.MaskedTransformerDecoderLayer.MaskedTransformerDecoderLayer}}
\pysigstartsignatures
\pysiglinewithargsret{\sphinxbfcode{\sphinxupquote{class\DUrole{w}{  }}}\sphinxcode{\sphinxupquote{beyondml.pt.layers.MaskedTransformerDecoderLayer.}}\sphinxbfcode{\sphinxupquote{MaskedTransformerDecoderLayer}}}{\emph{d\_model: int, nhead: int, dim\_feedforward: int = 2048, dropout: float = 0.1, activation: \textasciitilde{}typing.Union{[}str, \textasciitilde{}typing.Callable{[}{[}\textasciitilde{}torch.Tensor{]}, \textasciitilde{}torch.Tensor{]}{]} = \textless{}function relu\textgreater{}, layer\_norm\_eps: float = 1e\sphinxhyphen{}05, batch\_first: bool = False, norm\_first: bool = False, device=None, dtype=None}}{}
\pysigstopsignatures
\sphinxAtStartPar
Bases: \sphinxcode{\sphinxupquote{Module}}

\sphinxAtStartPar
TransformerDecoderLayer is made up of self\sphinxhyphen{}attn, multi\sphinxhyphen{}head\sphinxhyphen{}attn and feedforward network.
This standard decoder layer is based on the paper “Attention Is All You Need”.
:param d\_model: the number of expected features in the input (required).
:param nhead: the number of heads in the multiheadattention models (required).
:param dim\_feedforward: the dimension of the feedforward network model (default=2048).
:param dropout: the dropout value (default=0.1).
:param activation: the activation function of the intermediate layer, can be a string
\begin{quote}

\sphinxAtStartPar
(“relu” or “gelu”) or a unary callable. Default: relu
\end{quote}
\begin{quote}\begin{description}
\sphinxlineitem{Parameters}\begin{itemize}
\item {} 
\sphinxAtStartPar
\sphinxstyleliteralstrong{\sphinxupquote{layer\_norm\_eps}} \textendash{} the eps value in layer normalization components (default=1e\sphinxhyphen{}5).

\item {} 
\sphinxAtStartPar
\sphinxstyleliteralstrong{\sphinxupquote{batch\_first}} \textendash{} If \sphinxcode{\sphinxupquote{True}}, then the input and output tensors are provided
as (batch, seq, feature). Default: \sphinxcode{\sphinxupquote{False}} (seq, batch, feature).

\item {} 
\sphinxAtStartPar
\sphinxstyleliteralstrong{\sphinxupquote{norm\_first}} \textendash{} if \sphinxcode{\sphinxupquote{True}}, layer norm is done prior to self attention, multihead
attention and feedforward operations, respectively. Otherwise it’s done after.
Default: \sphinxcode{\sphinxupquote{False}} (after).

\end{itemize}

\end{description}\end{quote}
\index{forward() (beyondml.pt.layers.MaskedTransformerDecoderLayer.MaskedTransformerDecoderLayer method)@\spxentry{forward()}\spxextra{beyondml.pt.layers.MaskedTransformerDecoderLayer.MaskedTransformerDecoderLayer method}}

\begin{fulllineitems}
\phantomsection\label{\detokenize{beyondml.pt.layers:beyondml.pt.layers.MaskedTransformerDecoderLayer.MaskedTransformerDecoderLayer.forward}}
\pysigstartsignatures
\pysiglinewithargsret{\sphinxbfcode{\sphinxupquote{forward}}}{\emph{\DUrole{n}{tgt}\DUrole{p}{:}\DUrole{w}{  }\DUrole{n}{Tensor}}, \emph{\DUrole{n}{memory}\DUrole{p}{:}\DUrole{w}{  }\DUrole{n}{Tensor}}}{}
\pysigstopsignatures
\sphinxAtStartPar
Pass the inputs (and mask) through the decoder layer.
:param tgt: the sequence to the decoder layer.
:param memory: the sequence from the last layer of the encoder.
\begin{description}
\sphinxlineitem{Shape:}
\sphinxAtStartPar
see the docs in Pytorch Transformer class.

\end{description}

\end{fulllineitems}

\index{prune() (beyondml.pt.layers.MaskedTransformerDecoderLayer.MaskedTransformerDecoderLayer method)@\spxentry{prune()}\spxextra{beyondml.pt.layers.MaskedTransformerDecoderLayer.MaskedTransformerDecoderLayer method}}

\begin{fulllineitems}
\phantomsection\label{\detokenize{beyondml.pt.layers:beyondml.pt.layers.MaskedTransformerDecoderLayer.MaskedTransformerDecoderLayer.prune}}
\pysigstartsignatures
\pysiglinewithargsret{\sphinxbfcode{\sphinxupquote{prune}}}{\emph{\DUrole{n}{percentile}}}{}
\pysigstopsignatures
\end{fulllineitems}


\end{fulllineitems}



\subparagraph{beyondml.pt.layers.MaskedTransformerEncoderLayer module}
\label{\detokenize{beyondml.pt.layers:module-beyondml.pt.layers.MaskedTransformerEncoderLayer}}\label{\detokenize{beyondml.pt.layers:beyondml-pt-layers-maskedtransformerencoderlayer-module}}\index{module@\spxentry{module}!beyondml.pt.layers.MaskedTransformerEncoderLayer@\spxentry{beyondml.pt.layers.MaskedTransformerEncoderLayer}}\index{beyondml.pt.layers.MaskedTransformerEncoderLayer@\spxentry{beyondml.pt.layers.MaskedTransformerEncoderLayer}!module@\spxentry{module}}\index{MaskedTransformerEncoderLayer (class in beyondml.pt.layers.MaskedTransformerEncoderLayer)@\spxentry{MaskedTransformerEncoderLayer}\spxextra{class in beyondml.pt.layers.MaskedTransformerEncoderLayer}}

\begin{fulllineitems}
\phantomsection\label{\detokenize{beyondml.pt.layers:beyondml.pt.layers.MaskedTransformerEncoderLayer.MaskedTransformerEncoderLayer}}
\pysigstartsignatures
\pysiglinewithargsret{\sphinxbfcode{\sphinxupquote{class\DUrole{w}{  }}}\sphinxcode{\sphinxupquote{beyondml.pt.layers.MaskedTransformerEncoderLayer.}}\sphinxbfcode{\sphinxupquote{MaskedTransformerEncoderLayer}}}{\emph{d\_model: int, nhead: int, dim\_feedforward: int = 2048, dropout: float = 0.1, activation: \textasciitilde{}typing.Union{[}str, \textasciitilde{}typing.Callable{[}{[}\textasciitilde{}torch.Tensor{]}, \textasciitilde{}torch.Tensor{]}{]} = \textless{}function relu\textgreater{}, layer\_norm\_eps: float = 1e\sphinxhyphen{}05, batch\_first: bool = False, norm\_first: bool = False, device=None, dtype=None}}{}
\pysigstopsignatures
\sphinxAtStartPar
Bases: \sphinxcode{\sphinxupquote{Module}}

\sphinxAtStartPar
TransformerEncoderLayer is made up of self\sphinxhyphen{}attn and feedforward network.
:param d\_model: the number of expected features in the input (required).
:param nhead: the number of heads in the multiheadattention models (required).
:param dim\_feedforward: the dimension of the feedforward network model (default=2048).
:param dropout: the dropout value (default=0.1).
:param activation: the activation function of the intermediate layer, can be a string
\begin{quote}

\sphinxAtStartPar
(“relu” or “gelu”) or a unary callable. Default: relu
\end{quote}
\begin{quote}\begin{description}
\sphinxlineitem{Parameters}\begin{itemize}
\item {} 
\sphinxAtStartPar
\sphinxstyleliteralstrong{\sphinxupquote{layer\_norm\_eps}} \textendash{} the eps value in layer normalization components (default=1e\sphinxhyphen{}5).

\item {} 
\sphinxAtStartPar
\sphinxstyleliteralstrong{\sphinxupquote{batch\_first}} \textendash{} If \sphinxcode{\sphinxupquote{True}}, then the input and output tensors are provided
as (batch, seq, feature). Default: \sphinxcode{\sphinxupquote{False}} (seq, batch, feature).

\item {} 
\sphinxAtStartPar
\sphinxstyleliteralstrong{\sphinxupquote{norm\_first}} \textendash{} if \sphinxcode{\sphinxupquote{True}}, layer norm is done prior to attention and feedforward
operations, respectivaly. Otherwise it’s done after. Default: \sphinxcode{\sphinxupquote{False}} (after).

\end{itemize}

\end{description}\end{quote}
\index{forward() (beyondml.pt.layers.MaskedTransformerEncoderLayer.MaskedTransformerEncoderLayer method)@\spxentry{forward()}\spxextra{beyondml.pt.layers.MaskedTransformerEncoderLayer.MaskedTransformerEncoderLayer method}}

\begin{fulllineitems}
\phantomsection\label{\detokenize{beyondml.pt.layers:beyondml.pt.layers.MaskedTransformerEncoderLayer.MaskedTransformerEncoderLayer.forward}}
\pysigstartsignatures
\pysiglinewithargsret{\sphinxbfcode{\sphinxupquote{forward}}}{\emph{\DUrole{n}{src}\DUrole{p}{:}\DUrole{w}{  }\DUrole{n}{Tensor}}}{}
\pysigstopsignatures
\sphinxAtStartPar
Pass the input through the encoder layer.
:param src: the sequence to the encoder layer (required).

\end{fulllineitems}

\index{prune() (beyondml.pt.layers.MaskedTransformerEncoderLayer.MaskedTransformerEncoderLayer method)@\spxentry{prune()}\spxextra{beyondml.pt.layers.MaskedTransformerEncoderLayer.MaskedTransformerEncoderLayer method}}

\begin{fulllineitems}
\phantomsection\label{\detokenize{beyondml.pt.layers:beyondml.pt.layers.MaskedTransformerEncoderLayer.MaskedTransformerEncoderLayer.prune}}
\pysigstartsignatures
\pysiglinewithargsret{\sphinxbfcode{\sphinxupquote{prune}}}{\emph{\DUrole{n}{percentile}}}{}
\pysigstopsignatures
\end{fulllineitems}


\end{fulllineitems}



\subparagraph{beyondml.pt.layers.MultiConv2D module}
\label{\detokenize{beyondml.pt.layers:module-beyondml.pt.layers.MultiConv2D}}\label{\detokenize{beyondml.pt.layers:beyondml-pt-layers-multiconv2d-module}}\index{module@\spxentry{module}!beyondml.pt.layers.MultiConv2D@\spxentry{beyondml.pt.layers.MultiConv2D}}\index{beyondml.pt.layers.MultiConv2D@\spxentry{beyondml.pt.layers.MultiConv2D}!module@\spxentry{module}}\index{MultiConv2D (class in beyondml.pt.layers.MultiConv2D)@\spxentry{MultiConv2D}\spxextra{class in beyondml.pt.layers.MultiConv2D}}

\begin{fulllineitems}
\phantomsection\label{\detokenize{beyondml.pt.layers:beyondml.pt.layers.MultiConv2D.MultiConv2D}}
\pysigstartsignatures
\pysiglinewithargsret{\sphinxbfcode{\sphinxupquote{class\DUrole{w}{  }}}\sphinxcode{\sphinxupquote{beyondml.pt.layers.MultiConv2D.}}\sphinxbfcode{\sphinxupquote{MultiConv2D}}}{\emph{\DUrole{n}{kernel}}, \emph{\DUrole{n}{bias}}, \emph{\DUrole{n}{padding}\DUrole{o}{=}\DUrole{default_value}{\textquotesingle{}same\textquotesingle{}}}, \emph{\DUrole{n}{strides}\DUrole{o}{=}\DUrole{default_value}{1}}, \emph{\DUrole{n}{device}\DUrole{o}{=}\DUrole{default_value}{None}}, \emph{\DUrole{n}{dtype}\DUrole{o}{=}\DUrole{default_value}{None}}}{}
\pysigstopsignatures
\sphinxAtStartPar
Bases: \sphinxcode{\sphinxupquote{Module}}

\sphinxAtStartPar
Multi\sphinxhyphen{} 2D Convolutional layer initialized with weights rather than hyperparameters
\index{forward() (beyondml.pt.layers.MultiConv2D.MultiConv2D method)@\spxentry{forward()}\spxextra{beyondml.pt.layers.MultiConv2D.MultiConv2D method}}

\begin{fulllineitems}
\phantomsection\label{\detokenize{beyondml.pt.layers:beyondml.pt.layers.MultiConv2D.MultiConv2D.forward}}
\pysigstartsignatures
\pysiglinewithargsret{\sphinxbfcode{\sphinxupquote{forward}}}{\emph{\DUrole{n}{inputs}}}{}
\pysigstopsignatures
\sphinxAtStartPar
Call the layer on input data
\begin{quote}\begin{description}
\sphinxlineitem{Parameters}
\sphinxAtStartPar
\sphinxstyleliteralstrong{\sphinxupquote{inputs}} (\sphinxstyleliteralemphasis{\sphinxupquote{torch.Tensor}}) \textendash{} Inputs to call the layer’s logic on

\sphinxlineitem{Returns}
\sphinxAtStartPar
\sphinxstylestrong{results} \textendash{} The results of the layer’s logic

\sphinxlineitem{Return type}
\sphinxAtStartPar
torch.Tensor

\end{description}\end{quote}

\end{fulllineitems}


\end{fulllineitems}



\subparagraph{beyondml.pt.layers.MultiConv3D module}
\label{\detokenize{beyondml.pt.layers:module-beyondml.pt.layers.MultiConv3D}}\label{\detokenize{beyondml.pt.layers:beyondml-pt-layers-multiconv3d-module}}\index{module@\spxentry{module}!beyondml.pt.layers.MultiConv3D@\spxentry{beyondml.pt.layers.MultiConv3D}}\index{beyondml.pt.layers.MultiConv3D@\spxentry{beyondml.pt.layers.MultiConv3D}!module@\spxentry{module}}\index{MultiConv3D (class in beyondml.pt.layers.MultiConv3D)@\spxentry{MultiConv3D}\spxextra{class in beyondml.pt.layers.MultiConv3D}}

\begin{fulllineitems}
\phantomsection\label{\detokenize{beyondml.pt.layers:beyondml.pt.layers.MultiConv3D.MultiConv3D}}
\pysigstartsignatures
\pysiglinewithargsret{\sphinxbfcode{\sphinxupquote{class\DUrole{w}{  }}}\sphinxcode{\sphinxupquote{beyondml.pt.layers.MultiConv3D.}}\sphinxbfcode{\sphinxupquote{MultiConv3D}}}{\emph{\DUrole{n}{kernel}}, \emph{\DUrole{n}{bias}}, \emph{\DUrole{n}{padding}\DUrole{o}{=}\DUrole{default_value}{\textquotesingle{}same\textquotesingle{}}}, \emph{\DUrole{n}{strides}\DUrole{o}{=}\DUrole{default_value}{1}}, \emph{\DUrole{n}{device}\DUrole{o}{=}\DUrole{default_value}{None}}, \emph{\DUrole{n}{dtype}\DUrole{o}{=}\DUrole{default_value}{None}}}{}
\pysigstopsignatures
\sphinxAtStartPar
Bases: \sphinxcode{\sphinxupquote{Module}}

\sphinxAtStartPar
Multitask 3D Convolutional layer initialized with weights rather than with hyperparameters
\index{forward() (beyondml.pt.layers.MultiConv3D.MultiConv3D method)@\spxentry{forward()}\spxextra{beyondml.pt.layers.MultiConv3D.MultiConv3D method}}

\begin{fulllineitems}
\phantomsection\label{\detokenize{beyondml.pt.layers:beyondml.pt.layers.MultiConv3D.MultiConv3D.forward}}
\pysigstartsignatures
\pysiglinewithargsret{\sphinxbfcode{\sphinxupquote{forward}}}{\emph{\DUrole{n}{inputs}}}{}
\pysigstopsignatures
\sphinxAtStartPar
Call the layer on input data
\begin{quote}\begin{description}
\sphinxlineitem{Parameters}
\sphinxAtStartPar
\sphinxstyleliteralstrong{\sphinxupquote{inputs}} (\sphinxstyleliteralemphasis{\sphinxupquote{torch.Tensor}}) \textendash{} Inputs to call the layer’s logic on

\sphinxlineitem{Returns}
\sphinxAtStartPar
\sphinxstylestrong{results} \textendash{} The results of the layer’s logic

\sphinxlineitem{Return type}
\sphinxAtStartPar
torch.Tensor

\end{description}\end{quote}

\end{fulllineitems}


\end{fulllineitems}



\subparagraph{beyondml.pt.layers.MultiDense module}
\label{\detokenize{beyondml.pt.layers:module-beyondml.pt.layers.MultiDense}}\label{\detokenize{beyondml.pt.layers:beyondml-pt-layers-multidense-module}}\index{module@\spxentry{module}!beyondml.pt.layers.MultiDense@\spxentry{beyondml.pt.layers.MultiDense}}\index{beyondml.pt.layers.MultiDense@\spxentry{beyondml.pt.layers.MultiDense}!module@\spxentry{module}}\index{MultiDense (class in beyondml.pt.layers.MultiDense)@\spxentry{MultiDense}\spxextra{class in beyondml.pt.layers.MultiDense}}

\begin{fulllineitems}
\phantomsection\label{\detokenize{beyondml.pt.layers:beyondml.pt.layers.MultiDense.MultiDense}}
\pysigstartsignatures
\pysiglinewithargsret{\sphinxbfcode{\sphinxupquote{class\DUrole{w}{  }}}\sphinxcode{\sphinxupquote{beyondml.pt.layers.MultiDense.}}\sphinxbfcode{\sphinxupquote{MultiDense}}}{\emph{\DUrole{n}{weight}}, \emph{\DUrole{n}{bias}}, \emph{\DUrole{n}{device}\DUrole{o}{=}\DUrole{default_value}{None}}, \emph{\DUrole{n}{dtype}\DUrole{o}{=}\DUrole{default_value}{None}}}{}
\pysigstopsignatures
\sphinxAtStartPar
Bases: \sphinxcode{\sphinxupquote{Module}}

\sphinxAtStartPar
Multi\sphinxhyphen{}Fully\sphinxhyphen{}Connected layer initialized with weights rather than hyperparameters
\index{forward() (beyondml.pt.layers.MultiDense.MultiDense method)@\spxentry{forward()}\spxextra{beyondml.pt.layers.MultiDense.MultiDense method}}

\begin{fulllineitems}
\phantomsection\label{\detokenize{beyondml.pt.layers:beyondml.pt.layers.MultiDense.MultiDense.forward}}
\pysigstartsignatures
\pysiglinewithargsret{\sphinxbfcode{\sphinxupquote{forward}}}{\emph{\DUrole{n}{inputs}}}{}
\pysigstopsignatures
\sphinxAtStartPar
Call the layer on input data
\begin{quote}\begin{description}
\sphinxlineitem{Parameters}
\sphinxAtStartPar
\sphinxstyleliteralstrong{\sphinxupquote{inputs}} (\sphinxstyleliteralemphasis{\sphinxupquote{torch.Tensor}}) \textendash{} Inputs to call the layer’s logic on

\sphinxlineitem{Returns}
\sphinxAtStartPar
\sphinxstylestrong{results} \textendash{} The results of the layer’s logic

\sphinxlineitem{Return type}
\sphinxAtStartPar
torch.Tensor

\end{description}\end{quote}

\end{fulllineitems}


\end{fulllineitems}



\subparagraph{beyondml.pt.layers.MultiMaskedConv2D module}
\label{\detokenize{beyondml.pt.layers:module-beyondml.pt.layers.MultiMaskedConv2D}}\label{\detokenize{beyondml.pt.layers:beyondml-pt-layers-multimaskedconv2d-module}}\index{module@\spxentry{module}!beyondml.pt.layers.MultiMaskedConv2D@\spxentry{beyondml.pt.layers.MultiMaskedConv2D}}\index{beyondml.pt.layers.MultiMaskedConv2D@\spxentry{beyondml.pt.layers.MultiMaskedConv2D}!module@\spxentry{module}}\index{MultiMaskedConv2D (class in beyondml.pt.layers.MultiMaskedConv2D)@\spxentry{MultiMaskedConv2D}\spxextra{class in beyondml.pt.layers.MultiMaskedConv2D}}

\begin{fulllineitems}
\phantomsection\label{\detokenize{beyondml.pt.layers:beyondml.pt.layers.MultiMaskedConv2D.MultiMaskedConv2D}}
\pysigstartsignatures
\pysiglinewithargsret{\sphinxbfcode{\sphinxupquote{class\DUrole{w}{  }}}\sphinxcode{\sphinxupquote{beyondml.pt.layers.MultiMaskedConv2D.}}\sphinxbfcode{\sphinxupquote{MultiMaskedConv2D}}}{\emph{\DUrole{n}{in\_channels}}, \emph{\DUrole{n}{out\_channels}}, \emph{\DUrole{n}{num\_tasks}}, \emph{\DUrole{n}{kernel\_size}\DUrole{o}{=}\DUrole{default_value}{3}}, \emph{\DUrole{n}{padding}\DUrole{o}{=}\DUrole{default_value}{\textquotesingle{}same\textquotesingle{}}}, \emph{\DUrole{n}{strides}\DUrole{o}{=}\DUrole{default_value}{1}}, \emph{\DUrole{n}{device}\DUrole{o}{=}\DUrole{default_value}{None}}, \emph{\DUrole{n}{dtype}\DUrole{o}{=}\DUrole{default_value}{None}}}{}
\pysigstopsignatures
\sphinxAtStartPar
Bases: \sphinxcode{\sphinxupquote{Module}}

\sphinxAtStartPar
Multi 2D Convolutional layer which supports masking and pruning
\index{forward() (beyondml.pt.layers.MultiMaskedConv2D.MultiMaskedConv2D method)@\spxentry{forward()}\spxextra{beyondml.pt.layers.MultiMaskedConv2D.MultiMaskedConv2D method}}

\begin{fulllineitems}
\phantomsection\label{\detokenize{beyondml.pt.layers:beyondml.pt.layers.MultiMaskedConv2D.MultiMaskedConv2D.forward}}
\pysigstartsignatures
\pysiglinewithargsret{\sphinxbfcode{\sphinxupquote{forward}}}{\emph{\DUrole{n}{inputs}}}{}
\pysigstopsignatures
\sphinxAtStartPar
Call the layer on input data
\begin{quote}\begin{description}
\sphinxlineitem{Parameters}
\sphinxAtStartPar
\sphinxstyleliteralstrong{\sphinxupquote{inputs}} (\sphinxstyleliteralemphasis{\sphinxupquote{torch.Tensor}}) \textendash{} Inputs to call the layer’s logic on

\sphinxlineitem{Returns}
\sphinxAtStartPar
\sphinxstylestrong{results} \textendash{} The results of the layer’s logic

\sphinxlineitem{Return type}
\sphinxAtStartPar
torch.Tensor

\end{description}\end{quote}

\end{fulllineitems}

\index{in\_channels (beyondml.pt.layers.MultiMaskedConv2D.MultiMaskedConv2D property)@\spxentry{in\_channels}\spxextra{beyondml.pt.layers.MultiMaskedConv2D.MultiMaskedConv2D property}}

\begin{fulllineitems}
\phantomsection\label{\detokenize{beyondml.pt.layers:beyondml.pt.layers.MultiMaskedConv2D.MultiMaskedConv2D.in_channels}}
\pysigstartsignatures
\pysigline{\sphinxbfcode{\sphinxupquote{property\DUrole{w}{  }}}\sphinxbfcode{\sphinxupquote{in\_channels}}}
\pysigstopsignatures
\end{fulllineitems}

\index{kernel\_size (beyondml.pt.layers.MultiMaskedConv2D.MultiMaskedConv2D property)@\spxentry{kernel\_size}\spxextra{beyondml.pt.layers.MultiMaskedConv2D.MultiMaskedConv2D property}}

\begin{fulllineitems}
\phantomsection\label{\detokenize{beyondml.pt.layers:beyondml.pt.layers.MultiMaskedConv2D.MultiMaskedConv2D.kernel_size}}
\pysigstartsignatures
\pysigline{\sphinxbfcode{\sphinxupquote{property\DUrole{w}{  }}}\sphinxbfcode{\sphinxupquote{kernel\_size}}}
\pysigstopsignatures
\end{fulllineitems}

\index{out\_channels (beyondml.pt.layers.MultiMaskedConv2D.MultiMaskedConv2D property)@\spxentry{out\_channels}\spxextra{beyondml.pt.layers.MultiMaskedConv2D.MultiMaskedConv2D property}}

\begin{fulllineitems}
\phantomsection\label{\detokenize{beyondml.pt.layers:beyondml.pt.layers.MultiMaskedConv2D.MultiMaskedConv2D.out_channels}}
\pysigstartsignatures
\pysigline{\sphinxbfcode{\sphinxupquote{property\DUrole{w}{  }}}\sphinxbfcode{\sphinxupquote{out\_channels}}}
\pysigstopsignatures
\end{fulllineitems}

\index{prune() (beyondml.pt.layers.MultiMaskedConv2D.MultiMaskedConv2D method)@\spxentry{prune()}\spxextra{beyondml.pt.layers.MultiMaskedConv2D.MultiMaskedConv2D method}}

\begin{fulllineitems}
\phantomsection\label{\detokenize{beyondml.pt.layers:beyondml.pt.layers.MultiMaskedConv2D.MultiMaskedConv2D.prune}}
\pysigstartsignatures
\pysiglinewithargsret{\sphinxbfcode{\sphinxupquote{prune}}}{\emph{\DUrole{n}{percentile}}}{}
\pysigstopsignatures
\sphinxAtStartPar
Prune the layer by updating the layer’s mask
\begin{quote}\begin{description}
\sphinxlineitem{Parameters}
\sphinxAtStartPar
\sphinxstyleliteralstrong{\sphinxupquote{percentile}} (\sphinxstyleliteralemphasis{\sphinxupquote{int}}) \textendash{} Integer between 0 and 99 which represents the proportion of weights to be inactive

\end{description}\end{quote}
\subsubsection*{Notes}

\sphinxAtStartPar
Acts on the layer in place

\end{fulllineitems}


\end{fulllineitems}



\subparagraph{beyondml.pt.layers.MultiMaskedConv3D module}
\label{\detokenize{beyondml.pt.layers:module-beyondml.pt.layers.MultiMaskedConv3D}}\label{\detokenize{beyondml.pt.layers:beyondml-pt-layers-multimaskedconv3d-module}}\index{module@\spxentry{module}!beyondml.pt.layers.MultiMaskedConv3D@\spxentry{beyondml.pt.layers.MultiMaskedConv3D}}\index{beyondml.pt.layers.MultiMaskedConv3D@\spxentry{beyondml.pt.layers.MultiMaskedConv3D}!module@\spxentry{module}}\index{MultiMaskedConv3D (class in beyondml.pt.layers.MultiMaskedConv3D)@\spxentry{MultiMaskedConv3D}\spxextra{class in beyondml.pt.layers.MultiMaskedConv3D}}

\begin{fulllineitems}
\phantomsection\label{\detokenize{beyondml.pt.layers:beyondml.pt.layers.MultiMaskedConv3D.MultiMaskedConv3D}}
\pysigstartsignatures
\pysiglinewithargsret{\sphinxbfcode{\sphinxupquote{class\DUrole{w}{  }}}\sphinxcode{\sphinxupquote{beyondml.pt.layers.MultiMaskedConv3D.}}\sphinxbfcode{\sphinxupquote{MultiMaskedConv3D}}}{\emph{\DUrole{n}{in\_channels}}, \emph{\DUrole{n}{out\_channels}}, \emph{\DUrole{n}{num\_tasks}}, \emph{\DUrole{n}{kernel\_size}\DUrole{o}{=}\DUrole{default_value}{3}}, \emph{\DUrole{n}{padding}\DUrole{o}{=}\DUrole{default_value}{\textquotesingle{}same\textquotesingle{}}}, \emph{\DUrole{n}{strides}\DUrole{o}{=}\DUrole{default_value}{1}}, \emph{\DUrole{n}{device}\DUrole{o}{=}\DUrole{default_value}{None}}, \emph{\DUrole{n}{dtype}\DUrole{o}{=}\DUrole{default_value}{None}}}{}
\pysigstopsignatures
\sphinxAtStartPar
Bases: \sphinxcode{\sphinxupquote{Module}}

\sphinxAtStartPar
Masked Multitask 3D Convolutional layer
\index{forward() (beyondml.pt.layers.MultiMaskedConv3D.MultiMaskedConv3D method)@\spxentry{forward()}\spxextra{beyondml.pt.layers.MultiMaskedConv3D.MultiMaskedConv3D method}}

\begin{fulllineitems}
\phantomsection\label{\detokenize{beyondml.pt.layers:beyondml.pt.layers.MultiMaskedConv3D.MultiMaskedConv3D.forward}}
\pysigstartsignatures
\pysiglinewithargsret{\sphinxbfcode{\sphinxupquote{forward}}}{\emph{\DUrole{n}{inputs}}}{}
\pysigstopsignatures
\sphinxAtStartPar
Call the layer on input data
\begin{quote}\begin{description}
\sphinxlineitem{Parameters}
\sphinxAtStartPar
\sphinxstyleliteralstrong{\sphinxupquote{inputs}} (\sphinxstyleliteralemphasis{\sphinxupquote{torch.Tensor}}) \textendash{} Inputs to call the layer’s logic on

\sphinxlineitem{Returns}
\sphinxAtStartPar
\sphinxstylestrong{results} \textendash{} The results of the layer’s logic

\sphinxlineitem{Return type}
\sphinxAtStartPar
torch.Tensor

\end{description}\end{quote}

\end{fulllineitems}

\index{in\_channels (beyondml.pt.layers.MultiMaskedConv3D.MultiMaskedConv3D property)@\spxentry{in\_channels}\spxextra{beyondml.pt.layers.MultiMaskedConv3D.MultiMaskedConv3D property}}

\begin{fulllineitems}
\phantomsection\label{\detokenize{beyondml.pt.layers:beyondml.pt.layers.MultiMaskedConv3D.MultiMaskedConv3D.in_channels}}
\pysigstartsignatures
\pysigline{\sphinxbfcode{\sphinxupquote{property\DUrole{w}{  }}}\sphinxbfcode{\sphinxupquote{in\_channels}}}
\pysigstopsignatures
\end{fulllineitems}

\index{kernel\_size (beyondml.pt.layers.MultiMaskedConv3D.MultiMaskedConv3D property)@\spxentry{kernel\_size}\spxextra{beyondml.pt.layers.MultiMaskedConv3D.MultiMaskedConv3D property}}

\begin{fulllineitems}
\phantomsection\label{\detokenize{beyondml.pt.layers:beyondml.pt.layers.MultiMaskedConv3D.MultiMaskedConv3D.kernel_size}}
\pysigstartsignatures
\pysigline{\sphinxbfcode{\sphinxupquote{property\DUrole{w}{  }}}\sphinxbfcode{\sphinxupquote{kernel\_size}}}
\pysigstopsignatures
\end{fulllineitems}

\index{out\_channels (beyondml.pt.layers.MultiMaskedConv3D.MultiMaskedConv3D property)@\spxentry{out\_channels}\spxextra{beyondml.pt.layers.MultiMaskedConv3D.MultiMaskedConv3D property}}

\begin{fulllineitems}
\phantomsection\label{\detokenize{beyondml.pt.layers:beyondml.pt.layers.MultiMaskedConv3D.MultiMaskedConv3D.out_channels}}
\pysigstartsignatures
\pysigline{\sphinxbfcode{\sphinxupquote{property\DUrole{w}{  }}}\sphinxbfcode{\sphinxupquote{out\_channels}}}
\pysigstopsignatures
\end{fulllineitems}

\index{prune() (beyondml.pt.layers.MultiMaskedConv3D.MultiMaskedConv3D method)@\spxentry{prune()}\spxextra{beyondml.pt.layers.MultiMaskedConv3D.MultiMaskedConv3D method}}

\begin{fulllineitems}
\phantomsection\label{\detokenize{beyondml.pt.layers:beyondml.pt.layers.MultiMaskedConv3D.MultiMaskedConv3D.prune}}
\pysigstartsignatures
\pysiglinewithargsret{\sphinxbfcode{\sphinxupquote{prune}}}{\emph{\DUrole{n}{percentile}}}{}
\pysigstopsignatures
\sphinxAtStartPar
Prune the layer by updating the layer’s masks
\begin{quote}\begin{description}
\sphinxlineitem{Parameters}
\sphinxAtStartPar
\sphinxstyleliteralstrong{\sphinxupquote{percentile}} (\sphinxstyleliteralemphasis{\sphinxupquote{int}}) \textendash{} Integer between 0 and 99 which represents the proportion of weights to be inactive

\end{description}\end{quote}
\subsubsection*{Notes}

\sphinxAtStartPar
Acts on the layer in place

\end{fulllineitems}


\end{fulllineitems}



\subparagraph{beyondml.pt.layers.MultiMaskedDense module}
\label{\detokenize{beyondml.pt.layers:module-beyondml.pt.layers.MultiMaskedDense}}\label{\detokenize{beyondml.pt.layers:beyondml-pt-layers-multimaskeddense-module}}\index{module@\spxentry{module}!beyondml.pt.layers.MultiMaskedDense@\spxentry{beyondml.pt.layers.MultiMaskedDense}}\index{beyondml.pt.layers.MultiMaskedDense@\spxentry{beyondml.pt.layers.MultiMaskedDense}!module@\spxentry{module}}\index{MultiMaskedDense (class in beyondml.pt.layers.MultiMaskedDense)@\spxentry{MultiMaskedDense}\spxextra{class in beyondml.pt.layers.MultiMaskedDense}}

\begin{fulllineitems}
\phantomsection\label{\detokenize{beyondml.pt.layers:beyondml.pt.layers.MultiMaskedDense.MultiMaskedDense}}
\pysigstartsignatures
\pysiglinewithargsret{\sphinxbfcode{\sphinxupquote{class\DUrole{w}{  }}}\sphinxcode{\sphinxupquote{beyondml.pt.layers.MultiMaskedDense.}}\sphinxbfcode{\sphinxupquote{MultiMaskedDense}}}{\emph{\DUrole{n}{in\_features}}, \emph{\DUrole{n}{out\_features}}, \emph{\DUrole{n}{num\_tasks}}, \emph{\DUrole{n}{device}\DUrole{o}{=}\DUrole{default_value}{None}}, \emph{\DUrole{n}{dtype}\DUrole{o}{=}\DUrole{default_value}{None}}}{}
\pysigstopsignatures
\sphinxAtStartPar
Bases: \sphinxcode{\sphinxupquote{Module}}

\sphinxAtStartPar
Multi\sphinxhyphen{}Fully\sphinxhyphen{}Connected layer which supports masking and pruning
\index{forward() (beyondml.pt.layers.MultiMaskedDense.MultiMaskedDense method)@\spxentry{forward()}\spxextra{beyondml.pt.layers.MultiMaskedDense.MultiMaskedDense method}}

\begin{fulllineitems}
\phantomsection\label{\detokenize{beyondml.pt.layers:beyondml.pt.layers.MultiMaskedDense.MultiMaskedDense.forward}}
\pysigstartsignatures
\pysiglinewithargsret{\sphinxbfcode{\sphinxupquote{forward}}}{\emph{\DUrole{n}{inputs}}}{}
\pysigstopsignatures
\sphinxAtStartPar
Call the layer on input data
\begin{quote}\begin{description}
\sphinxlineitem{Parameters}
\sphinxAtStartPar
\sphinxstyleliteralstrong{\sphinxupquote{inputs}} (\sphinxstyleliteralemphasis{\sphinxupquote{torch.Tensor}}) \textendash{} Inputs to call the layer’s logic on

\sphinxlineitem{Returns}
\sphinxAtStartPar
\sphinxstylestrong{results} \textendash{} The results of the layer’s logic

\sphinxlineitem{Return type}
\sphinxAtStartPar
torch.Tensor

\end{description}\end{quote}

\end{fulllineitems}

\index{prune() (beyondml.pt.layers.MultiMaskedDense.MultiMaskedDense method)@\spxentry{prune()}\spxextra{beyondml.pt.layers.MultiMaskedDense.MultiMaskedDense method}}

\begin{fulllineitems}
\phantomsection\label{\detokenize{beyondml.pt.layers:beyondml.pt.layers.MultiMaskedDense.MultiMaskedDense.prune}}
\pysigstartsignatures
\pysiglinewithargsret{\sphinxbfcode{\sphinxupquote{prune}}}{\emph{\DUrole{n}{percentile}}}{}
\pysigstopsignatures
\sphinxAtStartPar
Prune the layer by updating the layer’s mask
\begin{quote}\begin{description}
\sphinxlineitem{Parameters}
\sphinxAtStartPar
\sphinxstyleliteralstrong{\sphinxupquote{percentile}} (\sphinxstyleliteralemphasis{\sphinxupquote{int}}) \textendash{} Integer between 0 and 99 which represents the proportion of weights to be inactive

\end{description}\end{quote}
\subsubsection*{Notes}

\sphinxAtStartPar
Acts on the layer in place

\end{fulllineitems}


\end{fulllineitems}



\subparagraph{beyondml.pt.layers.MultiMaxPool2D module}
\label{\detokenize{beyondml.pt.layers:module-beyondml.pt.layers.MultiMaxPool2D}}\label{\detokenize{beyondml.pt.layers:beyondml-pt-layers-multimaxpool2d-module}}\index{module@\spxentry{module}!beyondml.pt.layers.MultiMaxPool2D@\spxentry{beyondml.pt.layers.MultiMaxPool2D}}\index{beyondml.pt.layers.MultiMaxPool2D@\spxentry{beyondml.pt.layers.MultiMaxPool2D}!module@\spxentry{module}}\index{MultiMaxPool2D (class in beyondml.pt.layers.MultiMaxPool2D)@\spxentry{MultiMaxPool2D}\spxextra{class in beyondml.pt.layers.MultiMaxPool2D}}

\begin{fulllineitems}
\phantomsection\label{\detokenize{beyondml.pt.layers:beyondml.pt.layers.MultiMaxPool2D.MultiMaxPool2D}}
\pysigstartsignatures
\pysiglinewithargsret{\sphinxbfcode{\sphinxupquote{class\DUrole{w}{  }}}\sphinxcode{\sphinxupquote{beyondml.pt.layers.MultiMaxPool2D.}}\sphinxbfcode{\sphinxupquote{MultiMaxPool2D}}}{\emph{\DUrole{n}{kernel\_size}}, \emph{\DUrole{n}{stride}\DUrole{o}{=}\DUrole{default_value}{None}}, \emph{\DUrole{n}{padding}\DUrole{o}{=}\DUrole{default_value}{0}}, \emph{\DUrole{n}{dilation}\DUrole{o}{=}\DUrole{default_value}{1}}}{}
\pysigstopsignatures
\sphinxAtStartPar
Bases: \sphinxcode{\sphinxupquote{Module}}

\sphinxAtStartPar
Multitask implementation of 2\sphinxhyphen{}dimensional Max Pooling layer
\index{forward() (beyondml.pt.layers.MultiMaxPool2D.MultiMaxPool2D method)@\spxentry{forward()}\spxextra{beyondml.pt.layers.MultiMaxPool2D.MultiMaxPool2D method}}

\begin{fulllineitems}
\phantomsection\label{\detokenize{beyondml.pt.layers:beyondml.pt.layers.MultiMaxPool2D.MultiMaxPool2D.forward}}
\pysigstartsignatures
\pysiglinewithargsret{\sphinxbfcode{\sphinxupquote{forward}}}{\emph{\DUrole{n}{inputs}}}{}
\pysigstopsignatures
\sphinxAtStartPar
Call the layer on input data
\begin{quote}\begin{description}
\sphinxlineitem{Parameters}
\sphinxAtStartPar
\sphinxstyleliteralstrong{\sphinxupquote{inputs}} (\sphinxstyleliteralemphasis{\sphinxupquote{torch.Tensor}}) \textendash{} Inputs to call the layer’s logic on

\sphinxlineitem{Returns}
\sphinxAtStartPar
\sphinxstylestrong{results} \textendash{} The results of the layer’s logic

\sphinxlineitem{Return type}
\sphinxAtStartPar
torch.Tensor

\end{description}\end{quote}

\end{fulllineitems}


\end{fulllineitems}



\subparagraph{beyondml.pt.layers.MultiMaxPool3D module}
\label{\detokenize{beyondml.pt.layers:module-beyondml.pt.layers.MultiMaxPool3D}}\label{\detokenize{beyondml.pt.layers:beyondml-pt-layers-multimaxpool3d-module}}\index{module@\spxentry{module}!beyondml.pt.layers.MultiMaxPool3D@\spxentry{beyondml.pt.layers.MultiMaxPool3D}}\index{beyondml.pt.layers.MultiMaxPool3D@\spxentry{beyondml.pt.layers.MultiMaxPool3D}!module@\spxentry{module}}\index{MultiMaxPool3D (class in beyondml.pt.layers.MultiMaxPool3D)@\spxentry{MultiMaxPool3D}\spxextra{class in beyondml.pt.layers.MultiMaxPool3D}}

\begin{fulllineitems}
\phantomsection\label{\detokenize{beyondml.pt.layers:beyondml.pt.layers.MultiMaxPool3D.MultiMaxPool3D}}
\pysigstartsignatures
\pysiglinewithargsret{\sphinxbfcode{\sphinxupquote{class\DUrole{w}{  }}}\sphinxcode{\sphinxupquote{beyondml.pt.layers.MultiMaxPool3D.}}\sphinxbfcode{\sphinxupquote{MultiMaxPool3D}}}{\emph{\DUrole{n}{kernel\_size}}, \emph{\DUrole{n}{stride}\DUrole{o}{=}\DUrole{default_value}{None}}, \emph{\DUrole{n}{padding}\DUrole{o}{=}\DUrole{default_value}{0}}, \emph{\DUrole{n}{dilation}\DUrole{o}{=}\DUrole{default_value}{1}}}{}
\pysigstopsignatures
\sphinxAtStartPar
Bases: \sphinxcode{\sphinxupquote{Module}}

\sphinxAtStartPar
Multitask implementation of 2\sphinxhyphen{}dimensional Max Pooling layer
\index{forward() (beyondml.pt.layers.MultiMaxPool3D.MultiMaxPool3D method)@\spxentry{forward()}\spxextra{beyondml.pt.layers.MultiMaxPool3D.MultiMaxPool3D method}}

\begin{fulllineitems}
\phantomsection\label{\detokenize{beyondml.pt.layers:beyondml.pt.layers.MultiMaxPool3D.MultiMaxPool3D.forward}}
\pysigstartsignatures
\pysiglinewithargsret{\sphinxbfcode{\sphinxupquote{forward}}}{\emph{\DUrole{n}{inputs}}}{}
\pysigstopsignatures
\sphinxAtStartPar
Call the layer on input data
\begin{quote}\begin{description}
\sphinxlineitem{Parameters}
\sphinxAtStartPar
\sphinxstyleliteralstrong{\sphinxupquote{inputs}} (\sphinxstyleliteralemphasis{\sphinxupquote{torch.Tensor}}) \textendash{} Inputs to call the layer’s logic on

\sphinxlineitem{Returns}
\sphinxAtStartPar
\sphinxstylestrong{results} \textendash{} The results of the layer’s logic

\sphinxlineitem{Return type}
\sphinxAtStartPar
torch.Tensor

\end{description}\end{quote}

\end{fulllineitems}


\end{fulllineitems}



\subparagraph{beyondml.pt.layers.MultitaskNormalization module}
\label{\detokenize{beyondml.pt.layers:module-beyondml.pt.layers.MultitaskNormalization}}\label{\detokenize{beyondml.pt.layers:beyondml-pt-layers-multitasknormalization-module}}\index{module@\spxentry{module}!beyondml.pt.layers.MultitaskNormalization@\spxentry{beyondml.pt.layers.MultitaskNormalization}}\index{beyondml.pt.layers.MultitaskNormalization@\spxentry{beyondml.pt.layers.MultitaskNormalization}!module@\spxentry{module}}\index{MultitaskNormalization (class in beyondml.pt.layers.MultitaskNormalization)@\spxentry{MultitaskNormalization}\spxextra{class in beyondml.pt.layers.MultitaskNormalization}}

\begin{fulllineitems}
\phantomsection\label{\detokenize{beyondml.pt.layers:beyondml.pt.layers.MultitaskNormalization.MultitaskNormalization}}
\pysigstartsignatures
\pysiglinewithargsret{\sphinxbfcode{\sphinxupquote{class\DUrole{w}{  }}}\sphinxcode{\sphinxupquote{beyondml.pt.layers.MultitaskNormalization.}}\sphinxbfcode{\sphinxupquote{MultitaskNormalization}}}{\emph{\DUrole{n}{device}\DUrole{o}{=}\DUrole{default_value}{None}}, \emph{\DUrole{n}{dtype}\DUrole{o}{=}\DUrole{default_value}{None}}}{}
\pysigstopsignatures
\sphinxAtStartPar
Bases: \sphinxcode{\sphinxupquote{Module}}

\sphinxAtStartPar
Layer which normalizes a set of inputs to sum to 1
\index{forward() (beyondml.pt.layers.MultitaskNormalization.MultitaskNormalization method)@\spxentry{forward()}\spxextra{beyondml.pt.layers.MultitaskNormalization.MultitaskNormalization method}}

\begin{fulllineitems}
\phantomsection\label{\detokenize{beyondml.pt.layers:beyondml.pt.layers.MultitaskNormalization.MultitaskNormalization.forward}}
\pysigstartsignatures
\pysiglinewithargsret{\sphinxbfcode{\sphinxupquote{forward}}}{\emph{\DUrole{n}{inputs}}}{}
\pysigstopsignatures
\sphinxAtStartPar
Call the layer on input data
\begin{quote}\begin{description}
\sphinxlineitem{Parameters}
\sphinxAtStartPar
\sphinxstyleliteralstrong{\sphinxupquote{inputs}} (\sphinxstyleliteralemphasis{\sphinxupquote{torch.Tensor}}\sphinxstyleliteralemphasis{\sphinxupquote{ or }}\sphinxstyleliteralemphasis{\sphinxupquote{list}}\sphinxstyleliteralemphasis{\sphinxupquote{ of }}\sphinxstyleliteralemphasis{\sphinxupquote{Tensors}}) \textendash{} Inputs to call the layer’s logic on

\sphinxlineitem{Returns}
\sphinxAtStartPar
\sphinxstylestrong{results} \textendash{} The results of the layer’s logic

\sphinxlineitem{Return type}
\sphinxAtStartPar
torch.Tensor or list of Tensors

\end{description}\end{quote}

\end{fulllineitems}


\end{fulllineitems}



\subparagraph{beyondml.pt.layers.SelectorLayer module}
\label{\detokenize{beyondml.pt.layers:module-beyondml.pt.layers.SelectorLayer}}\label{\detokenize{beyondml.pt.layers:beyondml-pt-layers-selectorlayer-module}}\index{module@\spxentry{module}!beyondml.pt.layers.SelectorLayer@\spxentry{beyondml.pt.layers.SelectorLayer}}\index{beyondml.pt.layers.SelectorLayer@\spxentry{beyondml.pt.layers.SelectorLayer}!module@\spxentry{module}}\index{SelectorLayer (class in beyondml.pt.layers.SelectorLayer)@\spxentry{SelectorLayer}\spxextra{class in beyondml.pt.layers.SelectorLayer}}

\begin{fulllineitems}
\phantomsection\label{\detokenize{beyondml.pt.layers:beyondml.pt.layers.SelectorLayer.SelectorLayer}}
\pysigstartsignatures
\pysiglinewithargsret{\sphinxbfcode{\sphinxupquote{class\DUrole{w}{  }}}\sphinxcode{\sphinxupquote{beyondml.pt.layers.SelectorLayer.}}\sphinxbfcode{\sphinxupquote{SelectorLayer}}}{\emph{\DUrole{n}{sel\_index}}}{}
\pysigstopsignatures
\sphinxAtStartPar
Bases: \sphinxcode{\sphinxupquote{Module}}

\sphinxAtStartPar
Layer which selects an individual input based on index and only returns that one
\index{forward() (beyondml.pt.layers.SelectorLayer.SelectorLayer method)@\spxentry{forward()}\spxextra{beyondml.pt.layers.SelectorLayer.SelectorLayer method}}

\begin{fulllineitems}
\phantomsection\label{\detokenize{beyondml.pt.layers:beyondml.pt.layers.SelectorLayer.SelectorLayer.forward}}
\pysigstartsignatures
\pysiglinewithargsret{\sphinxbfcode{\sphinxupquote{forward}}}{\emph{\DUrole{n}{inputs}}}{}
\pysigstopsignatures
\sphinxAtStartPar
Call the layer on input data
\begin{quote}\begin{description}
\sphinxlineitem{Parameters}
\sphinxAtStartPar
\sphinxstyleliteralstrong{\sphinxupquote{inputs}} (\sphinxstyleliteralemphasis{\sphinxupquote{torch.Tensor}}) \textendash{} Inputs to call the layer’s logic on

\sphinxlineitem{Returns}
\sphinxAtStartPar
\sphinxstylestrong{results} \textendash{} The results of the layer’s logic

\sphinxlineitem{Return type}
\sphinxAtStartPar
torch.Tensor

\end{description}\end{quote}

\end{fulllineitems}

\index{sel\_index (beyondml.pt.layers.SelectorLayer.SelectorLayer property)@\spxentry{sel\_index}\spxextra{beyondml.pt.layers.SelectorLayer.SelectorLayer property}}

\begin{fulllineitems}
\phantomsection\label{\detokenize{beyondml.pt.layers:beyondml.pt.layers.SelectorLayer.SelectorLayer.sel_index}}
\pysigstartsignatures
\pysigline{\sphinxbfcode{\sphinxupquote{property\DUrole{w}{  }}}\sphinxbfcode{\sphinxupquote{sel\_index}}}
\pysigstopsignatures
\end{fulllineitems}


\end{fulllineitems}



\subparagraph{beyondml.pt.layers.SparseConv2D module}
\label{\detokenize{beyondml.pt.layers:module-beyondml.pt.layers.SparseConv2D}}\label{\detokenize{beyondml.pt.layers:beyondml-pt-layers-sparseconv2d-module}}\index{module@\spxentry{module}!beyondml.pt.layers.SparseConv2D@\spxentry{beyondml.pt.layers.SparseConv2D}}\index{beyondml.pt.layers.SparseConv2D@\spxentry{beyondml.pt.layers.SparseConv2D}!module@\spxentry{module}}\index{SparseConv2D (class in beyondml.pt.layers.SparseConv2D)@\spxentry{SparseConv2D}\spxextra{class in beyondml.pt.layers.SparseConv2D}}

\begin{fulllineitems}
\phantomsection\label{\detokenize{beyondml.pt.layers:beyondml.pt.layers.SparseConv2D.SparseConv2D}}
\pysigstartsignatures
\pysiglinewithargsret{\sphinxbfcode{\sphinxupquote{class\DUrole{w}{  }}}\sphinxcode{\sphinxupquote{beyondml.pt.layers.SparseConv2D.}}\sphinxbfcode{\sphinxupquote{SparseConv2D}}}{\emph{\DUrole{n}{kernel}}, \emph{\DUrole{n}{bias}}, \emph{\DUrole{n}{padding}\DUrole{o}{=}\DUrole{default_value}{\textquotesingle{}same\textquotesingle{}}}, \emph{\DUrole{n}{strides}\DUrole{o}{=}\DUrole{default_value}{1}}, \emph{\DUrole{n}{device}\DUrole{o}{=}\DUrole{default_value}{None}}, \emph{\DUrole{n}{dtype}\DUrole{o}{=}\DUrole{default_value}{None}}}{}
\pysigstopsignatures
\sphinxAtStartPar
Bases: \sphinxcode{\sphinxupquote{Module}}

\sphinxAtStartPar
Sparse implementation of a 2D Convolutional layer, expected to be converted from a
trained, pruned layer
\index{forward() (beyondml.pt.layers.SparseConv2D.SparseConv2D method)@\spxentry{forward()}\spxextra{beyondml.pt.layers.SparseConv2D.SparseConv2D method}}

\begin{fulllineitems}
\phantomsection\label{\detokenize{beyondml.pt.layers:beyondml.pt.layers.SparseConv2D.SparseConv2D.forward}}
\pysigstartsignatures
\pysiglinewithargsret{\sphinxbfcode{\sphinxupquote{forward}}}{\emph{\DUrole{n}{inputs}}}{}
\pysigstopsignatures
\sphinxAtStartPar
Call the layer on input data
\begin{quote}\begin{description}
\sphinxlineitem{Parameters}
\sphinxAtStartPar
\sphinxstyleliteralstrong{\sphinxupquote{inputs}} (\sphinxstyleliteralemphasis{\sphinxupquote{torch.Tensor}}) \textendash{} Inputs to call the layer’s logic on

\sphinxlineitem{Returns}
\sphinxAtStartPar
\sphinxstylestrong{results} \textendash{} The results of the layer’s logic

\sphinxlineitem{Return type}
\sphinxAtStartPar
torch.Tensor

\end{description}\end{quote}

\end{fulllineitems}


\end{fulllineitems}



\subparagraph{beyondml.pt.layers.SparseConv3D module}
\label{\detokenize{beyondml.pt.layers:module-beyondml.pt.layers.SparseConv3D}}\label{\detokenize{beyondml.pt.layers:beyondml-pt-layers-sparseconv3d-module}}\index{module@\spxentry{module}!beyondml.pt.layers.SparseConv3D@\spxentry{beyondml.pt.layers.SparseConv3D}}\index{beyondml.pt.layers.SparseConv3D@\spxentry{beyondml.pt.layers.SparseConv3D}!module@\spxentry{module}}\index{SparseConv3D (class in beyondml.pt.layers.SparseConv3D)@\spxentry{SparseConv3D}\spxextra{class in beyondml.pt.layers.SparseConv3D}}

\begin{fulllineitems}
\phantomsection\label{\detokenize{beyondml.pt.layers:beyondml.pt.layers.SparseConv3D.SparseConv3D}}
\pysigstartsignatures
\pysiglinewithargsret{\sphinxbfcode{\sphinxupquote{class\DUrole{w}{  }}}\sphinxcode{\sphinxupquote{beyondml.pt.layers.SparseConv3D.}}\sphinxbfcode{\sphinxupquote{SparseConv3D}}}{\emph{\DUrole{n}{kernel}}, \emph{\DUrole{n}{bias}}, \emph{\DUrole{n}{padding}\DUrole{o}{=}\DUrole{default_value}{\textquotesingle{}same\textquotesingle{}}}, \emph{\DUrole{n}{strides}\DUrole{o}{=}\DUrole{default_value}{1}}, \emph{\DUrole{n}{device}\DUrole{o}{=}\DUrole{default_value}{None}}, \emph{\DUrole{n}{dtype}\DUrole{o}{=}\DUrole{default_value}{None}}}{}
\pysigstopsignatures
\sphinxAtStartPar
Bases: \sphinxcode{\sphinxupquote{Module}}

\sphinxAtStartPar
Sparse 3D Convolutional layer, expected to be converted from a
trained, pruned layer
\index{forward() (beyondml.pt.layers.SparseConv3D.SparseConv3D method)@\spxentry{forward()}\spxextra{beyondml.pt.layers.SparseConv3D.SparseConv3D method}}

\begin{fulllineitems}
\phantomsection\label{\detokenize{beyondml.pt.layers:beyondml.pt.layers.SparseConv3D.SparseConv3D.forward}}
\pysigstartsignatures
\pysiglinewithargsret{\sphinxbfcode{\sphinxupquote{forward}}}{\emph{\DUrole{n}{inputs}}}{}
\pysigstopsignatures
\sphinxAtStartPar
Call the layer on input data
\begin{quote}\begin{description}
\sphinxlineitem{Parameters}
\sphinxAtStartPar
\sphinxstyleliteralstrong{\sphinxupquote{inputs}} (\sphinxstyleliteralemphasis{\sphinxupquote{torch.Tensor}}) \textendash{} Inputs to call the layer’s logic on

\sphinxlineitem{Returns}
\sphinxAtStartPar
\sphinxstylestrong{results} \textendash{} The results of the layer’s logic

\sphinxlineitem{Return type}
\sphinxAtStartPar
torch.Tensor

\end{description}\end{quote}

\end{fulllineitems}


\end{fulllineitems}



\subparagraph{beyondml.pt.layers.SparseDense module}
\label{\detokenize{beyondml.pt.layers:module-beyondml.pt.layers.SparseDense}}\label{\detokenize{beyondml.pt.layers:beyondml-pt-layers-sparsedense-module}}\index{module@\spxentry{module}!beyondml.pt.layers.SparseDense@\spxentry{beyondml.pt.layers.SparseDense}}\index{beyondml.pt.layers.SparseDense@\spxentry{beyondml.pt.layers.SparseDense}!module@\spxentry{module}}\index{SparseDense (class in beyondml.pt.layers.SparseDense)@\spxentry{SparseDense}\spxextra{class in beyondml.pt.layers.SparseDense}}

\begin{fulllineitems}
\phantomsection\label{\detokenize{beyondml.pt.layers:beyondml.pt.layers.SparseDense.SparseDense}}
\pysigstartsignatures
\pysiglinewithargsret{\sphinxbfcode{\sphinxupquote{class\DUrole{w}{  }}}\sphinxcode{\sphinxupquote{beyondml.pt.layers.SparseDense.}}\sphinxbfcode{\sphinxupquote{SparseDense}}}{\emph{\DUrole{n}{weight}}, \emph{\DUrole{n}{bias}}, \emph{\DUrole{n}{device}\DUrole{o}{=}\DUrole{default_value}{None}}, \emph{\DUrole{n}{dtype}\DUrole{o}{=}\DUrole{default_value}{None}}}{}
\pysigstopsignatures
\sphinxAtStartPar
Bases: \sphinxcode{\sphinxupquote{Module}}

\sphinxAtStartPar
Sparse implementation of a fully\sphinxhyphen{}connected layer
\index{forward() (beyondml.pt.layers.SparseDense.SparseDense method)@\spxentry{forward()}\spxextra{beyondml.pt.layers.SparseDense.SparseDense method}}

\begin{fulllineitems}
\phantomsection\label{\detokenize{beyondml.pt.layers:beyondml.pt.layers.SparseDense.SparseDense.forward}}
\pysigstartsignatures
\pysiglinewithargsret{\sphinxbfcode{\sphinxupquote{forward}}}{\emph{\DUrole{n}{inputs}}}{}
\pysigstopsignatures
\sphinxAtStartPar
Call the layer on input data
\begin{quote}\begin{description}
\sphinxlineitem{Parameters}
\sphinxAtStartPar
\sphinxstyleliteralstrong{\sphinxupquote{inputs}} (\sphinxstyleliteralemphasis{\sphinxupquote{torch.Tensor}}) \textendash{} Inputs to call the layer’s logic on

\sphinxlineitem{Returns}
\sphinxAtStartPar
\sphinxstylestrong{results} \textendash{} The results of the layer’s logic

\sphinxlineitem{Return type}
\sphinxAtStartPar
torch.Tensor

\end{description}\end{quote}

\end{fulllineitems}


\end{fulllineitems}



\subparagraph{beyondml.pt.layers.SparseMultiConv2D module}
\label{\detokenize{beyondml.pt.layers:module-beyondml.pt.layers.SparseMultiConv2D}}\label{\detokenize{beyondml.pt.layers:beyondml-pt-layers-sparsemulticonv2d-module}}\index{module@\spxentry{module}!beyondml.pt.layers.SparseMultiConv2D@\spxentry{beyondml.pt.layers.SparseMultiConv2D}}\index{beyondml.pt.layers.SparseMultiConv2D@\spxentry{beyondml.pt.layers.SparseMultiConv2D}!module@\spxentry{module}}\index{SparseMultiConv2D (class in beyondml.pt.layers.SparseMultiConv2D)@\spxentry{SparseMultiConv2D}\spxextra{class in beyondml.pt.layers.SparseMultiConv2D}}

\begin{fulllineitems}
\phantomsection\label{\detokenize{beyondml.pt.layers:beyondml.pt.layers.SparseMultiConv2D.SparseMultiConv2D}}
\pysigstartsignatures
\pysiglinewithargsret{\sphinxbfcode{\sphinxupquote{class\DUrole{w}{  }}}\sphinxcode{\sphinxupquote{beyondml.pt.layers.SparseMultiConv2D.}}\sphinxbfcode{\sphinxupquote{SparseMultiConv2D}}}{\emph{\DUrole{n}{kernel}}, \emph{\DUrole{n}{bias}}, \emph{\DUrole{n}{padding}\DUrole{o}{=}\DUrole{default_value}{\textquotesingle{}same\textquotesingle{}}}, \emph{\DUrole{n}{strides}\DUrole{o}{=}\DUrole{default_value}{1}}, \emph{\DUrole{n}{device}\DUrole{o}{=}\DUrole{default_value}{None}}, \emph{\DUrole{n}{dtype}\DUrole{o}{=}\DUrole{default_value}{None}}}{}
\pysigstopsignatures
\sphinxAtStartPar
Bases: \sphinxcode{\sphinxupquote{Module}}

\sphinxAtStartPar
Sparse implementation of a Multi 2D Convolutional layer
\index{forward() (beyondml.pt.layers.SparseMultiConv2D.SparseMultiConv2D method)@\spxentry{forward()}\spxextra{beyondml.pt.layers.SparseMultiConv2D.SparseMultiConv2D method}}

\begin{fulllineitems}
\phantomsection\label{\detokenize{beyondml.pt.layers:beyondml.pt.layers.SparseMultiConv2D.SparseMultiConv2D.forward}}
\pysigstartsignatures
\pysiglinewithargsret{\sphinxbfcode{\sphinxupquote{forward}}}{\emph{\DUrole{n}{inputs}}}{}
\pysigstopsignatures
\sphinxAtStartPar
Call the layer on input data
\begin{quote}\begin{description}
\sphinxlineitem{Parameters}
\sphinxAtStartPar
\sphinxstyleliteralstrong{\sphinxupquote{inputs}} (\sphinxstyleliteralemphasis{\sphinxupquote{torch.Tensor}}) \textendash{} Inputs to call the layer’s logic on

\sphinxlineitem{Returns}
\sphinxAtStartPar
\sphinxstylestrong{results} \textendash{} The results of the layer’s logic

\sphinxlineitem{Return type}
\sphinxAtStartPar
torch.Tensor

\end{description}\end{quote}

\end{fulllineitems}


\end{fulllineitems}



\subparagraph{beyondml.pt.layers.SparseMultiConv3D module}
\label{\detokenize{beyondml.pt.layers:module-beyondml.pt.layers.SparseMultiConv3D}}\label{\detokenize{beyondml.pt.layers:beyondml-pt-layers-sparsemulticonv3d-module}}\index{module@\spxentry{module}!beyondml.pt.layers.SparseMultiConv3D@\spxentry{beyondml.pt.layers.SparseMultiConv3D}}\index{beyondml.pt.layers.SparseMultiConv3D@\spxentry{beyondml.pt.layers.SparseMultiConv3D}!module@\spxentry{module}}\index{SparseMultiConv3D (class in beyondml.pt.layers.SparseMultiConv3D)@\spxentry{SparseMultiConv3D}\spxextra{class in beyondml.pt.layers.SparseMultiConv3D}}

\begin{fulllineitems}
\phantomsection\label{\detokenize{beyondml.pt.layers:beyondml.pt.layers.SparseMultiConv3D.SparseMultiConv3D}}
\pysigstartsignatures
\pysiglinewithargsret{\sphinxbfcode{\sphinxupquote{class\DUrole{w}{  }}}\sphinxcode{\sphinxupquote{beyondml.pt.layers.SparseMultiConv3D.}}\sphinxbfcode{\sphinxupquote{SparseMultiConv3D}}}{\emph{\DUrole{n}{kernel}}, \emph{\DUrole{n}{bias}}, \emph{\DUrole{n}{padding}\DUrole{o}{=}\DUrole{default_value}{\textquotesingle{}same\textquotesingle{}}}, \emph{\DUrole{n}{strides}\DUrole{o}{=}\DUrole{default_value}{1}}, \emph{\DUrole{n}{device}\DUrole{o}{=}\DUrole{default_value}{None}}, \emph{\DUrole{n}{dtype}\DUrole{o}{=}\DUrole{default_value}{None}}}{}
\pysigstopsignatures
\sphinxAtStartPar
Bases: \sphinxcode{\sphinxupquote{Module}}

\sphinxAtStartPar
Sparse implementation of a Multitask 3D Convolutional layer, expected to be converted from a
trained, pruned layer
\index{forward() (beyondml.pt.layers.SparseMultiConv3D.SparseMultiConv3D method)@\spxentry{forward()}\spxextra{beyondml.pt.layers.SparseMultiConv3D.SparseMultiConv3D method}}

\begin{fulllineitems}
\phantomsection\label{\detokenize{beyondml.pt.layers:beyondml.pt.layers.SparseMultiConv3D.SparseMultiConv3D.forward}}
\pysigstartsignatures
\pysiglinewithargsret{\sphinxbfcode{\sphinxupquote{forward}}}{\emph{\DUrole{n}{inputs}}}{}
\pysigstopsignatures
\sphinxAtStartPar
Call the layer on input data
\begin{quote}\begin{description}
\sphinxlineitem{Parameters}
\sphinxAtStartPar
\sphinxstyleliteralstrong{\sphinxupquote{inputs}} (\sphinxstyleliteralemphasis{\sphinxupquote{torch.Tensor}}) \textendash{} Inputs to call the layer’s logic on

\sphinxlineitem{Returns}
\sphinxAtStartPar
\sphinxstylestrong{results} \textendash{} The results of the layer’s logic

\sphinxlineitem{Return type}
\sphinxAtStartPar
torch.Tensor

\end{description}\end{quote}

\end{fulllineitems}


\end{fulllineitems}



\subparagraph{beyondml.pt.layers.SparseMultiDense module}
\label{\detokenize{beyondml.pt.layers:module-beyondml.pt.layers.SparseMultiDense}}\label{\detokenize{beyondml.pt.layers:beyondml-pt-layers-sparsemultidense-module}}\index{module@\spxentry{module}!beyondml.pt.layers.SparseMultiDense@\spxentry{beyondml.pt.layers.SparseMultiDense}}\index{beyondml.pt.layers.SparseMultiDense@\spxentry{beyondml.pt.layers.SparseMultiDense}!module@\spxentry{module}}\index{SparseMultiDense (class in beyondml.pt.layers.SparseMultiDense)@\spxentry{SparseMultiDense}\spxextra{class in beyondml.pt.layers.SparseMultiDense}}

\begin{fulllineitems}
\phantomsection\label{\detokenize{beyondml.pt.layers:beyondml.pt.layers.SparseMultiDense.SparseMultiDense}}
\pysigstartsignatures
\pysiglinewithargsret{\sphinxbfcode{\sphinxupquote{class\DUrole{w}{  }}}\sphinxcode{\sphinxupquote{beyondml.pt.layers.SparseMultiDense.}}\sphinxbfcode{\sphinxupquote{SparseMultiDense}}}{\emph{\DUrole{n}{weight}}, \emph{\DUrole{n}{bias}}, \emph{\DUrole{n}{device}\DUrole{o}{=}\DUrole{default_value}{None}}, \emph{\DUrole{n}{dtype}\DUrole{o}{=}\DUrole{default_value}{None}}}{}
\pysigstopsignatures
\sphinxAtStartPar
Bases: \sphinxcode{\sphinxupquote{Module}}

\sphinxAtStartPar
Sparse implementation of the Multi\sphinxhyphen{}Fully\sphinxhyphen{}Connected layer
\index{forward() (beyondml.pt.layers.SparseMultiDense.SparseMultiDense method)@\spxentry{forward()}\spxextra{beyondml.pt.layers.SparseMultiDense.SparseMultiDense method}}

\begin{fulllineitems}
\phantomsection\label{\detokenize{beyondml.pt.layers:beyondml.pt.layers.SparseMultiDense.SparseMultiDense.forward}}
\pysigstartsignatures
\pysiglinewithargsret{\sphinxbfcode{\sphinxupquote{forward}}}{\emph{\DUrole{n}{inputs}}}{}
\pysigstopsignatures
\sphinxAtStartPar
Call the layer on input data
\begin{quote}\begin{description}
\sphinxlineitem{Parameters}
\sphinxAtStartPar
\sphinxstyleliteralstrong{\sphinxupquote{inputs}} (\sphinxstyleliteralemphasis{\sphinxupquote{torch.Tensor}}) \textendash{} Inputs to call the layer’s logic on

\sphinxlineitem{Returns}
\sphinxAtStartPar
\sphinxstylestrong{results} \textendash{} The results of the layer’s logic

\sphinxlineitem{Return type}
\sphinxAtStartPar
torch.Tensor

\end{description}\end{quote}

\end{fulllineitems}


\end{fulllineitems}



\subparagraph{Module contents}
\label{\detokenize{beyondml.pt.layers:module-beyondml.pt.layers}}\label{\detokenize{beyondml.pt.layers:module-contents}}\index{module@\spxentry{module}!beyondml.pt.layers@\spxentry{beyondml.pt.layers}}\index{beyondml.pt.layers@\spxentry{beyondml.pt.layers}!module@\spxentry{module}}
\sphinxAtStartPar
Layers compatible with PyTorch models

\sphinxstepscope


\subparagraph{beyondml.pt.utils package}
\label{\detokenize{beyondml.pt.utils:beyondml-pt-utils-package}}\label{\detokenize{beyondml.pt.utils::doc}}

\subparagraph{Submodules}
\label{\detokenize{beyondml.pt.utils:submodules}}

\subparagraph{beyondml.pt.utils.utils module}
\label{\detokenize{beyondml.pt.utils:module-beyondml.pt.utils.utils}}\label{\detokenize{beyondml.pt.utils:beyondml-pt-utils-utils-module}}\index{module@\spxentry{module}!beyondml.pt.utils.utils@\spxentry{beyondml.pt.utils.utils}}\index{beyondml.pt.utils.utils@\spxentry{beyondml.pt.utils.utils}!module@\spxentry{module}}\index{prune\_model() (in module beyondml.pt.utils.utils)@\spxentry{prune\_model()}\spxextra{in module beyondml.pt.utils.utils}}

\begin{fulllineitems}
\phantomsection\label{\detokenize{beyondml.pt.utils:beyondml.pt.utils.utils.prune_model}}
\pysigstartsignatures
\pysiglinewithargsret{\sphinxcode{\sphinxupquote{beyondml.pt.utils.utils.}}\sphinxbfcode{\sphinxupquote{prune\_model}}}{\emph{\DUrole{n}{model}}, \emph{\DUrole{n}{percentile}}}{}
\pysigstopsignatures
\sphinxAtStartPar
Prune a compatible model
\begin{quote}\begin{description}
\sphinxlineitem{Parameters}\begin{itemize}
\item {} 
\sphinxAtStartPar
\sphinxstyleliteralstrong{\sphinxupquote{model}} (\sphinxstyleliteralemphasis{\sphinxupquote{PyTorch model}}) \textendash{} A model that has been developed to have a \sphinxtitleref{.layers} property containing layers to be pruned

\item {} 
\sphinxAtStartPar
\sphinxstyleliteralstrong{\sphinxupquote{percentile}} (\sphinxstyleliteralemphasis{\sphinxupquote{int}}) \textendash{} An integer between 0 and 99 which corresponds to how much to prune the model

\end{itemize}

\sphinxlineitem{Returns}
\sphinxAtStartPar
\sphinxstylestrong{pruned\_model} \textendash{} The pruned model

\sphinxlineitem{Return type}
\sphinxAtStartPar
PyTorch model

\end{description}\end{quote}
\subsubsection*{Notes}
\begin{itemize}
\item {} 
\sphinxAtStartPar
The model input \sphinxstylestrong{must} have a \sphinxtitleref{.layers} property to be able to function. Only layers within the
\sphinxtitleref{.layers} property which are recognized as prunable are pruned, via their own \sphinxtitleref{.prune()} method

\item {} 
\sphinxAtStartPar
Also acts on the model in place, but returns the model for ease of use

\end{itemize}

\end{fulllineitems}



\subparagraph{Module contents}
\label{\detokenize{beyondml.pt.utils:module-beyondml.pt.utils}}\label{\detokenize{beyondml.pt.utils:module-contents}}\index{module@\spxentry{module}!beyondml.pt.utils@\spxentry{beyondml.pt.utils}}\index{beyondml.pt.utils@\spxentry{beyondml.pt.utils}!module@\spxentry{module}}
\sphinxAtStartPar
Some additional utilities for building MANN models in PyTorch.


\subparagraph{Module contents}
\label{\detokenize{beyondml.pt:module-beyondml.pt}}\label{\detokenize{beyondml.pt:module-contents}}\index{module@\spxentry{module}!beyondml.pt@\spxentry{beyondml.pt}}\index{beyondml.pt@\spxentry{beyondml.pt}!module@\spxentry{module}}
\sphinxAtStartPar
\#\# PyTorch compatibility for building MANN models

\sphinxAtStartPar
The \sphinxtitleref{beyondml.pt} subpackage contains layers and utilities for creating and pruning models using {[}PyTorch{]}(\sphinxurl{https://pytorch.org}).
The package contains two subpackages, the \sphinxtitleref{beyondml.pt.layers} package, and the \sphinxtitleref{beyondml.pt.utils} package.

\sphinxAtStartPar
Within the \sphinxtitleref{layers} package, there is current functionality for the the following layers:
\sphinxhyphen{} \sphinxtitleref{beyondml.pt.layers.Conv2D}
\sphinxhyphen{} \sphinxtitleref{beyondml.pt.layers.Dense}
\sphinxhyphen{} \sphinxtitleref{beyondml.pt.layers.FilterLayer}
\sphinxhyphen{} \sphinxtitleref{beyondml.pt.layers.MaskedConv2D}
\sphinxhyphen{} \sphinxtitleref{beyondml.pt.layers.MaskedDense}
\sphinxhyphen{} \sphinxtitleref{beyondml.pt.layers.MultiConv2D}
\sphinxhyphen{} \sphinxtitleref{beyondml.pt.layers.MultiDense}
\sphinxhyphen{} \sphinxtitleref{beyondml.pt.layers.MultiMaskedConv2D}
\sphinxhyphen{} \sphinxtitleref{beyondml.pt.layers.MultiMaskedDense}
\sphinxhyphen{} \sphinxtitleref{beyondml.pt.layers.SelectorLayer}
\sphinxhyphen{} \sphinxtitleref{beyondml.pt.layers.SparseConv2D}
\sphinxhyphen{} \sphinxtitleref{beyondml.pt.layers.SparseDense}
\sphinxhyphen{} \sphinxtitleref{beyondml.pt.layers.SparseMultiConv2D}
\sphinxhyphen{} \sphinxtitleref{beyondml.pt.layers.SparseMultiDense}

\sphinxAtStartPar
Within the \sphinxtitleref{beyondml.pt.utils} package, there is currently only one function, the \sphinxtitleref{prune\_model} function. Because of
the openness of developing with PyTorch in comparison to TensorFlow, there is far less functionality that
can be supplied directly via BeyondML. Instead, for converting models from training to inference, the user
is left to devise the best way to do so by building his or her own classes.

\sphinxAtStartPar
\#\#\# Best Practices for Pruning
In order to use the \sphinxtitleref{utils.prune\_model} function, the model itself must have a \sphinxtitleref{.layers} property. This property
is used to determine which layers can be pruned. \sphinxstylestrong{Only layers which support pruning and which are included in the
\textasciigrave{}.layers\textasciigrave{} property are pruned,} meaning the user can determine which exact layers in the model he or she wants
pruned. Alternatively, the user can create their own pruning function or method on the class itself and prune that way,
utilizing each of the \sphinxtitleref{.prune()} methods of the layers provided.

\sphinxstepscope


\paragraph{beyondml.tflow package}
\label{\detokenize{beyondml.tflow:beyondml-tflow-package}}\label{\detokenize{beyondml.tflow::doc}}

\subparagraph{Subpackages}
\label{\detokenize{beyondml.tflow:subpackages}}
\sphinxstepscope


\subparagraph{beyondml.tflow.layers package}
\label{\detokenize{beyondml.tflow.layers:beyondml-tflow-layers-package}}\label{\detokenize{beyondml.tflow.layers::doc}}

\subparagraph{Submodules}
\label{\detokenize{beyondml.tflow.layers:submodules}}

\subparagraph{beyondml.tflow.layers.FilterLayer module}
\label{\detokenize{beyondml.tflow.layers:module-beyondml.tflow.layers.FilterLayer}}\label{\detokenize{beyondml.tflow.layers:beyondml-tflow-layers-filterlayer-module}}\index{module@\spxentry{module}!beyondml.tflow.layers.FilterLayer@\spxentry{beyondml.tflow.layers.FilterLayer}}\index{beyondml.tflow.layers.FilterLayer@\spxentry{beyondml.tflow.layers.FilterLayer}!module@\spxentry{module}}\index{FilterLayer (class in beyondml.tflow.layers.FilterLayer)@\spxentry{FilterLayer}\spxextra{class in beyondml.tflow.layers.FilterLayer}}

\begin{fulllineitems}
\phantomsection\label{\detokenize{beyondml.tflow.layers:beyondml.tflow.layers.FilterLayer.FilterLayer}}
\pysigstartsignatures
\pysiglinewithargsret{\sphinxbfcode{\sphinxupquote{class\DUrole{w}{  }}}\sphinxcode{\sphinxupquote{beyondml.tflow.layers.FilterLayer.}}\sphinxbfcode{\sphinxupquote{FilterLayer}}}{\emph{\DUrole{o}{*}\DUrole{n}{args}}, \emph{\DUrole{o}{**}\DUrole{n}{kwargs}}}{}
\pysigstopsignatures
\sphinxAtStartPar
Bases: \sphinxcode{\sphinxupquote{Layer}}

\sphinxAtStartPar
Layer which filters inputs based on status of \sphinxtitleref{on} or \sphinxtitleref{off}

\sphinxAtStartPar
Example:

\begin{sphinxVerbatim}[commandchars=\\\{\}]
\PYG{g+gp}{\PYGZgt{}\PYGZgt{}\PYGZgt{} }\PYG{c+c1}{\PYGZsh{} Create a model with just a FilterLayer}
\PYG{g+gp}{\PYGZgt{}\PYGZgt{}\PYGZgt{} }\PYG{n}{input\PYGZus{}layer} \PYG{o}{=} \PYG{n}{tf}\PYG{o}{.}\PYG{n}{keras}\PYG{o}{.}\PYG{n}{layers}\PYG{o}{.}\PYG{n}{Input}\PYG{p}{(}\PYG{l+m+mi}{10}\PYG{p}{)}
\PYG{g+gp}{\PYGZgt{}\PYGZgt{}\PYGZgt{} }\PYG{n}{filter\PYGZus{}layer} \PYG{o}{=} \PYG{n}{mann}\PYG{o}{.}\PYG{n}{layers}\PYG{o}{.}\PYG{n}{FilterLayer}\PYG{p}{(}\PYG{p}{)}\PYG{p}{(}\PYG{n}{input\PYGZus{}layer}\PYG{p}{)}
\PYG{g+gp}{\PYGZgt{}\PYGZgt{}\PYGZgt{} }\PYG{n}{model} \PYG{o}{=} \PYG{n}{tf}\PYG{o}{.}\PYG{n}{keras}\PYG{o}{.}\PYG{n}{models}\PYG{o}{.}\PYG{n}{Model}\PYG{p}{(}\PYG{n}{input\PYGZus{}layer}\PYG{p}{,} \PYG{n}{filter\PYGZus{}layer}\PYG{p}{)}
\PYG{g+gp}{\PYGZgt{}\PYGZgt{}\PYGZgt{} }\PYG{n}{model}\PYG{o}{.}\PYG{n}{compile}\PYG{p}{(}\PYG{p}{)}
\PYG{g+gp}{\PYGZgt{}\PYGZgt{}\PYGZgt{} }\PYG{c+c1}{\PYGZsh{} Call the model with the layer turned on}
\PYG{g+gp}{\PYGZgt{}\PYGZgt{}\PYGZgt{} }\PYG{n}{data} \PYG{o}{=} \PYG{n}{np}\PYG{o}{.}\PYG{n}{arange}\PYG{p}{(}\PYG{l+m+mi}{10}\PYG{p}{)}\PYG{o}{.}\PYG{n}{reshape}\PYG{p}{(}\PYG{p}{(}\PYG{l+m+mi}{1}\PYG{p}{,} \PYG{l+m+mi}{10}\PYG{p}{)}\PYG{p}{)}
\PYG{g+gp}{\PYGZgt{}\PYGZgt{}\PYGZgt{} }\PYG{n}{model}\PYG{o}{.}\PYG{n}{predict}\PYG{p}{(}\PYG{n}{data}\PYG{p}{)}
\PYG{g+go}{array([[0., 1., 2., 3., 4., 5., 6., 7., 8., 9.]], dtype=float32)}
\PYG{g+gp}{\PYGZgt{}\PYGZgt{}\PYGZgt{} }\PYG{c+c1}{\PYGZsh{} Turn off the FilterLayer and call it again}
\PYG{g+gp}{\PYGZgt{}\PYGZgt{}\PYGZgt{} }\PYG{n}{model}\PYG{o}{.}\PYG{n}{layers}\PYG{p}{[}\PYG{o}{\PYGZhy{}}\PYG{l+m+mi}{1}\PYG{p}{]}\PYG{o}{.}\PYG{n}{turn\PYGZus{}off}\PYG{p}{(}\PYG{p}{)}
\PYG{g+gp}{\PYGZgt{}\PYGZgt{}\PYGZgt{} }\PYG{c+c1}{\PYGZsh{} Model must be recompiled after turning the layer on or off}
\PYG{g+gp}{\PYGZgt{}\PYGZgt{}\PYGZgt{} }\PYG{n}{model}\PYG{o}{.}\PYG{n}{compile}\PYG{p}{(}\PYG{p}{)}
\PYG{g+gp}{\PYGZgt{}\PYGZgt{}\PYGZgt{} }\PYG{n}{model}\PYG{o}{.}\PYG{n}{predict}\PYG{p}{(}\PYG{n}{data}\PYG{p}{)}
\PYG{g+go}{array([[0., 0., 0., 0., 0., 0., 0., 0., 0., 0.]], dtype=float32)}
\end{sphinxVerbatim}
\index{call() (beyondml.tflow.layers.FilterLayer.FilterLayer method)@\spxentry{call()}\spxextra{beyondml.tflow.layers.FilterLayer.FilterLayer method}}

\begin{fulllineitems}
\phantomsection\label{\detokenize{beyondml.tflow.layers:beyondml.tflow.layers.FilterLayer.FilterLayer.call}}
\pysigstartsignatures
\pysiglinewithargsret{\sphinxbfcode{\sphinxupquote{call}}}{\emph{\DUrole{n}{inputs}}}{}
\pysigstopsignatures
\sphinxAtStartPar
This is where the layer’s logic lives and is called upon inputs
\begin{quote}\begin{description}
\sphinxlineitem{Parameters}
\sphinxAtStartPar
\sphinxstyleliteralstrong{\sphinxupquote{inputs}} (\sphinxstyleliteralemphasis{\sphinxupquote{TensorFlow Tensor}}\sphinxstyleliteralemphasis{\sphinxupquote{ or }}\sphinxstyleliteralemphasis{\sphinxupquote{Tensor\sphinxhyphen{}like}}) \textendash{} The inputs to the layer

\sphinxlineitem{Returns}
\sphinxAtStartPar
\sphinxstylestrong{outputs} \textendash{} The outputs of the layer’s logic

\sphinxlineitem{Return type}
\sphinxAtStartPar
TensorFlow Tensor

\end{description}\end{quote}

\end{fulllineitems}

\index{from\_config() (beyondml.tflow.layers.FilterLayer.FilterLayer class method)@\spxentry{from\_config()}\spxextra{beyondml.tflow.layers.FilterLayer.FilterLayer class method}}

\begin{fulllineitems}
\phantomsection\label{\detokenize{beyondml.tflow.layers:beyondml.tflow.layers.FilterLayer.FilterLayer.from_config}}
\pysigstartsignatures
\pysiglinewithargsret{\sphinxbfcode{\sphinxupquote{classmethod\DUrole{w}{  }}}\sphinxbfcode{\sphinxupquote{from\_config}}}{\emph{\DUrole{n}{config}}}{}
\pysigstopsignatures
\sphinxAtStartPar
Creates a layer from its config.

\sphinxAtStartPar
This method is the reverse of \sphinxtitleref{get\_config},
capable of instantiating the same layer from the config
dictionary. It does not handle layer connectivity
(handled by Network), nor weights (handled by \sphinxtitleref{set\_weights}).
\begin{quote}\begin{description}
\sphinxlineitem{Parameters}
\sphinxAtStartPar
\sphinxstyleliteralstrong{\sphinxupquote{config}} \textendash{} A Python dictionary, typically the
output of get\_config.

\sphinxlineitem{Returns}
\sphinxAtStartPar
A layer instance.

\end{description}\end{quote}

\end{fulllineitems}

\index{get\_config() (beyondml.tflow.layers.FilterLayer.FilterLayer method)@\spxentry{get\_config()}\spxextra{beyondml.tflow.layers.FilterLayer.FilterLayer method}}

\begin{fulllineitems}
\phantomsection\label{\detokenize{beyondml.tflow.layers:beyondml.tflow.layers.FilterLayer.FilterLayer.get_config}}
\pysigstartsignatures
\pysiglinewithargsret{\sphinxbfcode{\sphinxupquote{get\_config}}}{}{}
\pysigstopsignatures
\sphinxAtStartPar
Returns the config of the layer.

\sphinxAtStartPar
A layer config is a Python dictionary (serializable)
containing the configuration of a layer.
The same layer can be reinstantiated later
(without its trained weights) from this configuration.

\sphinxAtStartPar
The config of a layer does not include connectivity
information, nor the layer class name. These are handled
by \sphinxtitleref{Network} (one layer of abstraction above).

\sphinxAtStartPar
Note that \sphinxtitleref{get\_config()} does not guarantee to return a fresh copy of
dict every time it is called. The callers should make a copy of the
returned dict if they want to modify it.
\begin{quote}\begin{description}
\sphinxlineitem{Returns}
\sphinxAtStartPar
Python dictionary.

\end{description}\end{quote}

\end{fulllineitems}

\index{turn\_off() (beyondml.tflow.layers.FilterLayer.FilterLayer method)@\spxentry{turn\_off()}\spxextra{beyondml.tflow.layers.FilterLayer.FilterLayer method}}

\begin{fulllineitems}
\phantomsection\label{\detokenize{beyondml.tflow.layers:beyondml.tflow.layers.FilterLayer.FilterLayer.turn_off}}
\pysigstartsignatures
\pysiglinewithargsret{\sphinxbfcode{\sphinxupquote{turn\_off}}}{}{}
\pysigstopsignatures
\sphinxAtStartPar
Turn the layer \sphinxtitleref{off} so inputs are destroyed and all\sphinxhyphen{}zero tensors are output

\end{fulllineitems}

\index{turn\_on() (beyondml.tflow.layers.FilterLayer.FilterLayer method)@\spxentry{turn\_on()}\spxextra{beyondml.tflow.layers.FilterLayer.FilterLayer method}}

\begin{fulllineitems}
\phantomsection\label{\detokenize{beyondml.tflow.layers:beyondml.tflow.layers.FilterLayer.FilterLayer.turn_on}}
\pysigstartsignatures
\pysiglinewithargsret{\sphinxbfcode{\sphinxupquote{turn\_on}}}{}{}
\pysigstopsignatures
\sphinxAtStartPar
Turn the layer \sphinxtitleref{on} so inputs are returned unchanged as outputs

\end{fulllineitems}


\end{fulllineitems}



\subparagraph{beyondml.tflow.layers.MaskedConv2D module}
\label{\detokenize{beyondml.tflow.layers:module-beyondml.tflow.layers.MaskedConv2D}}\label{\detokenize{beyondml.tflow.layers:beyondml-tflow-layers-maskedconv2d-module}}\index{module@\spxentry{module}!beyondml.tflow.layers.MaskedConv2D@\spxentry{beyondml.tflow.layers.MaskedConv2D}}\index{beyondml.tflow.layers.MaskedConv2D@\spxentry{beyondml.tflow.layers.MaskedConv2D}!module@\spxentry{module}}\index{MaskedConv2D (class in beyondml.tflow.layers.MaskedConv2D)@\spxentry{MaskedConv2D}\spxextra{class in beyondml.tflow.layers.MaskedConv2D}}

\begin{fulllineitems}
\phantomsection\label{\detokenize{beyondml.tflow.layers:beyondml.tflow.layers.MaskedConv2D.MaskedConv2D}}
\pysigstartsignatures
\pysiglinewithargsret{\sphinxbfcode{\sphinxupquote{class\DUrole{w}{  }}}\sphinxcode{\sphinxupquote{beyondml.tflow.layers.MaskedConv2D.}}\sphinxbfcode{\sphinxupquote{MaskedConv2D}}}{\emph{\DUrole{o}{*}\DUrole{n}{args}}, \emph{\DUrole{o}{**}\DUrole{n}{kwargs}}}{}
\pysigstopsignatures
\sphinxAtStartPar
Bases: \sphinxcode{\sphinxupquote{Layer}}

\sphinxAtStartPar
Masked 2\sphinxhyphen{}dimensional convolutional layer. For full documentation of the convolutional architecture, see the
TensorFlow Keras Convolutional2D layer documentation.

\sphinxAtStartPar
This layer implements masking consistent with the BeyondML API to support developing sparse models.
\index{build() (beyondml.tflow.layers.MaskedConv2D.MaskedConv2D method)@\spxentry{build()}\spxextra{beyondml.tflow.layers.MaskedConv2D.MaskedConv2D method}}

\begin{fulllineitems}
\phantomsection\label{\detokenize{beyondml.tflow.layers:beyondml.tflow.layers.MaskedConv2D.MaskedConv2D.build}}
\pysigstartsignatures
\pysiglinewithargsret{\sphinxbfcode{\sphinxupquote{build}}}{\emph{\DUrole{n}{input\_shape}}}{}
\pysigstopsignatures
\sphinxAtStartPar
Build the layer in preparation to be trained or called. Should not be called directly,
but rather is called when the layer is added to a model

\end{fulllineitems}

\index{call() (beyondml.tflow.layers.MaskedConv2D.MaskedConv2D method)@\spxentry{call()}\spxextra{beyondml.tflow.layers.MaskedConv2D.MaskedConv2D method}}

\begin{fulllineitems}
\phantomsection\label{\detokenize{beyondml.tflow.layers:beyondml.tflow.layers.MaskedConv2D.MaskedConv2D.call}}
\pysigstartsignatures
\pysiglinewithargsret{\sphinxbfcode{\sphinxupquote{call}}}{\emph{\DUrole{n}{inputs}}}{}
\pysigstopsignatures
\sphinxAtStartPar
This is where the layer’s logic lives and is called upon inputs
\begin{quote}\begin{description}
\sphinxlineitem{Parameters}
\sphinxAtStartPar
\sphinxstyleliteralstrong{\sphinxupquote{inputs}} (\sphinxstyleliteralemphasis{\sphinxupquote{TensorFlow Tensor}}\sphinxstyleliteralemphasis{\sphinxupquote{ or }}\sphinxstyleliteralemphasis{\sphinxupquote{Tensor\sphinxhyphen{}like}}) \textendash{} The inputs to the layer

\sphinxlineitem{Returns}
\sphinxAtStartPar
\sphinxstylestrong{outputs} \textendash{} The outputs of the layer’s logic

\sphinxlineitem{Return type}
\sphinxAtStartPar
TensorFlow Tensor

\end{description}\end{quote}

\end{fulllineitems}

\index{from\_config() (beyondml.tflow.layers.MaskedConv2D.MaskedConv2D class method)@\spxentry{from\_config()}\spxextra{beyondml.tflow.layers.MaskedConv2D.MaskedConv2D class method}}

\begin{fulllineitems}
\phantomsection\label{\detokenize{beyondml.tflow.layers:beyondml.tflow.layers.MaskedConv2D.MaskedConv2D.from_config}}
\pysigstartsignatures
\pysiglinewithargsret{\sphinxbfcode{\sphinxupquote{classmethod\DUrole{w}{  }}}\sphinxbfcode{\sphinxupquote{from\_config}}}{\emph{\DUrole{n}{config}}}{}
\pysigstopsignatures
\sphinxAtStartPar
Creates a layer from its config.

\sphinxAtStartPar
This method is the reverse of \sphinxtitleref{get\_config},
capable of instantiating the same layer from the config
dictionary. It does not handle layer connectivity
(handled by Network), nor weights (handled by \sphinxtitleref{set\_weights}).
\begin{quote}\begin{description}
\sphinxlineitem{Parameters}
\sphinxAtStartPar
\sphinxstyleliteralstrong{\sphinxupquote{config}} \textendash{} A Python dictionary, typically the
output of get\_config.

\sphinxlineitem{Returns}
\sphinxAtStartPar
A layer instance.

\end{description}\end{quote}

\end{fulllineitems}

\index{get\_config() (beyondml.tflow.layers.MaskedConv2D.MaskedConv2D method)@\spxentry{get\_config()}\spxextra{beyondml.tflow.layers.MaskedConv2D.MaskedConv2D method}}

\begin{fulllineitems}
\phantomsection\label{\detokenize{beyondml.tflow.layers:beyondml.tflow.layers.MaskedConv2D.MaskedConv2D.get_config}}
\pysigstartsignatures
\pysiglinewithargsret{\sphinxbfcode{\sphinxupquote{get\_config}}}{}{}
\pysigstopsignatures
\sphinxAtStartPar
Returns the config of the layer.

\sphinxAtStartPar
A layer config is a Python dictionary (serializable)
containing the configuration of a layer.
The same layer can be reinstantiated later
(without its trained weights) from this configuration.

\sphinxAtStartPar
The config of a layer does not include connectivity
information, nor the layer class name. These are handled
by \sphinxtitleref{Network} (one layer of abstraction above).

\sphinxAtStartPar
Note that \sphinxtitleref{get\_config()} does not guarantee to return a fresh copy of
dict every time it is called. The callers should make a copy of the
returned dict if they want to modify it.
\begin{quote}\begin{description}
\sphinxlineitem{Returns}
\sphinxAtStartPar
Python dictionary.

\end{description}\end{quote}

\end{fulllineitems}

\index{kernel\_size (beyondml.tflow.layers.MaskedConv2D.MaskedConv2D property)@\spxentry{kernel\_size}\spxextra{beyondml.tflow.layers.MaskedConv2D.MaskedConv2D property}}

\begin{fulllineitems}
\phantomsection\label{\detokenize{beyondml.tflow.layers:beyondml.tflow.layers.MaskedConv2D.MaskedConv2D.kernel_size}}
\pysigstartsignatures
\pysigline{\sphinxbfcode{\sphinxupquote{property\DUrole{w}{  }}}\sphinxbfcode{\sphinxupquote{kernel\_size}}}
\pysigstopsignatures
\end{fulllineitems}

\index{set\_masks() (beyondml.tflow.layers.MaskedConv2D.MaskedConv2D method)@\spxentry{set\_masks()}\spxextra{beyondml.tflow.layers.MaskedConv2D.MaskedConv2D method}}

\begin{fulllineitems}
\phantomsection\label{\detokenize{beyondml.tflow.layers:beyondml.tflow.layers.MaskedConv2D.MaskedConv2D.set_masks}}
\pysigstartsignatures
\pysiglinewithargsret{\sphinxbfcode{\sphinxupquote{set\_masks}}}{\emph{\DUrole{n}{new\_masks}}}{}
\pysigstopsignatures
\sphinxAtStartPar
Set the masks for the layer
\begin{quote}\begin{description}
\sphinxlineitem{Parameters}
\sphinxAtStartPar
\sphinxstyleliteralstrong{\sphinxupquote{new\_masks}} (\sphinxstyleliteralemphasis{\sphinxupquote{list}}\sphinxstyleliteralemphasis{\sphinxupquote{ of }}\sphinxstyleliteralemphasis{\sphinxupquote{arrays}}\sphinxstyleliteralemphasis{\sphinxupquote{ or }}\sphinxstyleliteralemphasis{\sphinxupquote{array\sphinxhyphen{}likes}}) \textendash{} The new masks to set for the layer

\end{description}\end{quote}

\end{fulllineitems}


\end{fulllineitems}



\subparagraph{beyondml.tflow.layers.MaskedConv3D module}
\label{\detokenize{beyondml.tflow.layers:module-beyondml.tflow.layers.MaskedConv3D}}\label{\detokenize{beyondml.tflow.layers:beyondml-tflow-layers-maskedconv3d-module}}\index{module@\spxentry{module}!beyondml.tflow.layers.MaskedConv3D@\spxentry{beyondml.tflow.layers.MaskedConv3D}}\index{beyondml.tflow.layers.MaskedConv3D@\spxentry{beyondml.tflow.layers.MaskedConv3D}!module@\spxentry{module}}\index{MaskedConv3D (class in beyondml.tflow.layers.MaskedConv3D)@\spxentry{MaskedConv3D}\spxextra{class in beyondml.tflow.layers.MaskedConv3D}}

\begin{fulllineitems}
\phantomsection\label{\detokenize{beyondml.tflow.layers:beyondml.tflow.layers.MaskedConv3D.MaskedConv3D}}
\pysigstartsignatures
\pysiglinewithargsret{\sphinxbfcode{\sphinxupquote{class\DUrole{w}{  }}}\sphinxcode{\sphinxupquote{beyondml.tflow.layers.MaskedConv3D.}}\sphinxbfcode{\sphinxupquote{MaskedConv3D}}}{\emph{\DUrole{o}{*}\DUrole{n}{args}}, \emph{\DUrole{o}{**}\DUrole{n}{kwargs}}}{}
\pysigstopsignatures
\sphinxAtStartPar
Bases: \sphinxcode{\sphinxupquote{Layer}}

\sphinxAtStartPar
Masked 3\sphinxhyphen{}dimensional convolutional layer. For full documentation of the
convolutional architecture, see the TensorFlow Keras Convolutional3D layer documentation.

\sphinxAtStartPar
This layer implements masking consistent with the BeyondML API to support
developing sparse models
\index{build() (beyondml.tflow.layers.MaskedConv3D.MaskedConv3D method)@\spxentry{build()}\spxextra{beyondml.tflow.layers.MaskedConv3D.MaskedConv3D method}}

\begin{fulllineitems}
\phantomsection\label{\detokenize{beyondml.tflow.layers:beyondml.tflow.layers.MaskedConv3D.MaskedConv3D.build}}
\pysigstartsignatures
\pysiglinewithargsret{\sphinxbfcode{\sphinxupquote{build}}}{\emph{\DUrole{n}{input\_shape}}}{}
\pysigstopsignatures
\sphinxAtStartPar
Build the layer in preparation to be trained or called. Should not be called directly,
but rather is called when the layer is added to a model

\end{fulllineitems}

\index{call() (beyondml.tflow.layers.MaskedConv3D.MaskedConv3D method)@\spxentry{call()}\spxextra{beyondml.tflow.layers.MaskedConv3D.MaskedConv3D method}}

\begin{fulllineitems}
\phantomsection\label{\detokenize{beyondml.tflow.layers:beyondml.tflow.layers.MaskedConv3D.MaskedConv3D.call}}
\pysigstartsignatures
\pysiglinewithargsret{\sphinxbfcode{\sphinxupquote{call}}}{\emph{\DUrole{n}{inputs}}}{}
\pysigstopsignatures
\sphinxAtStartPar
This is where the layer’s logic lives and is called upon inputs
\begin{quote}\begin{description}
\sphinxlineitem{Parameters}
\sphinxAtStartPar
\sphinxstyleliteralstrong{\sphinxupquote{inputs}} (\sphinxstyleliteralemphasis{\sphinxupquote{TensorFlow Tensor}}\sphinxstyleliteralemphasis{\sphinxupquote{ or }}\sphinxstyleliteralemphasis{\sphinxupquote{Tensor\sphinxhyphen{}like}}) \textendash{} The inputs to the layer

\sphinxlineitem{Returns}
\sphinxAtStartPar
\sphinxstylestrong{outputs} \textendash{} The outputs of the layer’s logic

\sphinxlineitem{Return type}
\sphinxAtStartPar
TensorFlow Tensor

\end{description}\end{quote}

\end{fulllineitems}

\index{from\_config() (beyondml.tflow.layers.MaskedConv3D.MaskedConv3D class method)@\spxentry{from\_config()}\spxextra{beyondml.tflow.layers.MaskedConv3D.MaskedConv3D class method}}

\begin{fulllineitems}
\phantomsection\label{\detokenize{beyondml.tflow.layers:beyondml.tflow.layers.MaskedConv3D.MaskedConv3D.from_config}}
\pysigstartsignatures
\pysiglinewithargsret{\sphinxbfcode{\sphinxupquote{classmethod\DUrole{w}{  }}}\sphinxbfcode{\sphinxupquote{from\_config}}}{\emph{\DUrole{n}{config}}}{}
\pysigstopsignatures
\sphinxAtStartPar
Creates a layer from its config.

\sphinxAtStartPar
This method is the reverse of \sphinxtitleref{get\_config},
capable of instantiating the same layer from the config
dictionary. It does not handle layer connectivity
(handled by Network), nor weights (handled by \sphinxtitleref{set\_weights}).
\begin{quote}\begin{description}
\sphinxlineitem{Parameters}
\sphinxAtStartPar
\sphinxstyleliteralstrong{\sphinxupquote{config}} \textendash{} A Python dictionary, typically the
output of get\_config.

\sphinxlineitem{Returns}
\sphinxAtStartPar
A layer instance.

\end{description}\end{quote}

\end{fulllineitems}

\index{get\_config() (beyondml.tflow.layers.MaskedConv3D.MaskedConv3D method)@\spxentry{get\_config()}\spxextra{beyondml.tflow.layers.MaskedConv3D.MaskedConv3D method}}

\begin{fulllineitems}
\phantomsection\label{\detokenize{beyondml.tflow.layers:beyondml.tflow.layers.MaskedConv3D.MaskedConv3D.get_config}}
\pysigstartsignatures
\pysiglinewithargsret{\sphinxbfcode{\sphinxupquote{get\_config}}}{}{}
\pysigstopsignatures
\sphinxAtStartPar
Returns the config of the layer.

\sphinxAtStartPar
A layer config is a Python dictionary (serializable)
containing the configuration of a layer.
The same layer can be reinstantiated later
(without its trained weights) from this configuration.

\sphinxAtStartPar
The config of a layer does not include connectivity
information, nor the layer class name. These are handled
by \sphinxtitleref{Network} (one layer of abstraction above).

\sphinxAtStartPar
Note that \sphinxtitleref{get\_config()} does not guarantee to return a fresh copy of
dict every time it is called. The callers should make a copy of the
returned dict if they want to modify it.
\begin{quote}\begin{description}
\sphinxlineitem{Returns}
\sphinxAtStartPar
Python dictionary.

\end{description}\end{quote}

\end{fulllineitems}

\index{kernel\_size (beyondml.tflow.layers.MaskedConv3D.MaskedConv3D property)@\spxentry{kernel\_size}\spxextra{beyondml.tflow.layers.MaskedConv3D.MaskedConv3D property}}

\begin{fulllineitems}
\phantomsection\label{\detokenize{beyondml.tflow.layers:beyondml.tflow.layers.MaskedConv3D.MaskedConv3D.kernel_size}}
\pysigstartsignatures
\pysigline{\sphinxbfcode{\sphinxupquote{property\DUrole{w}{  }}}\sphinxbfcode{\sphinxupquote{kernel\_size}}}
\pysigstopsignatures
\end{fulllineitems}

\index{set\_masks() (beyondml.tflow.layers.MaskedConv3D.MaskedConv3D method)@\spxentry{set\_masks()}\spxextra{beyondml.tflow.layers.MaskedConv3D.MaskedConv3D method}}

\begin{fulllineitems}
\phantomsection\label{\detokenize{beyondml.tflow.layers:beyondml.tflow.layers.MaskedConv3D.MaskedConv3D.set_masks}}
\pysigstartsignatures
\pysiglinewithargsret{\sphinxbfcode{\sphinxupquote{set\_masks}}}{\emph{\DUrole{n}{new\_masks}}}{}
\pysigstopsignatures
\sphinxAtStartPar
Set the masks for the layer
\begin{quote}\begin{description}
\sphinxlineitem{Parameters}
\sphinxAtStartPar
\sphinxstyleliteralstrong{\sphinxupquote{new\_masks}} (\sphinxstyleliteralemphasis{\sphinxupquote{list}}\sphinxstyleliteralemphasis{\sphinxupquote{ of }}\sphinxstyleliteralemphasis{\sphinxupquote{arrays}}\sphinxstyleliteralemphasis{\sphinxupquote{ or }}\sphinxstyleliteralemphasis{\sphinxupquote{array\sphinxhyphen{}likes}}) \textendash{} The new masks to set for the layer

\end{description}\end{quote}

\end{fulllineitems}


\end{fulllineitems}



\subparagraph{beyondml.tflow.layers.MaskedDense module}
\label{\detokenize{beyondml.tflow.layers:module-beyondml.tflow.layers.MaskedDense}}\label{\detokenize{beyondml.tflow.layers:beyondml-tflow-layers-maskeddense-module}}\index{module@\spxentry{module}!beyondml.tflow.layers.MaskedDense@\spxentry{beyondml.tflow.layers.MaskedDense}}\index{beyondml.tflow.layers.MaskedDense@\spxentry{beyondml.tflow.layers.MaskedDense}!module@\spxentry{module}}\index{MaskedDense (class in beyondml.tflow.layers.MaskedDense)@\spxentry{MaskedDense}\spxextra{class in beyondml.tflow.layers.MaskedDense}}

\begin{fulllineitems}
\phantomsection\label{\detokenize{beyondml.tflow.layers:beyondml.tflow.layers.MaskedDense.MaskedDense}}
\pysigstartsignatures
\pysiglinewithargsret{\sphinxbfcode{\sphinxupquote{class\DUrole{w}{  }}}\sphinxcode{\sphinxupquote{beyondml.tflow.layers.MaskedDense.}}\sphinxbfcode{\sphinxupquote{MaskedDense}}}{\emph{\DUrole{o}{*}\DUrole{n}{args}}, \emph{\DUrole{o}{**}\DUrole{n}{kwargs}}}{}
\pysigstopsignatures
\sphinxAtStartPar
Bases: \sphinxcode{\sphinxupquote{Layer}}

\sphinxAtStartPar
Masked fully connected layer. For full documentation of the fully\sphinxhyphen{}connected architecture, see the
TensorFlow Keras Dense layer documentation.

\sphinxAtStartPar
This layer implements masking consistent with the BeyondML API to support developing sparse models.
\index{build() (beyondml.tflow.layers.MaskedDense.MaskedDense method)@\spxentry{build()}\spxextra{beyondml.tflow.layers.MaskedDense.MaskedDense method}}

\begin{fulllineitems}
\phantomsection\label{\detokenize{beyondml.tflow.layers:beyondml.tflow.layers.MaskedDense.MaskedDense.build}}
\pysigstartsignatures
\pysiglinewithargsret{\sphinxbfcode{\sphinxupquote{build}}}{\emph{\DUrole{n}{input\_shape}}}{}
\pysigstopsignatures
\sphinxAtStartPar
Build the layer in preparation to be trained or called. Should not be called directly,
but rather is called when the layer is added to a model

\end{fulllineitems}

\index{call() (beyondml.tflow.layers.MaskedDense.MaskedDense method)@\spxentry{call()}\spxextra{beyondml.tflow.layers.MaskedDense.MaskedDense method}}

\begin{fulllineitems}
\phantomsection\label{\detokenize{beyondml.tflow.layers:beyondml.tflow.layers.MaskedDense.MaskedDense.call}}
\pysigstartsignatures
\pysiglinewithargsret{\sphinxbfcode{\sphinxupquote{call}}}{\emph{\DUrole{n}{inputs}}}{}
\pysigstopsignatures
\sphinxAtStartPar
This is where the layer’s logic lives and is called upon inputs
\begin{quote}\begin{description}
\sphinxlineitem{Parameters}
\sphinxAtStartPar
\sphinxstyleliteralstrong{\sphinxupquote{inputs}} (\sphinxstyleliteralemphasis{\sphinxupquote{TensorFlow Tensor}}\sphinxstyleliteralemphasis{\sphinxupquote{ or }}\sphinxstyleliteralemphasis{\sphinxupquote{Tensor\sphinxhyphen{}like}}) \textendash{} The inputs to the layer

\sphinxlineitem{Returns}
\sphinxAtStartPar
\sphinxstylestrong{outputs} \textendash{} The outputs of the layer’s logic

\sphinxlineitem{Return type}
\sphinxAtStartPar
TensorFlow Tensor

\end{description}\end{quote}

\end{fulllineitems}

\index{from\_config() (beyondml.tflow.layers.MaskedDense.MaskedDense class method)@\spxentry{from\_config()}\spxextra{beyondml.tflow.layers.MaskedDense.MaskedDense class method}}

\begin{fulllineitems}
\phantomsection\label{\detokenize{beyondml.tflow.layers:beyondml.tflow.layers.MaskedDense.MaskedDense.from_config}}
\pysigstartsignatures
\pysiglinewithargsret{\sphinxbfcode{\sphinxupquote{classmethod\DUrole{w}{  }}}\sphinxbfcode{\sphinxupquote{from\_config}}}{\emph{\DUrole{n}{config}}}{}
\pysigstopsignatures
\sphinxAtStartPar
Creates a layer from its config.

\sphinxAtStartPar
This method is the reverse of \sphinxtitleref{get\_config},
capable of instantiating the same layer from the config
dictionary. It does not handle layer connectivity
(handled by Network), nor weights (handled by \sphinxtitleref{set\_weights}).
\begin{quote}\begin{description}
\sphinxlineitem{Parameters}
\sphinxAtStartPar
\sphinxstyleliteralstrong{\sphinxupquote{config}} \textendash{} A Python dictionary, typically the
output of get\_config.

\sphinxlineitem{Returns}
\sphinxAtStartPar
A layer instance.

\end{description}\end{quote}

\end{fulllineitems}

\index{get\_config() (beyondml.tflow.layers.MaskedDense.MaskedDense method)@\spxentry{get\_config()}\spxextra{beyondml.tflow.layers.MaskedDense.MaskedDense method}}

\begin{fulllineitems}
\phantomsection\label{\detokenize{beyondml.tflow.layers:beyondml.tflow.layers.MaskedDense.MaskedDense.get_config}}
\pysigstartsignatures
\pysiglinewithargsret{\sphinxbfcode{\sphinxupquote{get\_config}}}{}{}
\pysigstopsignatures
\sphinxAtStartPar
Returns the config of the layer.

\sphinxAtStartPar
A layer config is a Python dictionary (serializable)
containing the configuration of a layer.
The same layer can be reinstantiated later
(without its trained weights) from this configuration.

\sphinxAtStartPar
The config of a layer does not include connectivity
information, nor the layer class name. These are handled
by \sphinxtitleref{Network} (one layer of abstraction above).

\sphinxAtStartPar
Note that \sphinxtitleref{get\_config()} does not guarantee to return a fresh copy of
dict every time it is called. The callers should make a copy of the
returned dict if they want to modify it.
\begin{quote}\begin{description}
\sphinxlineitem{Returns}
\sphinxAtStartPar
Python dictionary.

\end{description}\end{quote}

\end{fulllineitems}

\index{set\_masks() (beyondml.tflow.layers.MaskedDense.MaskedDense method)@\spxentry{set\_masks()}\spxextra{beyondml.tflow.layers.MaskedDense.MaskedDense method}}

\begin{fulllineitems}
\phantomsection\label{\detokenize{beyondml.tflow.layers:beyondml.tflow.layers.MaskedDense.MaskedDense.set_masks}}
\pysigstartsignatures
\pysiglinewithargsret{\sphinxbfcode{\sphinxupquote{set\_masks}}}{\emph{\DUrole{n}{new\_masks}}}{}
\pysigstopsignatures
\sphinxAtStartPar
Set the masks for the layer
\begin{quote}\begin{description}
\sphinxlineitem{Parameters}
\sphinxAtStartPar
\sphinxstyleliteralstrong{\sphinxupquote{new\_masks}} (\sphinxstyleliteralemphasis{\sphinxupquote{list}}\sphinxstyleliteralemphasis{\sphinxupquote{ of }}\sphinxstyleliteralemphasis{\sphinxupquote{arrays}}\sphinxstyleliteralemphasis{\sphinxupquote{ or }}\sphinxstyleliteralemphasis{\sphinxupquote{array\sphinxhyphen{}likes}}) \textendash{} The new masks to set for the layer

\end{description}\end{quote}

\end{fulllineitems}


\end{fulllineitems}



\subparagraph{beyondml.tflow.layers.MultiConv2D module}
\label{\detokenize{beyondml.tflow.layers:module-beyondml.tflow.layers.MultiConv2D}}\label{\detokenize{beyondml.tflow.layers:beyondml-tflow-layers-multiconv2d-module}}\index{module@\spxentry{module}!beyondml.tflow.layers.MultiConv2D@\spxentry{beyondml.tflow.layers.MultiConv2D}}\index{beyondml.tflow.layers.MultiConv2D@\spxentry{beyondml.tflow.layers.MultiConv2D}!module@\spxentry{module}}\index{MultiConv2D (class in beyondml.tflow.layers.MultiConv2D)@\spxentry{MultiConv2D}\spxextra{class in beyondml.tflow.layers.MultiConv2D}}

\begin{fulllineitems}
\phantomsection\label{\detokenize{beyondml.tflow.layers:beyondml.tflow.layers.MultiConv2D.MultiConv2D}}
\pysigstartsignatures
\pysiglinewithargsret{\sphinxbfcode{\sphinxupquote{class\DUrole{w}{  }}}\sphinxcode{\sphinxupquote{beyondml.tflow.layers.MultiConv2D.}}\sphinxbfcode{\sphinxupquote{MultiConv2D}}}{\emph{\DUrole{o}{*}\DUrole{n}{args}}, \emph{\DUrole{o}{**}\DUrole{n}{kwargs}}}{}
\pysigstopsignatures
\sphinxAtStartPar
Bases: \sphinxcode{\sphinxupquote{Layer}}

\sphinxAtStartPar
Multitask 2\sphinxhyphen{}dimensional convolutional layer

\sphinxAtStartPar
This layer implements multiple stacks of convolutional weights to account for different ways individual
neurons activate for various tasks. It is expected that to train using the RSN2 algorithm that MultiMaskedConv2D
layers be used during training and then those layers be converted to this layer type.
\index{build() (beyondml.tflow.layers.MultiConv2D.MultiConv2D method)@\spxentry{build()}\spxextra{beyondml.tflow.layers.MultiConv2D.MultiConv2D method}}

\begin{fulllineitems}
\phantomsection\label{\detokenize{beyondml.tflow.layers:beyondml.tflow.layers.MultiConv2D.MultiConv2D.build}}
\pysigstartsignatures
\pysiglinewithargsret{\sphinxbfcode{\sphinxupquote{build}}}{\emph{\DUrole{n}{input\_shape}}}{}
\pysigstopsignatures
\sphinxAtStartPar
Build the layer in preparation to be trained or called. Should not be called directly,
but rather is called when the layer is added to a model

\end{fulllineitems}

\index{call() (beyondml.tflow.layers.MultiConv2D.MultiConv2D method)@\spxentry{call()}\spxextra{beyondml.tflow.layers.MultiConv2D.MultiConv2D method}}

\begin{fulllineitems}
\phantomsection\label{\detokenize{beyondml.tflow.layers:beyondml.tflow.layers.MultiConv2D.MultiConv2D.call}}
\pysigstartsignatures
\pysiglinewithargsret{\sphinxbfcode{\sphinxupquote{call}}}{\emph{\DUrole{n}{inputs}}}{}
\pysigstopsignatures
\sphinxAtStartPar
This is where the layer’s logic lives and is called upon inputs
\begin{quote}\begin{description}
\sphinxlineitem{Parameters}
\sphinxAtStartPar
\sphinxstyleliteralstrong{\sphinxupquote{inputs}} (\sphinxstyleliteralemphasis{\sphinxupquote{TensorFlow Tensor}}\sphinxstyleliteralemphasis{\sphinxupquote{ or }}\sphinxstyleliteralemphasis{\sphinxupquote{Tensor\sphinxhyphen{}like}}) \textendash{} The inputs to the layer

\sphinxlineitem{Returns}
\sphinxAtStartPar
\sphinxstylestrong{outputs} \textendash{} The outputs of the layer’s logic

\sphinxlineitem{Return type}
\sphinxAtStartPar
TensorFlow Tensor

\end{description}\end{quote}

\end{fulllineitems}

\index{from\_config() (beyondml.tflow.layers.MultiConv2D.MultiConv2D class method)@\spxentry{from\_config()}\spxextra{beyondml.tflow.layers.MultiConv2D.MultiConv2D class method}}

\begin{fulllineitems}
\phantomsection\label{\detokenize{beyondml.tflow.layers:beyondml.tflow.layers.MultiConv2D.MultiConv2D.from_config}}
\pysigstartsignatures
\pysiglinewithargsret{\sphinxbfcode{\sphinxupquote{classmethod\DUrole{w}{  }}}\sphinxbfcode{\sphinxupquote{from\_config}}}{\emph{\DUrole{n}{config}}}{}
\pysigstopsignatures
\sphinxAtStartPar
Creates a layer from its config.

\sphinxAtStartPar
This method is the reverse of \sphinxtitleref{get\_config},
capable of instantiating the same layer from the config
dictionary. It does not handle layer connectivity
(handled by Network), nor weights (handled by \sphinxtitleref{set\_weights}).
\begin{quote}\begin{description}
\sphinxlineitem{Parameters}
\sphinxAtStartPar
\sphinxstyleliteralstrong{\sphinxupquote{config}} \textendash{} A Python dictionary, typically the
output of get\_config.

\sphinxlineitem{Returns}
\sphinxAtStartPar
A layer instance.

\end{description}\end{quote}

\end{fulllineitems}

\index{get\_config() (beyondml.tflow.layers.MultiConv2D.MultiConv2D method)@\spxentry{get\_config()}\spxextra{beyondml.tflow.layers.MultiConv2D.MultiConv2D method}}

\begin{fulllineitems}
\phantomsection\label{\detokenize{beyondml.tflow.layers:beyondml.tflow.layers.MultiConv2D.MultiConv2D.get_config}}
\pysigstartsignatures
\pysiglinewithargsret{\sphinxbfcode{\sphinxupquote{get\_config}}}{}{}
\pysigstopsignatures
\sphinxAtStartPar
Returns the config of the layer.

\sphinxAtStartPar
A layer config is a Python dictionary (serializable)
containing the configuration of a layer.
The same layer can be reinstantiated later
(without its trained weights) from this configuration.

\sphinxAtStartPar
The config of a layer does not include connectivity
information, nor the layer class name. These are handled
by \sphinxtitleref{Network} (one layer of abstraction above).

\sphinxAtStartPar
Note that \sphinxtitleref{get\_config()} does not guarantee to return a fresh copy of
dict every time it is called. The callers should make a copy of the
returned dict if they want to modify it.
\begin{quote}\begin{description}
\sphinxlineitem{Returns}
\sphinxAtStartPar
Python dictionary.

\end{description}\end{quote}

\end{fulllineitems}

\index{kernel\_size (beyondml.tflow.layers.MultiConv2D.MultiConv2D property)@\spxentry{kernel\_size}\spxextra{beyondml.tflow.layers.MultiConv2D.MultiConv2D property}}

\begin{fulllineitems}
\phantomsection\label{\detokenize{beyondml.tflow.layers:beyondml.tflow.layers.MultiConv2D.MultiConv2D.kernel_size}}
\pysigstartsignatures
\pysigline{\sphinxbfcode{\sphinxupquote{property\DUrole{w}{  }}}\sphinxbfcode{\sphinxupquote{kernel\_size}}}
\pysigstopsignatures
\end{fulllineitems}


\end{fulllineitems}



\subparagraph{beyondml.tflow.layers.MultiConv3D module}
\label{\detokenize{beyondml.tflow.layers:module-beyondml.tflow.layers.MultiConv3D}}\label{\detokenize{beyondml.tflow.layers:beyondml-tflow-layers-multiconv3d-module}}\index{module@\spxentry{module}!beyondml.tflow.layers.MultiConv3D@\spxentry{beyondml.tflow.layers.MultiConv3D}}\index{beyondml.tflow.layers.MultiConv3D@\spxentry{beyondml.tflow.layers.MultiConv3D}!module@\spxentry{module}}\index{MultiConv3D (class in beyondml.tflow.layers.MultiConv3D)@\spxentry{MultiConv3D}\spxextra{class in beyondml.tflow.layers.MultiConv3D}}

\begin{fulllineitems}
\phantomsection\label{\detokenize{beyondml.tflow.layers:beyondml.tflow.layers.MultiConv3D.MultiConv3D}}
\pysigstartsignatures
\pysiglinewithargsret{\sphinxbfcode{\sphinxupquote{class\DUrole{w}{  }}}\sphinxcode{\sphinxupquote{beyondml.tflow.layers.MultiConv3D.}}\sphinxbfcode{\sphinxupquote{MultiConv3D}}}{\emph{\DUrole{o}{*}\DUrole{n}{args}}, \emph{\DUrole{o}{**}\DUrole{n}{kwargs}}}{}
\pysigstopsignatures
\sphinxAtStartPar
Bases: \sphinxcode{\sphinxupquote{Layer}}

\sphinxAtStartPar
Multitask 3\sphinxhyphen{}dimensional convolutional layer

\sphinxAtStartPar
This layer implements multiple stacks of convolutional weights to account for different ways individual
neurons activate for various tasks. It is expected that to train using the RSN2 algorithm that MultiMaskedConv3D
layers be used during training and then those layers be converted to this layer type.
\index{build() (beyondml.tflow.layers.MultiConv3D.MultiConv3D method)@\spxentry{build()}\spxextra{beyondml.tflow.layers.MultiConv3D.MultiConv3D method}}

\begin{fulllineitems}
\phantomsection\label{\detokenize{beyondml.tflow.layers:beyondml.tflow.layers.MultiConv3D.MultiConv3D.build}}
\pysigstartsignatures
\pysiglinewithargsret{\sphinxbfcode{\sphinxupquote{build}}}{\emph{\DUrole{n}{input\_shape}}}{}
\pysigstopsignatures
\sphinxAtStartPar
Build the layer in preparation to be trained or called. Should not be called directly,
but rather is called when the layer is added to a model

\end{fulllineitems}

\index{call() (beyondml.tflow.layers.MultiConv3D.MultiConv3D method)@\spxentry{call()}\spxextra{beyondml.tflow.layers.MultiConv3D.MultiConv3D method}}

\begin{fulllineitems}
\phantomsection\label{\detokenize{beyondml.tflow.layers:beyondml.tflow.layers.MultiConv3D.MultiConv3D.call}}
\pysigstartsignatures
\pysiglinewithargsret{\sphinxbfcode{\sphinxupquote{call}}}{\emph{\DUrole{n}{inputs}}}{}
\pysigstopsignatures
\sphinxAtStartPar
This is where the layer’s logic lives and is called upon inputs
\begin{quote}\begin{description}
\sphinxlineitem{Parameters}
\sphinxAtStartPar
\sphinxstyleliteralstrong{\sphinxupquote{inputs}} (\sphinxstyleliteralemphasis{\sphinxupquote{TensorFlow Tensor}}\sphinxstyleliteralemphasis{\sphinxupquote{ or }}\sphinxstyleliteralemphasis{\sphinxupquote{Tensor\sphinxhyphen{}like}}) \textendash{} The inputs to the layer

\sphinxlineitem{Returns}
\sphinxAtStartPar
\sphinxstylestrong{outputs} \textendash{} The outputs of the layer’s logic

\sphinxlineitem{Return type}
\sphinxAtStartPar
TensorFlow Tensor

\end{description}\end{quote}

\end{fulllineitems}

\index{from\_config() (beyondml.tflow.layers.MultiConv3D.MultiConv3D class method)@\spxentry{from\_config()}\spxextra{beyondml.tflow.layers.MultiConv3D.MultiConv3D class method}}

\begin{fulllineitems}
\phantomsection\label{\detokenize{beyondml.tflow.layers:beyondml.tflow.layers.MultiConv3D.MultiConv3D.from_config}}
\pysigstartsignatures
\pysiglinewithargsret{\sphinxbfcode{\sphinxupquote{classmethod\DUrole{w}{  }}}\sphinxbfcode{\sphinxupquote{from\_config}}}{\emph{\DUrole{n}{config}}}{}
\pysigstopsignatures
\sphinxAtStartPar
Creates a layer from its config.

\sphinxAtStartPar
This method is the reverse of \sphinxtitleref{get\_config},
capable of instantiating the same layer from the config
dictionary. It does not handle layer connectivity
(handled by Network), nor weights (handled by \sphinxtitleref{set\_weights}).
\begin{quote}\begin{description}
\sphinxlineitem{Parameters}
\sphinxAtStartPar
\sphinxstyleliteralstrong{\sphinxupquote{config}} \textendash{} A Python dictionary, typically the
output of get\_config.

\sphinxlineitem{Returns}
\sphinxAtStartPar
A layer instance.

\end{description}\end{quote}

\end{fulllineitems}

\index{get\_config() (beyondml.tflow.layers.MultiConv3D.MultiConv3D method)@\spxentry{get\_config()}\spxextra{beyondml.tflow.layers.MultiConv3D.MultiConv3D method}}

\begin{fulllineitems}
\phantomsection\label{\detokenize{beyondml.tflow.layers:beyondml.tflow.layers.MultiConv3D.MultiConv3D.get_config}}
\pysigstartsignatures
\pysiglinewithargsret{\sphinxbfcode{\sphinxupquote{get\_config}}}{}{}
\pysigstopsignatures
\sphinxAtStartPar
Returns the config of the layer.

\sphinxAtStartPar
A layer config is a Python dictionary (serializable)
containing the configuration of a layer.
The same layer can be reinstantiated later
(without its trained weights) from this configuration.

\sphinxAtStartPar
The config of a layer does not include connectivity
information, nor the layer class name. These are handled
by \sphinxtitleref{Network} (one layer of abstraction above).

\sphinxAtStartPar
Note that \sphinxtitleref{get\_config()} does not guarantee to return a fresh copy of
dict every time it is called. The callers should make a copy of the
returned dict if they want to modify it.
\begin{quote}\begin{description}
\sphinxlineitem{Returns}
\sphinxAtStartPar
Python dictionary.

\end{description}\end{quote}

\end{fulllineitems}

\index{kernel\_size (beyondml.tflow.layers.MultiConv3D.MultiConv3D property)@\spxentry{kernel\_size}\spxextra{beyondml.tflow.layers.MultiConv3D.MultiConv3D property}}

\begin{fulllineitems}
\phantomsection\label{\detokenize{beyondml.tflow.layers:beyondml.tflow.layers.MultiConv3D.MultiConv3D.kernel_size}}
\pysigstartsignatures
\pysigline{\sphinxbfcode{\sphinxupquote{property\DUrole{w}{  }}}\sphinxbfcode{\sphinxupquote{kernel\_size}}}
\pysigstopsignatures
\end{fulllineitems}


\end{fulllineitems}



\subparagraph{beyondml.tflow.layers.MultiDense module}
\label{\detokenize{beyondml.tflow.layers:module-beyondml.tflow.layers.MultiDense}}\label{\detokenize{beyondml.tflow.layers:beyondml-tflow-layers-multidense-module}}\index{module@\spxentry{module}!beyondml.tflow.layers.MultiDense@\spxentry{beyondml.tflow.layers.MultiDense}}\index{beyondml.tflow.layers.MultiDense@\spxentry{beyondml.tflow.layers.MultiDense}!module@\spxentry{module}}\index{MultiDense (class in beyondml.tflow.layers.MultiDense)@\spxentry{MultiDense}\spxextra{class in beyondml.tflow.layers.MultiDense}}

\begin{fulllineitems}
\phantomsection\label{\detokenize{beyondml.tflow.layers:beyondml.tflow.layers.MultiDense.MultiDense}}
\pysigstartsignatures
\pysiglinewithargsret{\sphinxbfcode{\sphinxupquote{class\DUrole{w}{  }}}\sphinxcode{\sphinxupquote{beyondml.tflow.layers.MultiDense.}}\sphinxbfcode{\sphinxupquote{MultiDense}}}{\emph{\DUrole{o}{*}\DUrole{n}{args}}, \emph{\DUrole{o}{**}\DUrole{n}{kwargs}}}{}
\pysigstopsignatures
\sphinxAtStartPar
Bases: \sphinxcode{\sphinxupquote{Layer}}

\sphinxAtStartPar
Multitask fully connected layer

\sphinxAtStartPar
This layer implements multiple stacks of fully connected weights to account for different
ways neurons can activate for various tasks. It is expected that to train using the RSN2 algorithm
that MultiMaskedDense layers be used during training and then those layers be converted to this layer type.
\index{build() (beyondml.tflow.layers.MultiDense.MultiDense method)@\spxentry{build()}\spxextra{beyondml.tflow.layers.MultiDense.MultiDense method}}

\begin{fulllineitems}
\phantomsection\label{\detokenize{beyondml.tflow.layers:beyondml.tflow.layers.MultiDense.MultiDense.build}}
\pysigstartsignatures
\pysiglinewithargsret{\sphinxbfcode{\sphinxupquote{build}}}{\emph{\DUrole{n}{input\_shape}}}{}
\pysigstopsignatures
\sphinxAtStartPar
Build the layer in preparation to be trained or called. Should not be called directly,
but rather is called when the layer is added to a model

\end{fulllineitems}

\index{call() (beyondml.tflow.layers.MultiDense.MultiDense method)@\spxentry{call()}\spxextra{beyondml.tflow.layers.MultiDense.MultiDense method}}

\begin{fulllineitems}
\phantomsection\label{\detokenize{beyondml.tflow.layers:beyondml.tflow.layers.MultiDense.MultiDense.call}}
\pysigstartsignatures
\pysiglinewithargsret{\sphinxbfcode{\sphinxupquote{call}}}{\emph{\DUrole{n}{inputs}}}{}
\pysigstopsignatures
\sphinxAtStartPar
This is where the layer’s logic lives and is called upon inputs
\begin{quote}\begin{description}
\sphinxlineitem{Parameters}
\sphinxAtStartPar
\sphinxstyleliteralstrong{\sphinxupquote{inputs}} (\sphinxstyleliteralemphasis{\sphinxupquote{TensorFlow Tensor}}\sphinxstyleliteralemphasis{\sphinxupquote{ or }}\sphinxstyleliteralemphasis{\sphinxupquote{Tensor\sphinxhyphen{}like}}) \textendash{} The inputs to the layer

\sphinxlineitem{Returns}
\sphinxAtStartPar
\sphinxstylestrong{outputs} \textendash{} The outputs of the layer’s logic

\sphinxlineitem{Return type}
\sphinxAtStartPar
TensorFlow Tensor

\end{description}\end{quote}

\end{fulllineitems}

\index{from\_config() (beyondml.tflow.layers.MultiDense.MultiDense class method)@\spxentry{from\_config()}\spxextra{beyondml.tflow.layers.MultiDense.MultiDense class method}}

\begin{fulllineitems}
\phantomsection\label{\detokenize{beyondml.tflow.layers:beyondml.tflow.layers.MultiDense.MultiDense.from_config}}
\pysigstartsignatures
\pysiglinewithargsret{\sphinxbfcode{\sphinxupquote{classmethod\DUrole{w}{  }}}\sphinxbfcode{\sphinxupquote{from\_config}}}{\emph{\DUrole{n}{config}}}{}
\pysigstopsignatures
\sphinxAtStartPar
Creates a layer from its config.

\sphinxAtStartPar
This method is the reverse of \sphinxtitleref{get\_config},
capable of instantiating the same layer from the config
dictionary. It does not handle layer connectivity
(handled by Network), nor weights (handled by \sphinxtitleref{set\_weights}).
\begin{quote}\begin{description}
\sphinxlineitem{Parameters}
\sphinxAtStartPar
\sphinxstyleliteralstrong{\sphinxupquote{config}} \textendash{} A Python dictionary, typically the
output of get\_config.

\sphinxlineitem{Returns}
\sphinxAtStartPar
A layer instance.

\end{description}\end{quote}

\end{fulllineitems}

\index{get\_config() (beyondml.tflow.layers.MultiDense.MultiDense method)@\spxentry{get\_config()}\spxextra{beyondml.tflow.layers.MultiDense.MultiDense method}}

\begin{fulllineitems}
\phantomsection\label{\detokenize{beyondml.tflow.layers:beyondml.tflow.layers.MultiDense.MultiDense.get_config}}
\pysigstartsignatures
\pysiglinewithargsret{\sphinxbfcode{\sphinxupquote{get\_config}}}{}{}
\pysigstopsignatures
\sphinxAtStartPar
Returns the config of the layer.

\sphinxAtStartPar
A layer config is a Python dictionary (serializable)
containing the configuration of a layer.
The same layer can be reinstantiated later
(without its trained weights) from this configuration.

\sphinxAtStartPar
The config of a layer does not include connectivity
information, nor the layer class name. These are handled
by \sphinxtitleref{Network} (one layer of abstraction above).

\sphinxAtStartPar
Note that \sphinxtitleref{get\_config()} does not guarantee to return a fresh copy of
dict every time it is called. The callers should make a copy of the
returned dict if they want to modify it.
\begin{quote}\begin{description}
\sphinxlineitem{Returns}
\sphinxAtStartPar
Python dictionary.

\end{description}\end{quote}

\end{fulllineitems}


\end{fulllineitems}



\subparagraph{beyondml.tflow.layers.MultiMaskedConv2D module}
\label{\detokenize{beyondml.tflow.layers:module-beyondml.tflow.layers.MultiMaskedConv2D}}\label{\detokenize{beyondml.tflow.layers:beyondml-tflow-layers-multimaskedconv2d-module}}\index{module@\spxentry{module}!beyondml.tflow.layers.MultiMaskedConv2D@\spxentry{beyondml.tflow.layers.MultiMaskedConv2D}}\index{beyondml.tflow.layers.MultiMaskedConv2D@\spxentry{beyondml.tflow.layers.MultiMaskedConv2D}!module@\spxentry{module}}\index{MultiMaskedConv2D (class in beyondml.tflow.layers.MultiMaskedConv2D)@\spxentry{MultiMaskedConv2D}\spxextra{class in beyondml.tflow.layers.MultiMaskedConv2D}}

\begin{fulllineitems}
\phantomsection\label{\detokenize{beyondml.tflow.layers:beyondml.tflow.layers.MultiMaskedConv2D.MultiMaskedConv2D}}
\pysigstartsignatures
\pysiglinewithargsret{\sphinxbfcode{\sphinxupquote{class\DUrole{w}{  }}}\sphinxcode{\sphinxupquote{beyondml.tflow.layers.MultiMaskedConv2D.}}\sphinxbfcode{\sphinxupquote{MultiMaskedConv2D}}}{\emph{\DUrole{o}{*}\DUrole{n}{args}}, \emph{\DUrole{o}{**}\DUrole{n}{kwargs}}}{}
\pysigstopsignatures
\sphinxAtStartPar
Bases: \sphinxcode{\sphinxupquote{Layer}}

\sphinxAtStartPar
Masked multitask 2\sphinxhyphen{}dimensional convolutional layer. This layer implements
multiple stacks of the convolutional architecture and implements masking consistent
with the BeyondML API to support developing sparse multitask models.
\index{build() (beyondml.tflow.layers.MultiMaskedConv2D.MultiMaskedConv2D method)@\spxentry{build()}\spxextra{beyondml.tflow.layers.MultiMaskedConv2D.MultiMaskedConv2D method}}

\begin{fulllineitems}
\phantomsection\label{\detokenize{beyondml.tflow.layers:beyondml.tflow.layers.MultiMaskedConv2D.MultiMaskedConv2D.build}}
\pysigstartsignatures
\pysiglinewithargsret{\sphinxbfcode{\sphinxupquote{build}}}{\emph{\DUrole{n}{input\_shape}}}{}
\pysigstopsignatures
\sphinxAtStartPar
Build the layer in preparation to be trained or called. Should not be called directly,
but rather is called when the layer is added to a model

\end{fulllineitems}

\index{call() (beyondml.tflow.layers.MultiMaskedConv2D.MultiMaskedConv2D method)@\spxentry{call()}\spxextra{beyondml.tflow.layers.MultiMaskedConv2D.MultiMaskedConv2D method}}

\begin{fulllineitems}
\phantomsection\label{\detokenize{beyondml.tflow.layers:beyondml.tflow.layers.MultiMaskedConv2D.MultiMaskedConv2D.call}}
\pysigstartsignatures
\pysiglinewithargsret{\sphinxbfcode{\sphinxupquote{call}}}{\emph{\DUrole{n}{inputs}}}{}
\pysigstopsignatures
\sphinxAtStartPar
This is where the layer’s logic lives and is called upon inputs
\begin{quote}\begin{description}
\sphinxlineitem{Parameters}
\sphinxAtStartPar
\sphinxstyleliteralstrong{\sphinxupquote{inputs}} (\sphinxstyleliteralemphasis{\sphinxupquote{TensorFlow Tensor}}\sphinxstyleliteralemphasis{\sphinxupquote{ or }}\sphinxstyleliteralemphasis{\sphinxupquote{Tensor\sphinxhyphen{}like}}) \textendash{} The inputs to the layer

\sphinxlineitem{Returns}
\sphinxAtStartPar
\sphinxstylestrong{outputs} \textendash{} The outputs of the layer’s logic

\sphinxlineitem{Return type}
\sphinxAtStartPar
TensorFlow Tensor

\end{description}\end{quote}

\end{fulllineitems}

\index{from\_config() (beyondml.tflow.layers.MultiMaskedConv2D.MultiMaskedConv2D class method)@\spxentry{from\_config()}\spxextra{beyondml.tflow.layers.MultiMaskedConv2D.MultiMaskedConv2D class method}}

\begin{fulllineitems}
\phantomsection\label{\detokenize{beyondml.tflow.layers:beyondml.tflow.layers.MultiMaskedConv2D.MultiMaskedConv2D.from_config}}
\pysigstartsignatures
\pysiglinewithargsret{\sphinxbfcode{\sphinxupquote{classmethod\DUrole{w}{  }}}\sphinxbfcode{\sphinxupquote{from\_config}}}{\emph{\DUrole{n}{config}}}{}
\pysigstopsignatures
\sphinxAtStartPar
Creates a layer from its config.

\sphinxAtStartPar
This method is the reverse of \sphinxtitleref{get\_config},
capable of instantiating the same layer from the config
dictionary. It does not handle layer connectivity
(handled by Network), nor weights (handled by \sphinxtitleref{set\_weights}).
\begin{quote}\begin{description}
\sphinxlineitem{Parameters}
\sphinxAtStartPar
\sphinxstyleliteralstrong{\sphinxupquote{config}} \textendash{} A Python dictionary, typically the
output of get\_config.

\sphinxlineitem{Returns}
\sphinxAtStartPar
A layer instance.

\end{description}\end{quote}

\end{fulllineitems}

\index{get\_config() (beyondml.tflow.layers.MultiMaskedConv2D.MultiMaskedConv2D method)@\spxentry{get\_config()}\spxextra{beyondml.tflow.layers.MultiMaskedConv2D.MultiMaskedConv2D method}}

\begin{fulllineitems}
\phantomsection\label{\detokenize{beyondml.tflow.layers:beyondml.tflow.layers.MultiMaskedConv2D.MultiMaskedConv2D.get_config}}
\pysigstartsignatures
\pysiglinewithargsret{\sphinxbfcode{\sphinxupquote{get\_config}}}{}{}
\pysigstopsignatures
\sphinxAtStartPar
Returns the config of the layer.

\sphinxAtStartPar
A layer config is a Python dictionary (serializable)
containing the configuration of a layer.
The same layer can be reinstantiated later
(without its trained weights) from this configuration.

\sphinxAtStartPar
The config of a layer does not include connectivity
information, nor the layer class name. These are handled
by \sphinxtitleref{Network} (one layer of abstraction above).

\sphinxAtStartPar
Note that \sphinxtitleref{get\_config()} does not guarantee to return a fresh copy of
dict every time it is called. The callers should make a copy of the
returned dict if they want to modify it.
\begin{quote}\begin{description}
\sphinxlineitem{Returns}
\sphinxAtStartPar
Python dictionary.

\end{description}\end{quote}

\end{fulllineitems}

\index{kernel\_size (beyondml.tflow.layers.MultiMaskedConv2D.MultiMaskedConv2D property)@\spxentry{kernel\_size}\spxextra{beyondml.tflow.layers.MultiMaskedConv2D.MultiMaskedConv2D property}}

\begin{fulllineitems}
\phantomsection\label{\detokenize{beyondml.tflow.layers:beyondml.tflow.layers.MultiMaskedConv2D.MultiMaskedConv2D.kernel_size}}
\pysigstartsignatures
\pysigline{\sphinxbfcode{\sphinxupquote{property\DUrole{w}{  }}}\sphinxbfcode{\sphinxupquote{kernel\_size}}}
\pysigstopsignatures
\end{fulllineitems}

\index{set\_masks() (beyondml.tflow.layers.MultiMaskedConv2D.MultiMaskedConv2D method)@\spxentry{set\_masks()}\spxextra{beyondml.tflow.layers.MultiMaskedConv2D.MultiMaskedConv2D method}}

\begin{fulllineitems}
\phantomsection\label{\detokenize{beyondml.tflow.layers:beyondml.tflow.layers.MultiMaskedConv2D.MultiMaskedConv2D.set_masks}}
\pysigstartsignatures
\pysiglinewithargsret{\sphinxbfcode{\sphinxupquote{set\_masks}}}{\emph{\DUrole{n}{new\_masks}}}{}
\pysigstopsignatures
\end{fulllineitems}


\end{fulllineitems}



\subparagraph{beyondml.tflow.layers.MultiMaskedConv3D module}
\label{\detokenize{beyondml.tflow.layers:module-beyondml.tflow.layers.MultiMaskedConv3D}}\label{\detokenize{beyondml.tflow.layers:beyondml-tflow-layers-multimaskedconv3d-module}}\index{module@\spxentry{module}!beyondml.tflow.layers.MultiMaskedConv3D@\spxentry{beyondml.tflow.layers.MultiMaskedConv3D}}\index{beyondml.tflow.layers.MultiMaskedConv3D@\spxentry{beyondml.tflow.layers.MultiMaskedConv3D}!module@\spxentry{module}}\index{MultiMaskedConv3D (class in beyondml.tflow.layers.MultiMaskedConv3D)@\spxentry{MultiMaskedConv3D}\spxextra{class in beyondml.tflow.layers.MultiMaskedConv3D}}

\begin{fulllineitems}
\phantomsection\label{\detokenize{beyondml.tflow.layers:beyondml.tflow.layers.MultiMaskedConv3D.MultiMaskedConv3D}}
\pysigstartsignatures
\pysiglinewithargsret{\sphinxbfcode{\sphinxupquote{class\DUrole{w}{  }}}\sphinxcode{\sphinxupquote{beyondml.tflow.layers.MultiMaskedConv3D.}}\sphinxbfcode{\sphinxupquote{MultiMaskedConv3D}}}{\emph{\DUrole{o}{*}\DUrole{n}{args}}, \emph{\DUrole{o}{**}\DUrole{n}{kwargs}}}{}
\pysigstopsignatures
\sphinxAtStartPar
Bases: \sphinxcode{\sphinxupquote{Layer}}

\sphinxAtStartPar
Masked multitask 3\sphinxhyphen{}dimensional convoluational layer. This layer implements
multiple stacks of the convolutional architecture and implements masking
consistent with the BeyondML API to support developing sparse multitask models.
\index{build() (beyondml.tflow.layers.MultiMaskedConv3D.MultiMaskedConv3D method)@\spxentry{build()}\spxextra{beyondml.tflow.layers.MultiMaskedConv3D.MultiMaskedConv3D method}}

\begin{fulllineitems}
\phantomsection\label{\detokenize{beyondml.tflow.layers:beyondml.tflow.layers.MultiMaskedConv3D.MultiMaskedConv3D.build}}
\pysigstartsignatures
\pysiglinewithargsret{\sphinxbfcode{\sphinxupquote{build}}}{\emph{\DUrole{n}{input\_shape}}}{}
\pysigstopsignatures
\sphinxAtStartPar
Build the layer in preparation to be trained or called. Should not be called directly,
but rather is called when the layer is added to a model

\end{fulllineitems}

\index{call() (beyondml.tflow.layers.MultiMaskedConv3D.MultiMaskedConv3D method)@\spxentry{call()}\spxextra{beyondml.tflow.layers.MultiMaskedConv3D.MultiMaskedConv3D method}}

\begin{fulllineitems}
\phantomsection\label{\detokenize{beyondml.tflow.layers:beyondml.tflow.layers.MultiMaskedConv3D.MultiMaskedConv3D.call}}
\pysigstartsignatures
\pysiglinewithargsret{\sphinxbfcode{\sphinxupquote{call}}}{\emph{\DUrole{n}{inputs}}}{}
\pysigstopsignatures
\sphinxAtStartPar
This is where the layer’s logic lives and is called upon inputs
\begin{quote}\begin{description}
\sphinxlineitem{Parameters}
\sphinxAtStartPar
\sphinxstyleliteralstrong{\sphinxupquote{inputs}} (\sphinxstyleliteralemphasis{\sphinxupquote{TensorFlow Tensor}}\sphinxstyleliteralemphasis{\sphinxupquote{ or }}\sphinxstyleliteralemphasis{\sphinxupquote{Tensor\sphinxhyphen{}like}}) \textendash{} The inputs to the layer

\sphinxlineitem{Returns}
\sphinxAtStartPar
\sphinxstylestrong{outputs} \textendash{} The outputs of the layer’s logic

\sphinxlineitem{Return type}
\sphinxAtStartPar
TensorFlow Tensor

\end{description}\end{quote}

\end{fulllineitems}

\index{from\_config() (beyondml.tflow.layers.MultiMaskedConv3D.MultiMaskedConv3D class method)@\spxentry{from\_config()}\spxextra{beyondml.tflow.layers.MultiMaskedConv3D.MultiMaskedConv3D class method}}

\begin{fulllineitems}
\phantomsection\label{\detokenize{beyondml.tflow.layers:beyondml.tflow.layers.MultiMaskedConv3D.MultiMaskedConv3D.from_config}}
\pysigstartsignatures
\pysiglinewithargsret{\sphinxbfcode{\sphinxupquote{classmethod\DUrole{w}{  }}}\sphinxbfcode{\sphinxupquote{from\_config}}}{\emph{\DUrole{n}{config}}}{}
\pysigstopsignatures
\sphinxAtStartPar
Creates a layer from its config.

\sphinxAtStartPar
This method is the reverse of \sphinxtitleref{get\_config},
capable of instantiating the same layer from the config
dictionary. It does not handle layer connectivity
(handled by Network), nor weights (handled by \sphinxtitleref{set\_weights}).
\begin{quote}\begin{description}
\sphinxlineitem{Parameters}
\sphinxAtStartPar
\sphinxstyleliteralstrong{\sphinxupquote{config}} \textendash{} A Python dictionary, typically the
output of get\_config.

\sphinxlineitem{Returns}
\sphinxAtStartPar
A layer instance.

\end{description}\end{quote}

\end{fulllineitems}

\index{get\_config() (beyondml.tflow.layers.MultiMaskedConv3D.MultiMaskedConv3D method)@\spxentry{get\_config()}\spxextra{beyondml.tflow.layers.MultiMaskedConv3D.MultiMaskedConv3D method}}

\begin{fulllineitems}
\phantomsection\label{\detokenize{beyondml.tflow.layers:beyondml.tflow.layers.MultiMaskedConv3D.MultiMaskedConv3D.get_config}}
\pysigstartsignatures
\pysiglinewithargsret{\sphinxbfcode{\sphinxupquote{get\_config}}}{}{}
\pysigstopsignatures
\sphinxAtStartPar
Returns the config of the layer.

\sphinxAtStartPar
A layer config is a Python dictionary (serializable)
containing the configuration of a layer.
The same layer can be reinstantiated later
(without its trained weights) from this configuration.

\sphinxAtStartPar
The config of a layer does not include connectivity
information, nor the layer class name. These are handled
by \sphinxtitleref{Network} (one layer of abstraction above).

\sphinxAtStartPar
Note that \sphinxtitleref{get\_config()} does not guarantee to return a fresh copy of
dict every time it is called. The callers should make a copy of the
returned dict if they want to modify it.
\begin{quote}\begin{description}
\sphinxlineitem{Returns}
\sphinxAtStartPar
Python dictionary.

\end{description}\end{quote}

\end{fulllineitems}

\index{kernel\_size (beyondml.tflow.layers.MultiMaskedConv3D.MultiMaskedConv3D property)@\spxentry{kernel\_size}\spxextra{beyondml.tflow.layers.MultiMaskedConv3D.MultiMaskedConv3D property}}

\begin{fulllineitems}
\phantomsection\label{\detokenize{beyondml.tflow.layers:beyondml.tflow.layers.MultiMaskedConv3D.MultiMaskedConv3D.kernel_size}}
\pysigstartsignatures
\pysigline{\sphinxbfcode{\sphinxupquote{property\DUrole{w}{  }}}\sphinxbfcode{\sphinxupquote{kernel\_size}}}
\pysigstopsignatures
\end{fulllineitems}

\index{set\_masks() (beyondml.tflow.layers.MultiMaskedConv3D.MultiMaskedConv3D method)@\spxentry{set\_masks()}\spxextra{beyondml.tflow.layers.MultiMaskedConv3D.MultiMaskedConv3D method}}

\begin{fulllineitems}
\phantomsection\label{\detokenize{beyondml.tflow.layers:beyondml.tflow.layers.MultiMaskedConv3D.MultiMaskedConv3D.set_masks}}
\pysigstartsignatures
\pysiglinewithargsret{\sphinxbfcode{\sphinxupquote{set\_masks}}}{\emph{\DUrole{n}{new\_masks}}}{}
\pysigstopsignatures
\end{fulllineitems}


\end{fulllineitems}



\subparagraph{beyondml.tflow.layers.MultiMaskedDense module}
\label{\detokenize{beyondml.tflow.layers:module-beyondml.tflow.layers.MultiMaskedDense}}\label{\detokenize{beyondml.tflow.layers:beyondml-tflow-layers-multimaskeddense-module}}\index{module@\spxentry{module}!beyondml.tflow.layers.MultiMaskedDense@\spxentry{beyondml.tflow.layers.MultiMaskedDense}}\index{beyondml.tflow.layers.MultiMaskedDense@\spxentry{beyondml.tflow.layers.MultiMaskedDense}!module@\spxentry{module}}\index{MultiMaskedDense (class in beyondml.tflow.layers.MultiMaskedDense)@\spxentry{MultiMaskedDense}\spxextra{class in beyondml.tflow.layers.MultiMaskedDense}}

\begin{fulllineitems}
\phantomsection\label{\detokenize{beyondml.tflow.layers:beyondml.tflow.layers.MultiMaskedDense.MultiMaskedDense}}
\pysigstartsignatures
\pysiglinewithargsret{\sphinxbfcode{\sphinxupquote{class\DUrole{w}{  }}}\sphinxcode{\sphinxupquote{beyondml.tflow.layers.MultiMaskedDense.}}\sphinxbfcode{\sphinxupquote{MultiMaskedDense}}}{\emph{\DUrole{o}{*}\DUrole{n}{args}}, \emph{\DUrole{o}{**}\DUrole{n}{kwargs}}}{}
\pysigstopsignatures
\sphinxAtStartPar
Bases: \sphinxcode{\sphinxupquote{Layer}}

\sphinxAtStartPar
Masked multitask fully connected layer. This layer implements multiple stacks
of the fully\sphinxhyphen{}connected architecture and implements masking with the BeyondML API
to support developing sparse multitask models.
\index{build() (beyondml.tflow.layers.MultiMaskedDense.MultiMaskedDense method)@\spxentry{build()}\spxextra{beyondml.tflow.layers.MultiMaskedDense.MultiMaskedDense method}}

\begin{fulllineitems}
\phantomsection\label{\detokenize{beyondml.tflow.layers:beyondml.tflow.layers.MultiMaskedDense.MultiMaskedDense.build}}
\pysigstartsignatures
\pysiglinewithargsret{\sphinxbfcode{\sphinxupquote{build}}}{\emph{\DUrole{n}{input\_shape}}}{}
\pysigstopsignatures
\sphinxAtStartPar
Build the layer in preparation to be trained or called. Should not be called directly,
but rather is called when the layer is added to a model

\end{fulllineitems}

\index{call() (beyondml.tflow.layers.MultiMaskedDense.MultiMaskedDense method)@\spxentry{call()}\spxextra{beyondml.tflow.layers.MultiMaskedDense.MultiMaskedDense method}}

\begin{fulllineitems}
\phantomsection\label{\detokenize{beyondml.tflow.layers:beyondml.tflow.layers.MultiMaskedDense.MultiMaskedDense.call}}
\pysigstartsignatures
\pysiglinewithargsret{\sphinxbfcode{\sphinxupquote{call}}}{\emph{\DUrole{n}{inputs}}}{}
\pysigstopsignatures
\sphinxAtStartPar
This is where the layer’s logic lives and is called upon inputs
\begin{quote}\begin{description}
\sphinxlineitem{Parameters}
\sphinxAtStartPar
\sphinxstyleliteralstrong{\sphinxupquote{inputs}} (\sphinxstyleliteralemphasis{\sphinxupquote{TensorFlow Tensor}}\sphinxstyleliteralemphasis{\sphinxupquote{ or }}\sphinxstyleliteralemphasis{\sphinxupquote{Tensor\sphinxhyphen{}like}}) \textendash{} The inputs to the layer

\sphinxlineitem{Returns}
\sphinxAtStartPar
\sphinxstylestrong{outputs} \textendash{} The outputs of the layer’s logic

\sphinxlineitem{Return type}
\sphinxAtStartPar
TensorFlow Tensor

\end{description}\end{quote}

\end{fulllineitems}

\index{from\_config() (beyondml.tflow.layers.MultiMaskedDense.MultiMaskedDense class method)@\spxentry{from\_config()}\spxextra{beyondml.tflow.layers.MultiMaskedDense.MultiMaskedDense class method}}

\begin{fulllineitems}
\phantomsection\label{\detokenize{beyondml.tflow.layers:beyondml.tflow.layers.MultiMaskedDense.MultiMaskedDense.from_config}}
\pysigstartsignatures
\pysiglinewithargsret{\sphinxbfcode{\sphinxupquote{classmethod\DUrole{w}{  }}}\sphinxbfcode{\sphinxupquote{from\_config}}}{\emph{\DUrole{n}{config}}}{}
\pysigstopsignatures
\sphinxAtStartPar
Creates a layer from its config.

\sphinxAtStartPar
This method is the reverse of \sphinxtitleref{get\_config},
capable of instantiating the same layer from the config
dictionary. It does not handle layer connectivity
(handled by Network), nor weights (handled by \sphinxtitleref{set\_weights}).
\begin{quote}\begin{description}
\sphinxlineitem{Parameters}
\sphinxAtStartPar
\sphinxstyleliteralstrong{\sphinxupquote{config}} \textendash{} A Python dictionary, typically the
output of get\_config.

\sphinxlineitem{Returns}
\sphinxAtStartPar
A layer instance.

\end{description}\end{quote}

\end{fulllineitems}

\index{get\_config() (beyondml.tflow.layers.MultiMaskedDense.MultiMaskedDense method)@\spxentry{get\_config()}\spxextra{beyondml.tflow.layers.MultiMaskedDense.MultiMaskedDense method}}

\begin{fulllineitems}
\phantomsection\label{\detokenize{beyondml.tflow.layers:beyondml.tflow.layers.MultiMaskedDense.MultiMaskedDense.get_config}}
\pysigstartsignatures
\pysiglinewithargsret{\sphinxbfcode{\sphinxupquote{get\_config}}}{}{}
\pysigstopsignatures
\sphinxAtStartPar
Returns the config of the layer.

\sphinxAtStartPar
A layer config is a Python dictionary (serializable)
containing the configuration of a layer.
The same layer can be reinstantiated later
(without its trained weights) from this configuration.

\sphinxAtStartPar
The config of a layer does not include connectivity
information, nor the layer class name. These are handled
by \sphinxtitleref{Network} (one layer of abstraction above).

\sphinxAtStartPar
Note that \sphinxtitleref{get\_config()} does not guarantee to return a fresh copy of
dict every time it is called. The callers should make a copy of the
returned dict if they want to modify it.
\begin{quote}\begin{description}
\sphinxlineitem{Returns}
\sphinxAtStartPar
Python dictionary.

\end{description}\end{quote}

\end{fulllineitems}

\index{set\_masks() (beyondml.tflow.layers.MultiMaskedDense.MultiMaskedDense method)@\spxentry{set\_masks()}\spxextra{beyondml.tflow.layers.MultiMaskedDense.MultiMaskedDense method}}

\begin{fulllineitems}
\phantomsection\label{\detokenize{beyondml.tflow.layers:beyondml.tflow.layers.MultiMaskedDense.MultiMaskedDense.set_masks}}
\pysigstartsignatures
\pysiglinewithargsret{\sphinxbfcode{\sphinxupquote{set\_masks}}}{\emph{\DUrole{n}{new\_masks}}}{}
\pysigstopsignatures
\sphinxAtStartPar
Set the masks for the layer
\begin{quote}\begin{description}
\sphinxlineitem{Parameters}
\sphinxAtStartPar
\sphinxstyleliteralstrong{\sphinxupquote{new\_masks}} (\sphinxstyleliteralemphasis{\sphinxupquote{list}}\sphinxstyleliteralemphasis{\sphinxupquote{ of }}\sphinxstyleliteralemphasis{\sphinxupquote{arrays}}\sphinxstyleliteralemphasis{\sphinxupquote{ or }}\sphinxstyleliteralemphasis{\sphinxupquote{array\sphinxhyphen{}likes}}) \textendash{} The new masks to set for the layer

\end{description}\end{quote}

\end{fulllineitems}


\end{fulllineitems}



\subparagraph{beyondml.tflow.layers.MultiMaxPool2D module}
\label{\detokenize{beyondml.tflow.layers:module-beyondml.tflow.layers.MultiMaxPool2D}}\label{\detokenize{beyondml.tflow.layers:beyondml-tflow-layers-multimaxpool2d-module}}\index{module@\spxentry{module}!beyondml.tflow.layers.MultiMaxPool2D@\spxentry{beyondml.tflow.layers.MultiMaxPool2D}}\index{beyondml.tflow.layers.MultiMaxPool2D@\spxentry{beyondml.tflow.layers.MultiMaxPool2D}!module@\spxentry{module}}\index{MultiMaxPool2D (class in beyondml.tflow.layers.MultiMaxPool2D)@\spxentry{MultiMaxPool2D}\spxextra{class in beyondml.tflow.layers.MultiMaxPool2D}}

\begin{fulllineitems}
\phantomsection\label{\detokenize{beyondml.tflow.layers:beyondml.tflow.layers.MultiMaxPool2D.MultiMaxPool2D}}
\pysigstartsignatures
\pysiglinewithargsret{\sphinxbfcode{\sphinxupquote{class\DUrole{w}{  }}}\sphinxcode{\sphinxupquote{beyondml.tflow.layers.MultiMaxPool2D.}}\sphinxbfcode{\sphinxupquote{MultiMaxPool2D}}}{\emph{\DUrole{o}{*}\DUrole{n}{args}}, \emph{\DUrole{o}{**}\DUrole{n}{kwargs}}}{}
\pysigstopsignatures
\sphinxAtStartPar
Bases: \sphinxcode{\sphinxupquote{Layer}}

\sphinxAtStartPar
Multitask Max Pooling Layer. This layer implements the Max Pooling algorithm
across multiple inputs for developing multitask models
\index{call() (beyondml.tflow.layers.MultiMaxPool2D.MultiMaxPool2D method)@\spxentry{call()}\spxextra{beyondml.tflow.layers.MultiMaxPool2D.MultiMaxPool2D method}}

\begin{fulllineitems}
\phantomsection\label{\detokenize{beyondml.tflow.layers:beyondml.tflow.layers.MultiMaxPool2D.MultiMaxPool2D.call}}
\pysigstartsignatures
\pysiglinewithargsret{\sphinxbfcode{\sphinxupquote{call}}}{\emph{\DUrole{n}{inputs}}}{}
\pysigstopsignatures
\sphinxAtStartPar
This is where the layer’s logic lives and is called upon inputs
\begin{quote}\begin{description}
\sphinxlineitem{Parameters}
\sphinxAtStartPar
\sphinxstyleliteralstrong{\sphinxupquote{inputs}} (\sphinxstyleliteralemphasis{\sphinxupquote{TensorFlow Tensor}}\sphinxstyleliteralemphasis{\sphinxupquote{ or }}\sphinxstyleliteralemphasis{\sphinxupquote{Tensor\sphinxhyphen{}like}}) \textendash{} The inputs to the layer

\sphinxlineitem{Returns}
\sphinxAtStartPar
\sphinxstylestrong{outputs} \textendash{} The outputs of the layer’s logic

\sphinxlineitem{Return type}
\sphinxAtStartPar
TensorFlow Tensor

\end{description}\end{quote}

\end{fulllineitems}

\index{from\_config() (beyondml.tflow.layers.MultiMaxPool2D.MultiMaxPool2D class method)@\spxentry{from\_config()}\spxextra{beyondml.tflow.layers.MultiMaxPool2D.MultiMaxPool2D class method}}

\begin{fulllineitems}
\phantomsection\label{\detokenize{beyondml.tflow.layers:beyondml.tflow.layers.MultiMaxPool2D.MultiMaxPool2D.from_config}}
\pysigstartsignatures
\pysiglinewithargsret{\sphinxbfcode{\sphinxupquote{classmethod\DUrole{w}{  }}}\sphinxbfcode{\sphinxupquote{from\_config}}}{\emph{\DUrole{n}{config}}}{}
\pysigstopsignatures
\sphinxAtStartPar
Creates a layer from its config.

\sphinxAtStartPar
This method is the reverse of \sphinxtitleref{get\_config},
capable of instantiating the same layer from the config
dictionary. It does not handle layer connectivity
(handled by Network), nor weights (handled by \sphinxtitleref{set\_weights}).
\begin{quote}\begin{description}
\sphinxlineitem{Parameters}
\sphinxAtStartPar
\sphinxstyleliteralstrong{\sphinxupquote{config}} \textendash{} A Python dictionary, typically the
output of get\_config.

\sphinxlineitem{Returns}
\sphinxAtStartPar
A layer instance.

\end{description}\end{quote}

\end{fulllineitems}

\index{get\_config() (beyondml.tflow.layers.MultiMaxPool2D.MultiMaxPool2D method)@\spxentry{get\_config()}\spxextra{beyondml.tflow.layers.MultiMaxPool2D.MultiMaxPool2D method}}

\begin{fulllineitems}
\phantomsection\label{\detokenize{beyondml.tflow.layers:beyondml.tflow.layers.MultiMaxPool2D.MultiMaxPool2D.get_config}}
\pysigstartsignatures
\pysiglinewithargsret{\sphinxbfcode{\sphinxupquote{get\_config}}}{}{}
\pysigstopsignatures
\sphinxAtStartPar
Returns the config of the layer.

\sphinxAtStartPar
A layer config is a Python dictionary (serializable)
containing the configuration of a layer.
The same layer can be reinstantiated later
(without its trained weights) from this configuration.

\sphinxAtStartPar
The config of a layer does not include connectivity
information, nor the layer class name. These are handled
by \sphinxtitleref{Network} (one layer of abstraction above).

\sphinxAtStartPar
Note that \sphinxtitleref{get\_config()} does not guarantee to return a fresh copy of
dict every time it is called. The callers should make a copy of the
returned dict if they want to modify it.
\begin{quote}\begin{description}
\sphinxlineitem{Returns}
\sphinxAtStartPar
Python dictionary.

\end{description}\end{quote}

\end{fulllineitems}


\end{fulllineitems}



\subparagraph{beyondml.tflow.layers.MultiMaxPool3D module}
\label{\detokenize{beyondml.tflow.layers:module-beyondml.tflow.layers.MultiMaxPool3D}}\label{\detokenize{beyondml.tflow.layers:beyondml-tflow-layers-multimaxpool3d-module}}\index{module@\spxentry{module}!beyondml.tflow.layers.MultiMaxPool3D@\spxentry{beyondml.tflow.layers.MultiMaxPool3D}}\index{beyondml.tflow.layers.MultiMaxPool3D@\spxentry{beyondml.tflow.layers.MultiMaxPool3D}!module@\spxentry{module}}\index{MultiMaxPool3D (class in beyondml.tflow.layers.MultiMaxPool3D)@\spxentry{MultiMaxPool3D}\spxextra{class in beyondml.tflow.layers.MultiMaxPool3D}}

\begin{fulllineitems}
\phantomsection\label{\detokenize{beyondml.tflow.layers:beyondml.tflow.layers.MultiMaxPool3D.MultiMaxPool3D}}
\pysigstartsignatures
\pysiglinewithargsret{\sphinxbfcode{\sphinxupquote{class\DUrole{w}{  }}}\sphinxcode{\sphinxupquote{beyondml.tflow.layers.MultiMaxPool3D.}}\sphinxbfcode{\sphinxupquote{MultiMaxPool3D}}}{\emph{\DUrole{o}{*}\DUrole{n}{args}}, \emph{\DUrole{o}{**}\DUrole{n}{kwargs}}}{}
\pysigstopsignatures
\sphinxAtStartPar
Bases: \sphinxcode{\sphinxupquote{Layer}}

\sphinxAtStartPar
Multitask 3D Max Pooling Layer. This layer implements the Max Pooling
algorithm across multiple inputs for developing multitask models
\index{call() (beyondml.tflow.layers.MultiMaxPool3D.MultiMaxPool3D method)@\spxentry{call()}\spxextra{beyondml.tflow.layers.MultiMaxPool3D.MultiMaxPool3D method}}

\begin{fulllineitems}
\phantomsection\label{\detokenize{beyondml.tflow.layers:beyondml.tflow.layers.MultiMaxPool3D.MultiMaxPool3D.call}}
\pysigstartsignatures
\pysiglinewithargsret{\sphinxbfcode{\sphinxupquote{call}}}{\emph{\DUrole{n}{inputs}}}{}
\pysigstopsignatures
\sphinxAtStartPar
This is where the layer’s logic lives and is called upon inputs
\begin{quote}\begin{description}
\sphinxlineitem{Parameters}
\sphinxAtStartPar
\sphinxstyleliteralstrong{\sphinxupquote{inputs}} (\sphinxstyleliteralemphasis{\sphinxupquote{TensorFlow Tensor}}\sphinxstyleliteralemphasis{\sphinxupquote{ or }}\sphinxstyleliteralemphasis{\sphinxupquote{Tensor\sphinxhyphen{}like}}) \textendash{} The inputs to the layer

\sphinxlineitem{Returns}
\sphinxAtStartPar
\sphinxstylestrong{outputs} \textendash{} The outputs of the layer’s logic

\sphinxlineitem{Return type}
\sphinxAtStartPar
TensorFlow Tensor

\end{description}\end{quote}

\end{fulllineitems}

\index{from\_config() (beyondml.tflow.layers.MultiMaxPool3D.MultiMaxPool3D class method)@\spxentry{from\_config()}\spxextra{beyondml.tflow.layers.MultiMaxPool3D.MultiMaxPool3D class method}}

\begin{fulllineitems}
\phantomsection\label{\detokenize{beyondml.tflow.layers:beyondml.tflow.layers.MultiMaxPool3D.MultiMaxPool3D.from_config}}
\pysigstartsignatures
\pysiglinewithargsret{\sphinxbfcode{\sphinxupquote{classmethod\DUrole{w}{  }}}\sphinxbfcode{\sphinxupquote{from\_config}}}{\emph{\DUrole{n}{config}}}{}
\pysigstopsignatures
\sphinxAtStartPar
Creates a layer from its config.

\sphinxAtStartPar
This method is the reverse of \sphinxtitleref{get\_config},
capable of instantiating the same layer from the config
dictionary. It does not handle layer connectivity
(handled by Network), nor weights (handled by \sphinxtitleref{set\_weights}).
\begin{quote}\begin{description}
\sphinxlineitem{Parameters}
\sphinxAtStartPar
\sphinxstyleliteralstrong{\sphinxupquote{config}} \textendash{} A Python dictionary, typically the
output of get\_config.

\sphinxlineitem{Returns}
\sphinxAtStartPar
A layer instance.

\end{description}\end{quote}

\end{fulllineitems}

\index{get\_config() (beyondml.tflow.layers.MultiMaxPool3D.MultiMaxPool3D method)@\spxentry{get\_config()}\spxextra{beyondml.tflow.layers.MultiMaxPool3D.MultiMaxPool3D method}}

\begin{fulllineitems}
\phantomsection\label{\detokenize{beyondml.tflow.layers:beyondml.tflow.layers.MultiMaxPool3D.MultiMaxPool3D.get_config}}
\pysigstartsignatures
\pysiglinewithargsret{\sphinxbfcode{\sphinxupquote{get\_config}}}{}{}
\pysigstopsignatures
\sphinxAtStartPar
Returns the config of the layer.

\sphinxAtStartPar
A layer config is a Python dictionary (serializable)
containing the configuration of a layer.
The same layer can be reinstantiated later
(without its trained weights) from this configuration.

\sphinxAtStartPar
The config of a layer does not include connectivity
information, nor the layer class name. These are handled
by \sphinxtitleref{Network} (one layer of abstraction above).

\sphinxAtStartPar
Note that \sphinxtitleref{get\_config()} does not guarantee to return a fresh copy of
dict every time it is called. The callers should make a copy of the
returned dict if they want to modify it.
\begin{quote}\begin{description}
\sphinxlineitem{Returns}
\sphinxAtStartPar
Python dictionary.

\end{description}\end{quote}

\end{fulllineitems}


\end{fulllineitems}



\subparagraph{beyondml.tflow.layers.MultitaskNormalization module}
\label{\detokenize{beyondml.tflow.layers:module-beyondml.tflow.layers.MultitaskNormalization}}\label{\detokenize{beyondml.tflow.layers:beyondml-tflow-layers-multitasknormalization-module}}\index{module@\spxentry{module}!beyondml.tflow.layers.MultitaskNormalization@\spxentry{beyondml.tflow.layers.MultitaskNormalization}}\index{beyondml.tflow.layers.MultitaskNormalization@\spxentry{beyondml.tflow.layers.MultitaskNormalization}!module@\spxentry{module}}\index{MultitaskNormalization (class in beyondml.tflow.layers.MultitaskNormalization)@\spxentry{MultitaskNormalization}\spxextra{class in beyondml.tflow.layers.MultitaskNormalization}}

\begin{fulllineitems}
\phantomsection\label{\detokenize{beyondml.tflow.layers:beyondml.tflow.layers.MultitaskNormalization.MultitaskNormalization}}
\pysigstartsignatures
\pysiglinewithargsret{\sphinxbfcode{\sphinxupquote{class\DUrole{w}{  }}}\sphinxcode{\sphinxupquote{beyondml.tflow.layers.MultitaskNormalization.}}\sphinxbfcode{\sphinxupquote{MultitaskNormalization}}}{\emph{\DUrole{o}{*}\DUrole{n}{args}}, \emph{\DUrole{o}{**}\DUrole{n}{kwargs}}}{}
\pysigstopsignatures
\sphinxAtStartPar
Bases: \sphinxcode{\sphinxupquote{Layer}}

\sphinxAtStartPar
Multitask layer which normalizes all inputs to sum to 1
\index{build() (beyondml.tflow.layers.MultitaskNormalization.MultitaskNormalization method)@\spxentry{build()}\spxextra{beyondml.tflow.layers.MultitaskNormalization.MultitaskNormalization method}}

\begin{fulllineitems}
\phantomsection\label{\detokenize{beyondml.tflow.layers:beyondml.tflow.layers.MultitaskNormalization.MultitaskNormalization.build}}
\pysigstartsignatures
\pysiglinewithargsret{\sphinxbfcode{\sphinxupquote{build}}}{\emph{\DUrole{n}{input\_shape}}}{}
\pysigstopsignatures
\sphinxAtStartPar
Creates the variables of the layer (optional, for subclass implementers).

\sphinxAtStartPar
This is a method that implementers of subclasses of \sphinxtitleref{Layer} or \sphinxtitleref{Model}
can override if they need a state\sphinxhyphen{}creation step in\sphinxhyphen{}between
layer instantiation and layer call. It is invoked automatically before
the first execution of \sphinxtitleref{call()}.

\sphinxAtStartPar
This is typically used to create the weights of \sphinxtitleref{Layer} subclasses
(at the discretion of the subclass implementer).
\begin{quote}\begin{description}
\sphinxlineitem{Parameters}
\sphinxAtStartPar
\sphinxstyleliteralstrong{\sphinxupquote{input\_shape}} \textendash{} Instance of \sphinxtitleref{TensorShape}, or list of instances of
\sphinxtitleref{TensorShape} if the layer expects a list of inputs
(one instance per input).

\end{description}\end{quote}

\end{fulllineitems}

\index{call() (beyondml.tflow.layers.MultitaskNormalization.MultitaskNormalization method)@\spxentry{call()}\spxextra{beyondml.tflow.layers.MultitaskNormalization.MultitaskNormalization method}}

\begin{fulllineitems}
\phantomsection\label{\detokenize{beyondml.tflow.layers:beyondml.tflow.layers.MultitaskNormalization.MultitaskNormalization.call}}
\pysigstartsignatures
\pysiglinewithargsret{\sphinxbfcode{\sphinxupquote{call}}}{\emph{\DUrole{n}{inputs}}}{}
\pysigstopsignatures
\sphinxAtStartPar
This is where the layer’s logic lives and is called upon inputs
\begin{quote}\begin{description}
\sphinxlineitem{Parameters}
\sphinxAtStartPar
\sphinxstyleliteralstrong{\sphinxupquote{inputs}} (\sphinxstyleliteralemphasis{\sphinxupquote{TensorFlow Tensor}}\sphinxstyleliteralemphasis{\sphinxupquote{ or }}\sphinxstyleliteralemphasis{\sphinxupquote{Tensor\sphinxhyphen{}like}}) \textendash{} The inputs to the layer

\sphinxlineitem{Returns}
\sphinxAtStartPar
\sphinxstylestrong{outputs} \textendash{} The outputs of the layer’s logic

\sphinxlineitem{Return type}
\sphinxAtStartPar
TensorFlow Tensor

\end{description}\end{quote}

\end{fulllineitems}

\index{from\_config() (beyondml.tflow.layers.MultitaskNormalization.MultitaskNormalization class method)@\spxentry{from\_config()}\spxextra{beyondml.tflow.layers.MultitaskNormalization.MultitaskNormalization class method}}

\begin{fulllineitems}
\phantomsection\label{\detokenize{beyondml.tflow.layers:beyondml.tflow.layers.MultitaskNormalization.MultitaskNormalization.from_config}}
\pysigstartsignatures
\pysiglinewithargsret{\sphinxbfcode{\sphinxupquote{classmethod\DUrole{w}{  }}}\sphinxbfcode{\sphinxupquote{from\_config}}}{\emph{\DUrole{n}{config}}}{}
\pysigstopsignatures
\sphinxAtStartPar
Creates a layer from its config.

\sphinxAtStartPar
This method is the reverse of \sphinxtitleref{get\_config},
capable of instantiating the same layer from the config
dictionary. It does not handle layer connectivity
(handled by Network), nor weights (handled by \sphinxtitleref{set\_weights}).
\begin{quote}\begin{description}
\sphinxlineitem{Parameters}
\sphinxAtStartPar
\sphinxstyleliteralstrong{\sphinxupquote{config}} \textendash{} A Python dictionary, typically the
output of get\_config.

\sphinxlineitem{Returns}
\sphinxAtStartPar
A layer instance.

\end{description}\end{quote}

\end{fulllineitems}

\index{get\_config() (beyondml.tflow.layers.MultitaskNormalization.MultitaskNormalization method)@\spxentry{get\_config()}\spxextra{beyondml.tflow.layers.MultitaskNormalization.MultitaskNormalization method}}

\begin{fulllineitems}
\phantomsection\label{\detokenize{beyondml.tflow.layers:beyondml.tflow.layers.MultitaskNormalization.MultitaskNormalization.get_config}}
\pysigstartsignatures
\pysiglinewithargsret{\sphinxbfcode{\sphinxupquote{get\_config}}}{}{}
\pysigstopsignatures
\sphinxAtStartPar
Returns the config of the layer.

\sphinxAtStartPar
A layer config is a Python dictionary (serializable)
containing the configuration of a layer.
The same layer can be reinstantiated later
(without its trained weights) from this configuration.

\sphinxAtStartPar
The config of a layer does not include connectivity
information, nor the layer class name. These are handled
by \sphinxtitleref{Network} (one layer of abstraction above).

\sphinxAtStartPar
Note that \sphinxtitleref{get\_config()} does not guarantee to return a fresh copy of
dict every time it is called. The callers should make a copy of the
returned dict if they want to modify it.
\begin{quote}\begin{description}
\sphinxlineitem{Returns}
\sphinxAtStartPar
Python dictionary.

\end{description}\end{quote}

\end{fulllineitems}


\end{fulllineitems}



\subparagraph{beyondml.tflow.layers.SelectorLayer module}
\label{\detokenize{beyondml.tflow.layers:module-beyondml.tflow.layers.SelectorLayer}}\label{\detokenize{beyondml.tflow.layers:beyondml-tflow-layers-selectorlayer-module}}\index{module@\spxentry{module}!beyondml.tflow.layers.SelectorLayer@\spxentry{beyondml.tflow.layers.SelectorLayer}}\index{beyondml.tflow.layers.SelectorLayer@\spxentry{beyondml.tflow.layers.SelectorLayer}!module@\spxentry{module}}\index{SelectorLayer (class in beyondml.tflow.layers.SelectorLayer)@\spxentry{SelectorLayer}\spxextra{class in beyondml.tflow.layers.SelectorLayer}}

\begin{fulllineitems}
\phantomsection\label{\detokenize{beyondml.tflow.layers:beyondml.tflow.layers.SelectorLayer.SelectorLayer}}
\pysigstartsignatures
\pysiglinewithargsret{\sphinxbfcode{\sphinxupquote{class\DUrole{w}{  }}}\sphinxcode{\sphinxupquote{beyondml.tflow.layers.SelectorLayer.}}\sphinxbfcode{\sphinxupquote{SelectorLayer}}}{\emph{\DUrole{o}{*}\DUrole{n}{args}}, \emph{\DUrole{o}{**}\DUrole{n}{kwargs}}}{}
\pysigstopsignatures
\sphinxAtStartPar
Bases: \sphinxcode{\sphinxupquote{Layer}}

\sphinxAtStartPar
Layer which selects individual inputs

\sphinxAtStartPar
Example:

\begin{sphinxVerbatim}[commandchars=\\\{\}]
\PYG{g+gp}{\PYGZgt{}\PYGZgt{}\PYGZgt{} }\PYG{c+c1}{\PYGZsh{} Create a model with two inputs and one SelectorLayer}
\PYG{g+gp}{\PYGZgt{}\PYGZgt{}\PYGZgt{} }\PYG{n}{input\PYGZus{}1} \PYG{o}{=} \PYG{n}{tf}\PYG{o}{.}\PYG{n}{keras}\PYG{o}{.}\PYG{n}{layers}\PYG{o}{.}\PYG{n}{Input}\PYG{p}{(}\PYG{l+m+mi}{10}\PYG{p}{)}
\PYG{g+gp}{\PYGZgt{}\PYGZgt{}\PYGZgt{} }\PYG{n}{input\PYGZus{}2} \PYG{o}{=} \PYG{n}{tf}\PYG{o}{.}\PYG{n}{keras}\PYG{o}{.}\PYG{n}{layers}\PYG{o}{.}\PYG{n}{Input}\PYG{p}{(}\PYG{l+m+mi}{10}\PYG{p}{)}
\PYG{g+gp}{\PYGZgt{}\PYGZgt{}\PYGZgt{} }\PYG{n}{selector} \PYG{o}{=} \PYG{n}{mann}\PYG{o}{.}\PYG{n}{layers}\PYG{o}{.}\PYG{n}{SelectorLayer}\PYG{p}{(}\PYG{l+m+mi}{1}\PYG{p}{)}\PYG{p}{(}\PYG{p}{[}\PYG{n}{input\PYGZus{}1}\PYG{p}{,} \PYG{n}{input\PYGZus{}2}\PYG{p}{]}\PYG{p}{)} \PYG{c+c1}{\PYGZsh{} 1 here indicates to select the second input and return it}
\PYG{g+gp}{\PYGZgt{}\PYGZgt{}\PYGZgt{} }\PYG{n}{model} \PYG{o}{=} \PYG{n}{tf}\PYG{o}{.}\PYG{n}{keras}\PYG{o}{.}\PYG{n}{models}\PYG{o}{.}\PYG{n}{Model}\PYG{p}{(}\PYG{p}{[}\PYG{n}{input\PYGZus{}1}\PYG{p}{,} \PYG{n}{input\PYGZus{}2}\PYG{p}{]}\PYG{p}{,} \PYG{n}{selector}\PYG{p}{)}
\PYG{g+gp}{\PYGZgt{}\PYGZgt{}\PYGZgt{} }\PYG{n}{model}\PYG{o}{.}\PYG{n}{compile}\PYG{p}{(}\PYG{p}{)}
\PYG{g+gp}{\PYGZgt{}\PYGZgt{}\PYGZgt{} }\PYG{c+c1}{\PYGZsh{} Call the model}
\PYG{g+gp}{\PYGZgt{}\PYGZgt{}\PYGZgt{} }\PYG{n}{data1} \PYG{o}{=} \PYG{n}{np}\PYG{o}{.}\PYG{n}{arange}\PYG{p}{(}\PYG{l+m+mi}{10}\PYG{p}{)}\PYG{o}{.}\PYG{n}{reshape}\PYG{p}{(}\PYG{p}{(}\PYG{l+m+mi}{1}\PYG{p}{,} \PYG{l+m+mi}{10}\PYG{p}{)}\PYG{p}{)}
\PYG{g+gp}{\PYGZgt{}\PYGZgt{}\PYGZgt{} }\PYG{n}{data2} \PYG{o}{=} \PYG{l+m+mi}{2}\PYG{o}{*}\PYG{n}{np}\PYG{o}{.}\PYG{n}{arange}\PYG{p}{(}\PYG{l+m+mi}{10}\PYG{p}{)}\PYG{o}{.}\PYG{n}{reshape}\PYG{p}{(}\PYG{p}{(}\PYG{l+m+mi}{1}\PYG{p}{,} \PYG{l+m+mi}{10}\PYG{p}{)}\PYG{p}{)}
\PYG{g+gp}{\PYGZgt{}\PYGZgt{}\PYGZgt{} }\PYG{n}{model}\PYG{o}{.}\PYG{n}{predict}\PYG{p}{(}\PYG{p}{[}\PYG{n}{data1}\PYG{p}{,} \PYG{n}{data2}\PYG{p}{]}\PYG{p}{)}
\PYG{g+go}{array([[ 0.,  2.,  4.,  6.,  8., 10., 12., 14., 16., 18.]], dtype=float32)}
\end{sphinxVerbatim}
\index{call() (beyondml.tflow.layers.SelectorLayer.SelectorLayer method)@\spxentry{call()}\spxextra{beyondml.tflow.layers.SelectorLayer.SelectorLayer method}}

\begin{fulllineitems}
\phantomsection\label{\detokenize{beyondml.tflow.layers:beyondml.tflow.layers.SelectorLayer.SelectorLayer.call}}
\pysigstartsignatures
\pysiglinewithargsret{\sphinxbfcode{\sphinxupquote{call}}}{\emph{\DUrole{n}{inputs}}}{}
\pysigstopsignatures
\sphinxAtStartPar
This is where the layer’s logic lives and is called upon inputs
\begin{quote}\begin{description}
\sphinxlineitem{Parameters}
\sphinxAtStartPar
\sphinxstyleliteralstrong{\sphinxupquote{inputs}} (\sphinxstyleliteralemphasis{\sphinxupquote{TensorFlow Tensor}}\sphinxstyleliteralemphasis{\sphinxupquote{ or }}\sphinxstyleliteralemphasis{\sphinxupquote{Tensor\sphinxhyphen{}like}}) \textendash{} The inputs to the layer

\sphinxlineitem{Returns}
\sphinxAtStartPar
\sphinxstylestrong{outputs} \textendash{} The outputs of the layer’s logic

\sphinxlineitem{Return type}
\sphinxAtStartPar
TensorFlow Tensor

\end{description}\end{quote}

\end{fulllineitems}

\index{from\_config() (beyondml.tflow.layers.SelectorLayer.SelectorLayer class method)@\spxentry{from\_config()}\spxextra{beyondml.tflow.layers.SelectorLayer.SelectorLayer class method}}

\begin{fulllineitems}
\phantomsection\label{\detokenize{beyondml.tflow.layers:beyondml.tflow.layers.SelectorLayer.SelectorLayer.from_config}}
\pysigstartsignatures
\pysiglinewithargsret{\sphinxbfcode{\sphinxupquote{classmethod\DUrole{w}{  }}}\sphinxbfcode{\sphinxupquote{from\_config}}}{\emph{\DUrole{n}{config}}}{}
\pysigstopsignatures
\sphinxAtStartPar
Creates a layer from its config.

\sphinxAtStartPar
This method is the reverse of \sphinxtitleref{get\_config},
capable of instantiating the same layer from the config
dictionary. It does not handle layer connectivity
(handled by Network), nor weights (handled by \sphinxtitleref{set\_weights}).
\begin{quote}\begin{description}
\sphinxlineitem{Parameters}
\sphinxAtStartPar
\sphinxstyleliteralstrong{\sphinxupquote{config}} \textendash{} A Python dictionary, typically the
output of get\_config.

\sphinxlineitem{Returns}
\sphinxAtStartPar
A layer instance.

\end{description}\end{quote}

\end{fulllineitems}

\index{get\_config() (beyondml.tflow.layers.SelectorLayer.SelectorLayer method)@\spxentry{get\_config()}\spxextra{beyondml.tflow.layers.SelectorLayer.SelectorLayer method}}

\begin{fulllineitems}
\phantomsection\label{\detokenize{beyondml.tflow.layers:beyondml.tflow.layers.SelectorLayer.SelectorLayer.get_config}}
\pysigstartsignatures
\pysiglinewithargsret{\sphinxbfcode{\sphinxupquote{get\_config}}}{}{}
\pysigstopsignatures
\sphinxAtStartPar
Returns the config of the layer.

\sphinxAtStartPar
A layer config is a Python dictionary (serializable)
containing the configuration of a layer.
The same layer can be reinstantiated later
(without its trained weights) from this configuration.

\sphinxAtStartPar
The config of a layer does not include connectivity
information, nor the layer class name. These are handled
by \sphinxtitleref{Network} (one layer of abstraction above).

\sphinxAtStartPar
Note that \sphinxtitleref{get\_config()} does not guarantee to return a fresh copy of
dict every time it is called. The callers should make a copy of the
returned dict if they want to modify it.
\begin{quote}\begin{description}
\sphinxlineitem{Returns}
\sphinxAtStartPar
Python dictionary.

\end{description}\end{quote}

\end{fulllineitems}

\index{sel\_index (beyondml.tflow.layers.SelectorLayer.SelectorLayer property)@\spxentry{sel\_index}\spxextra{beyondml.tflow.layers.SelectorLayer.SelectorLayer property}}

\begin{fulllineitems}
\phantomsection\label{\detokenize{beyondml.tflow.layers:beyondml.tflow.layers.SelectorLayer.SelectorLayer.sel_index}}
\pysigstartsignatures
\pysigline{\sphinxbfcode{\sphinxupquote{property\DUrole{w}{  }}}\sphinxbfcode{\sphinxupquote{sel\_index}}}
\pysigstopsignatures
\end{fulllineitems}


\end{fulllineitems}



\subparagraph{beyondml.tflow.layers.SparseConv2D module}
\label{\detokenize{beyondml.tflow.layers:module-beyondml.tflow.layers.SparseConv2D}}\label{\detokenize{beyondml.tflow.layers:beyondml-tflow-layers-sparseconv2d-module}}\index{module@\spxentry{module}!beyondml.tflow.layers.SparseConv2D@\spxentry{beyondml.tflow.layers.SparseConv2D}}\index{beyondml.tflow.layers.SparseConv2D@\spxentry{beyondml.tflow.layers.SparseConv2D}!module@\spxentry{module}}\index{SparseConv2D (class in beyondml.tflow.layers.SparseConv2D)@\spxentry{SparseConv2D}\spxextra{class in beyondml.tflow.layers.SparseConv2D}}

\begin{fulllineitems}
\phantomsection\label{\detokenize{beyondml.tflow.layers:beyondml.tflow.layers.SparseConv2D.SparseConv2D}}
\pysigstartsignatures
\pysiglinewithargsret{\sphinxbfcode{\sphinxupquote{class\DUrole{w}{  }}}\sphinxcode{\sphinxupquote{beyondml.tflow.layers.SparseConv2D.}}\sphinxbfcode{\sphinxupquote{SparseConv2D}}}{\emph{\DUrole{o}{*}\DUrole{n}{args}}, \emph{\DUrole{o}{**}\DUrole{n}{kwargs}}}{}
\pysigstopsignatures
\sphinxAtStartPar
Bases: \sphinxcode{\sphinxupquote{Layer}}

\sphinxAtStartPar
Sparse implementation of the Convolutional layer. If used in a model,
must be saved and loaded via pickle
\index{build() (beyondml.tflow.layers.SparseConv2D.SparseConv2D method)@\spxentry{build()}\spxextra{beyondml.tflow.layers.SparseConv2D.SparseConv2D method}}

\begin{fulllineitems}
\phantomsection\label{\detokenize{beyondml.tflow.layers:beyondml.tflow.layers.SparseConv2D.SparseConv2D.build}}
\pysigstartsignatures
\pysiglinewithargsret{\sphinxbfcode{\sphinxupquote{build}}}{\emph{\DUrole{n}{input\_shape}}}{}
\pysigstopsignatures
\sphinxAtStartPar
Build the layer in preparation to be trained or called. Should not be called directly,
but rather is called when the layer is added to a model

\end{fulllineitems}

\index{call() (beyondml.tflow.layers.SparseConv2D.SparseConv2D method)@\spxentry{call()}\spxextra{beyondml.tflow.layers.SparseConv2D.SparseConv2D method}}

\begin{fulllineitems}
\phantomsection\label{\detokenize{beyondml.tflow.layers:beyondml.tflow.layers.SparseConv2D.SparseConv2D.call}}
\pysigstartsignatures
\pysiglinewithargsret{\sphinxbfcode{\sphinxupquote{call}}}{\emph{\DUrole{n}{inputs}}}{}
\pysigstopsignatures
\sphinxAtStartPar
This is where the layer’s logic lives and is called upon inputs
\begin{quote}\begin{description}
\sphinxlineitem{Parameters}
\sphinxAtStartPar
\sphinxstyleliteralstrong{\sphinxupquote{inputs}} (\sphinxstyleliteralemphasis{\sphinxupquote{TensorFlow Tensor}}\sphinxstyleliteralemphasis{\sphinxupquote{ or }}\sphinxstyleliteralemphasis{\sphinxupquote{Tensor\sphinxhyphen{}like}}) \textendash{} The inputs to the layer

\sphinxlineitem{Returns}
\sphinxAtStartPar
\sphinxstylestrong{outputs} \textendash{} The outputs of the layer’s logic

\sphinxlineitem{Return type}
\sphinxAtStartPar
TensorFlow Tensor

\end{description}\end{quote}

\end{fulllineitems}

\index{from\_config() (beyondml.tflow.layers.SparseConv2D.SparseConv2D class method)@\spxentry{from\_config()}\spxextra{beyondml.tflow.layers.SparseConv2D.SparseConv2D class method}}

\begin{fulllineitems}
\phantomsection\label{\detokenize{beyondml.tflow.layers:beyondml.tflow.layers.SparseConv2D.SparseConv2D.from_config}}
\pysigstartsignatures
\pysiglinewithargsret{\sphinxbfcode{\sphinxupquote{classmethod\DUrole{w}{  }}}\sphinxbfcode{\sphinxupquote{from\_config}}}{\emph{\DUrole{n}{config}}}{}
\pysigstopsignatures
\sphinxAtStartPar
Creates a layer from its config.

\sphinxAtStartPar
This method is the reverse of \sphinxtitleref{get\_config},
capable of instantiating the same layer from the config
dictionary. It does not handle layer connectivity
(handled by Network), nor weights (handled by \sphinxtitleref{set\_weights}).
\begin{quote}\begin{description}
\sphinxlineitem{Parameters}
\sphinxAtStartPar
\sphinxstyleliteralstrong{\sphinxupquote{config}} \textendash{} A Python dictionary, typically the
output of get\_config.

\sphinxlineitem{Returns}
\sphinxAtStartPar
A layer instance.

\end{description}\end{quote}

\end{fulllineitems}

\index{from\_layer() (beyondml.tflow.layers.SparseConv2D.SparseConv2D class method)@\spxentry{from\_layer()}\spxextra{beyondml.tflow.layers.SparseConv2D.SparseConv2D class method}}

\begin{fulllineitems}
\phantomsection\label{\detokenize{beyondml.tflow.layers:beyondml.tflow.layers.SparseConv2D.SparseConv2D.from_layer}}
\pysigstartsignatures
\pysiglinewithargsret{\sphinxbfcode{\sphinxupquote{classmethod\DUrole{w}{  }}}\sphinxbfcode{\sphinxupquote{from\_layer}}}{\emph{\DUrole{n}{layer}}}{}
\pysigstopsignatures
\sphinxAtStartPar
Create a layer from an instance of another layer

\end{fulllineitems}

\index{get\_config() (beyondml.tflow.layers.SparseConv2D.SparseConv2D method)@\spxentry{get\_config()}\spxextra{beyondml.tflow.layers.SparseConv2D.SparseConv2D method}}

\begin{fulllineitems}
\phantomsection\label{\detokenize{beyondml.tflow.layers:beyondml.tflow.layers.SparseConv2D.SparseConv2D.get_config}}
\pysigstartsignatures
\pysiglinewithargsret{\sphinxbfcode{\sphinxupquote{get\_config}}}{}{}
\pysigstopsignatures
\sphinxAtStartPar
Returns the config of the layer.

\sphinxAtStartPar
A layer config is a Python dictionary (serializable)
containing the configuration of a layer.
The same layer can be reinstantiated later
(without its trained weights) from this configuration.

\sphinxAtStartPar
The config of a layer does not include connectivity
information, nor the layer class name. These are handled
by \sphinxtitleref{Network} (one layer of abstraction above).

\sphinxAtStartPar
Note that \sphinxtitleref{get\_config()} does not guarantee to return a fresh copy of
dict every time it is called. The callers should make a copy of the
returned dict if they want to modify it.
\begin{quote}\begin{description}
\sphinxlineitem{Returns}
\sphinxAtStartPar
Python dictionary.

\end{description}\end{quote}

\end{fulllineitems}


\end{fulllineitems}



\subparagraph{beyondml.tflow.layers.SparseConv3D module}
\label{\detokenize{beyondml.tflow.layers:module-beyondml.tflow.layers.SparseConv3D}}\label{\detokenize{beyondml.tflow.layers:beyondml-tflow-layers-sparseconv3d-module}}\index{module@\spxentry{module}!beyondml.tflow.layers.SparseConv3D@\spxentry{beyondml.tflow.layers.SparseConv3D}}\index{beyondml.tflow.layers.SparseConv3D@\spxentry{beyondml.tflow.layers.SparseConv3D}!module@\spxentry{module}}\index{SparseConv3D (class in beyondml.tflow.layers.SparseConv3D)@\spxentry{SparseConv3D}\spxextra{class in beyondml.tflow.layers.SparseConv3D}}

\begin{fulllineitems}
\phantomsection\label{\detokenize{beyondml.tflow.layers:beyondml.tflow.layers.SparseConv3D.SparseConv3D}}
\pysigstartsignatures
\pysiglinewithargsret{\sphinxbfcode{\sphinxupquote{class\DUrole{w}{  }}}\sphinxcode{\sphinxupquote{beyondml.tflow.layers.SparseConv3D.}}\sphinxbfcode{\sphinxupquote{SparseConv3D}}}{\emph{\DUrole{o}{*}\DUrole{n}{args}}, \emph{\DUrole{o}{**}\DUrole{n}{kwargs}}}{}
\pysigstopsignatures
\sphinxAtStartPar
Bases: \sphinxcode{\sphinxupquote{Layer}}

\sphinxAtStartPar
Sparse implementation of the Convolutional layer. If used in a model,
must be saved and loaded via pickle
\index{build() (beyondml.tflow.layers.SparseConv3D.SparseConv3D method)@\spxentry{build()}\spxextra{beyondml.tflow.layers.SparseConv3D.SparseConv3D method}}

\begin{fulllineitems}
\phantomsection\label{\detokenize{beyondml.tflow.layers:beyondml.tflow.layers.SparseConv3D.SparseConv3D.build}}
\pysigstartsignatures
\pysiglinewithargsret{\sphinxbfcode{\sphinxupquote{build}}}{\emph{\DUrole{n}{input\_shape}}}{}
\pysigstopsignatures
\sphinxAtStartPar
Build the layer in preparation to be trained or called. Should not be called directly,
but rather is called when the layer is added to a model

\end{fulllineitems}

\index{call() (beyondml.tflow.layers.SparseConv3D.SparseConv3D method)@\spxentry{call()}\spxextra{beyondml.tflow.layers.SparseConv3D.SparseConv3D method}}

\begin{fulllineitems}
\phantomsection\label{\detokenize{beyondml.tflow.layers:beyondml.tflow.layers.SparseConv3D.SparseConv3D.call}}
\pysigstartsignatures
\pysiglinewithargsret{\sphinxbfcode{\sphinxupquote{call}}}{\emph{\DUrole{n}{inputs}}}{}
\pysigstopsignatures
\sphinxAtStartPar
This is where the layer’s logic lives and is called upon inputs
\begin{quote}\begin{description}
\sphinxlineitem{Parameters}
\sphinxAtStartPar
\sphinxstyleliteralstrong{\sphinxupquote{inputs}} (\sphinxstyleliteralemphasis{\sphinxupquote{TensorFlow Tensor}}\sphinxstyleliteralemphasis{\sphinxupquote{ or }}\sphinxstyleliteralemphasis{\sphinxupquote{Tensor\sphinxhyphen{}like}}) \textendash{} The inputs to the layer

\sphinxlineitem{Returns}
\sphinxAtStartPar
\sphinxstylestrong{outputs} \textendash{} The outputs of the layer’s logic

\sphinxlineitem{Return type}
\sphinxAtStartPar
TensorFlow Tensor

\end{description}\end{quote}

\end{fulllineitems}

\index{from\_config() (beyondml.tflow.layers.SparseConv3D.SparseConv3D class method)@\spxentry{from\_config()}\spxextra{beyondml.tflow.layers.SparseConv3D.SparseConv3D class method}}

\begin{fulllineitems}
\phantomsection\label{\detokenize{beyondml.tflow.layers:beyondml.tflow.layers.SparseConv3D.SparseConv3D.from_config}}
\pysigstartsignatures
\pysiglinewithargsret{\sphinxbfcode{\sphinxupquote{classmethod\DUrole{w}{  }}}\sphinxbfcode{\sphinxupquote{from\_config}}}{\emph{\DUrole{n}{config}}}{}
\pysigstopsignatures
\sphinxAtStartPar
Creates a layer from its config.

\sphinxAtStartPar
This method is the reverse of \sphinxtitleref{get\_config},
capable of instantiating the same layer from the config
dictionary. It does not handle layer connectivity
(handled by Network), nor weights (handled by \sphinxtitleref{set\_weights}).
\begin{quote}\begin{description}
\sphinxlineitem{Parameters}
\sphinxAtStartPar
\sphinxstyleliteralstrong{\sphinxupquote{config}} \textendash{} A Python dictionary, typically the
output of get\_config.

\sphinxlineitem{Returns}
\sphinxAtStartPar
A layer instance.

\end{description}\end{quote}

\end{fulllineitems}

\index{from\_layer() (beyondml.tflow.layers.SparseConv3D.SparseConv3D class method)@\spxentry{from\_layer()}\spxextra{beyondml.tflow.layers.SparseConv3D.SparseConv3D class method}}

\begin{fulllineitems}
\phantomsection\label{\detokenize{beyondml.tflow.layers:beyondml.tflow.layers.SparseConv3D.SparseConv3D.from_layer}}
\pysigstartsignatures
\pysiglinewithargsret{\sphinxbfcode{\sphinxupquote{classmethod\DUrole{w}{  }}}\sphinxbfcode{\sphinxupquote{from\_layer}}}{\emph{\DUrole{n}{layer}}}{}
\pysigstopsignatures
\sphinxAtStartPar
Create a layer from an instance of another layer

\end{fulllineitems}

\index{get\_config() (beyondml.tflow.layers.SparseConv3D.SparseConv3D method)@\spxentry{get\_config()}\spxextra{beyondml.tflow.layers.SparseConv3D.SparseConv3D method}}

\begin{fulllineitems}
\phantomsection\label{\detokenize{beyondml.tflow.layers:beyondml.tflow.layers.SparseConv3D.SparseConv3D.get_config}}
\pysigstartsignatures
\pysiglinewithargsret{\sphinxbfcode{\sphinxupquote{get\_config}}}{}{}
\pysigstopsignatures
\sphinxAtStartPar
Returns the config of the layer.

\sphinxAtStartPar
A layer config is a Python dictionary (serializable)
containing the configuration of a layer.
The same layer can be reinstantiated later
(without its trained weights) from this configuration.

\sphinxAtStartPar
The config of a layer does not include connectivity
information, nor the layer class name. These are handled
by \sphinxtitleref{Network} (one layer of abstraction above).

\sphinxAtStartPar
Note that \sphinxtitleref{get\_config()} does not guarantee to return a fresh copy of
dict every time it is called. The callers should make a copy of the
returned dict if they want to modify it.
\begin{quote}\begin{description}
\sphinxlineitem{Returns}
\sphinxAtStartPar
Python dictionary.

\end{description}\end{quote}

\end{fulllineitems}


\end{fulllineitems}



\subparagraph{beyondml.tflow.layers.SparseDense module}
\label{\detokenize{beyondml.tflow.layers:module-beyondml.tflow.layers.SparseDense}}\label{\detokenize{beyondml.tflow.layers:beyondml-tflow-layers-sparsedense-module}}\index{module@\spxentry{module}!beyondml.tflow.layers.SparseDense@\spxentry{beyondml.tflow.layers.SparseDense}}\index{beyondml.tflow.layers.SparseDense@\spxentry{beyondml.tflow.layers.SparseDense}!module@\spxentry{module}}\index{SparseDense (class in beyondml.tflow.layers.SparseDense)@\spxentry{SparseDense}\spxextra{class in beyondml.tflow.layers.SparseDense}}

\begin{fulllineitems}
\phantomsection\label{\detokenize{beyondml.tflow.layers:beyondml.tflow.layers.SparseDense.SparseDense}}
\pysigstartsignatures
\pysiglinewithargsret{\sphinxbfcode{\sphinxupquote{class\DUrole{w}{  }}}\sphinxcode{\sphinxupquote{beyondml.tflow.layers.SparseDense.}}\sphinxbfcode{\sphinxupquote{SparseDense}}}{\emph{\DUrole{o}{*}\DUrole{n}{args}}, \emph{\DUrole{o}{**}\DUrole{n}{kwargs}}}{}
\pysigstopsignatures
\sphinxAtStartPar
Bases: \sphinxcode{\sphinxupquote{Layer}}

\sphinxAtStartPar
Sparse implementation of the Dense layer. If used in a model, must be saved and loaded via pickle
\index{build() (beyondml.tflow.layers.SparseDense.SparseDense method)@\spxentry{build()}\spxextra{beyondml.tflow.layers.SparseDense.SparseDense method}}

\begin{fulllineitems}
\phantomsection\label{\detokenize{beyondml.tflow.layers:beyondml.tflow.layers.SparseDense.SparseDense.build}}
\pysigstartsignatures
\pysiglinewithargsret{\sphinxbfcode{\sphinxupquote{build}}}{\emph{\DUrole{n}{input\_shape}}}{}
\pysigstopsignatures
\sphinxAtStartPar
Build the layer in preparation to be trained or called. Should not be called directly,
but rather is called when the layer is added to a model

\end{fulllineitems}

\index{call() (beyondml.tflow.layers.SparseDense.SparseDense method)@\spxentry{call()}\spxextra{beyondml.tflow.layers.SparseDense.SparseDense method}}

\begin{fulllineitems}
\phantomsection\label{\detokenize{beyondml.tflow.layers:beyondml.tflow.layers.SparseDense.SparseDense.call}}
\pysigstartsignatures
\pysiglinewithargsret{\sphinxbfcode{\sphinxupquote{call}}}{\emph{\DUrole{n}{inputs}}}{}
\pysigstopsignatures
\sphinxAtStartPar
This is where the layer’s logic lives and is called upon inputs
\begin{quote}\begin{description}
\sphinxlineitem{Parameters}
\sphinxAtStartPar
\sphinxstyleliteralstrong{\sphinxupquote{inputs}} (\sphinxstyleliteralemphasis{\sphinxupquote{TensorFlow Tensor}}\sphinxstyleliteralemphasis{\sphinxupquote{ or }}\sphinxstyleliteralemphasis{\sphinxupquote{Tensor\sphinxhyphen{}like}}) \textendash{} The inputs to the layer

\sphinxlineitem{Returns}
\sphinxAtStartPar
\sphinxstylestrong{outputs} \textendash{} The outputs of the layer’s logic

\sphinxlineitem{Return type}
\sphinxAtStartPar
TensorFlow Tensor

\end{description}\end{quote}

\end{fulllineitems}

\index{from\_config() (beyondml.tflow.layers.SparseDense.SparseDense class method)@\spxentry{from\_config()}\spxextra{beyondml.tflow.layers.SparseDense.SparseDense class method}}

\begin{fulllineitems}
\phantomsection\label{\detokenize{beyondml.tflow.layers:beyondml.tflow.layers.SparseDense.SparseDense.from_config}}
\pysigstartsignatures
\pysiglinewithargsret{\sphinxbfcode{\sphinxupquote{classmethod\DUrole{w}{  }}}\sphinxbfcode{\sphinxupquote{from\_config}}}{\emph{\DUrole{n}{config}}}{}
\pysigstopsignatures
\sphinxAtStartPar
Creates a layer from its config.

\sphinxAtStartPar
This method is the reverse of \sphinxtitleref{get\_config},
capable of instantiating the same layer from the config
dictionary. It does not handle layer connectivity
(handled by Network), nor weights (handled by \sphinxtitleref{set\_weights}).
\begin{quote}\begin{description}
\sphinxlineitem{Parameters}
\sphinxAtStartPar
\sphinxstyleliteralstrong{\sphinxupquote{config}} \textendash{} A Python dictionary, typically the
output of get\_config.

\sphinxlineitem{Returns}
\sphinxAtStartPar
A layer instance.

\end{description}\end{quote}

\end{fulllineitems}

\index{from\_layer() (beyondml.tflow.layers.SparseDense.SparseDense class method)@\spxentry{from\_layer()}\spxextra{beyondml.tflow.layers.SparseDense.SparseDense class method}}

\begin{fulllineitems}
\phantomsection\label{\detokenize{beyondml.tflow.layers:beyondml.tflow.layers.SparseDense.SparseDense.from_layer}}
\pysigstartsignatures
\pysiglinewithargsret{\sphinxbfcode{\sphinxupquote{classmethod\DUrole{w}{  }}}\sphinxbfcode{\sphinxupquote{from\_layer}}}{\emph{\DUrole{n}{layer}}}{}
\pysigstopsignatures
\sphinxAtStartPar
Create a layer from an instance of another layer

\end{fulllineitems}

\index{get\_config() (beyondml.tflow.layers.SparseDense.SparseDense method)@\spxentry{get\_config()}\spxextra{beyondml.tflow.layers.SparseDense.SparseDense method}}

\begin{fulllineitems}
\phantomsection\label{\detokenize{beyondml.tflow.layers:beyondml.tflow.layers.SparseDense.SparseDense.get_config}}
\pysigstartsignatures
\pysiglinewithargsret{\sphinxbfcode{\sphinxupquote{get\_config}}}{}{}
\pysigstopsignatures
\sphinxAtStartPar
Returns the config of the layer.

\sphinxAtStartPar
A layer config is a Python dictionary (serializable)
containing the configuration of a layer.
The same layer can be reinstantiated later
(without its trained weights) from this configuration.

\sphinxAtStartPar
The config of a layer does not include connectivity
information, nor the layer class name. These are handled
by \sphinxtitleref{Network} (one layer of abstraction above).

\sphinxAtStartPar
Note that \sphinxtitleref{get\_config()} does not guarantee to return a fresh copy of
dict every time it is called. The callers should make a copy of the
returned dict if they want to modify it.
\begin{quote}\begin{description}
\sphinxlineitem{Returns}
\sphinxAtStartPar
Python dictionary.

\end{description}\end{quote}

\end{fulllineitems}


\end{fulllineitems}



\subparagraph{beyondml.tflow.layers.SparseMultiConv2D module}
\label{\detokenize{beyondml.tflow.layers:module-beyondml.tflow.layers.SparseMultiConv2D}}\label{\detokenize{beyondml.tflow.layers:beyondml-tflow-layers-sparsemulticonv2d-module}}\index{module@\spxentry{module}!beyondml.tflow.layers.SparseMultiConv2D@\spxentry{beyondml.tflow.layers.SparseMultiConv2D}}\index{beyondml.tflow.layers.SparseMultiConv2D@\spxentry{beyondml.tflow.layers.SparseMultiConv2D}!module@\spxentry{module}}\index{SparseMultiConv2D (class in beyondml.tflow.layers.SparseMultiConv2D)@\spxentry{SparseMultiConv2D}\spxextra{class in beyondml.tflow.layers.SparseMultiConv2D}}

\begin{fulllineitems}
\phantomsection\label{\detokenize{beyondml.tflow.layers:beyondml.tflow.layers.SparseMultiConv2D.SparseMultiConv2D}}
\pysigstartsignatures
\pysiglinewithargsret{\sphinxbfcode{\sphinxupquote{class\DUrole{w}{  }}}\sphinxcode{\sphinxupquote{beyondml.tflow.layers.SparseMultiConv2D.}}\sphinxbfcode{\sphinxupquote{SparseMultiConv2D}}}{\emph{\DUrole{o}{*}\DUrole{n}{args}}, \emph{\DUrole{o}{**}\DUrole{n}{kwargs}}}{}
\pysigstopsignatures
\sphinxAtStartPar
Bases: \sphinxcode{\sphinxupquote{Layer}}

\sphinxAtStartPar
Sparse implementation of the MultiConv layer. If used in a model, must be saved and loaded via pickle
\index{build() (beyondml.tflow.layers.SparseMultiConv2D.SparseMultiConv2D method)@\spxentry{build()}\spxextra{beyondml.tflow.layers.SparseMultiConv2D.SparseMultiConv2D method}}

\begin{fulllineitems}
\phantomsection\label{\detokenize{beyondml.tflow.layers:beyondml.tflow.layers.SparseMultiConv2D.SparseMultiConv2D.build}}
\pysigstartsignatures
\pysiglinewithargsret{\sphinxbfcode{\sphinxupquote{build}}}{\emph{\DUrole{n}{input\_shapes}}}{}
\pysigstopsignatures
\sphinxAtStartPar
Build the layer in preparation to be trained or called. Should not be called directly,
but rather is called when the layer is added to a model

\end{fulllineitems}

\index{call() (beyondml.tflow.layers.SparseMultiConv2D.SparseMultiConv2D method)@\spxentry{call()}\spxextra{beyondml.tflow.layers.SparseMultiConv2D.SparseMultiConv2D method}}

\begin{fulllineitems}
\phantomsection\label{\detokenize{beyondml.tflow.layers:beyondml.tflow.layers.SparseMultiConv2D.SparseMultiConv2D.call}}
\pysigstartsignatures
\pysiglinewithargsret{\sphinxbfcode{\sphinxupquote{call}}}{\emph{\DUrole{n}{inputs}}}{}
\pysigstopsignatures
\sphinxAtStartPar
This is where the layer’s logic lives and is called upon inputs
\begin{quote}\begin{description}
\sphinxlineitem{Parameters}
\sphinxAtStartPar
\sphinxstyleliteralstrong{\sphinxupquote{inputs}} (\sphinxstyleliteralemphasis{\sphinxupquote{TensorFlow Tensor}}\sphinxstyleliteralemphasis{\sphinxupquote{ or }}\sphinxstyleliteralemphasis{\sphinxupquote{Tensor\sphinxhyphen{}like}}) \textendash{} The inputs to the layer

\sphinxlineitem{Returns}
\sphinxAtStartPar
\sphinxstylestrong{outputs} \textendash{} The outputs of the layer’s logic

\sphinxlineitem{Return type}
\sphinxAtStartPar
TensorFlow Tensor

\end{description}\end{quote}

\end{fulllineitems}

\index{from\_config() (beyondml.tflow.layers.SparseMultiConv2D.SparseMultiConv2D class method)@\spxentry{from\_config()}\spxextra{beyondml.tflow.layers.SparseMultiConv2D.SparseMultiConv2D class method}}

\begin{fulllineitems}
\phantomsection\label{\detokenize{beyondml.tflow.layers:beyondml.tflow.layers.SparseMultiConv2D.SparseMultiConv2D.from_config}}
\pysigstartsignatures
\pysiglinewithargsret{\sphinxbfcode{\sphinxupquote{classmethod\DUrole{w}{  }}}\sphinxbfcode{\sphinxupquote{from\_config}}}{\emph{\DUrole{n}{config}}}{}
\pysigstopsignatures
\sphinxAtStartPar
Creates a layer from its config.

\sphinxAtStartPar
This method is the reverse of \sphinxtitleref{get\_config},
capable of instantiating the same layer from the config
dictionary. It does not handle layer connectivity
(handled by Network), nor weights (handled by \sphinxtitleref{set\_weights}).
\begin{quote}\begin{description}
\sphinxlineitem{Parameters}
\sphinxAtStartPar
\sphinxstyleliteralstrong{\sphinxupquote{config}} \textendash{} A Python dictionary, typically the
output of get\_config.

\sphinxlineitem{Returns}
\sphinxAtStartPar
A layer instance.

\end{description}\end{quote}

\end{fulllineitems}

\index{from\_layer() (beyondml.tflow.layers.SparseMultiConv2D.SparseMultiConv2D class method)@\spxentry{from\_layer()}\spxextra{beyondml.tflow.layers.SparseMultiConv2D.SparseMultiConv2D class method}}

\begin{fulllineitems}
\phantomsection\label{\detokenize{beyondml.tflow.layers:beyondml.tflow.layers.SparseMultiConv2D.SparseMultiConv2D.from_layer}}
\pysigstartsignatures
\pysiglinewithargsret{\sphinxbfcode{\sphinxupquote{classmethod\DUrole{w}{  }}}\sphinxbfcode{\sphinxupquote{from\_layer}}}{\emph{\DUrole{n}{layer}}}{}
\pysigstopsignatures
\sphinxAtStartPar
Create a layer from an instance of another layer

\end{fulllineitems}

\index{get\_config() (beyondml.tflow.layers.SparseMultiConv2D.SparseMultiConv2D method)@\spxentry{get\_config()}\spxextra{beyondml.tflow.layers.SparseMultiConv2D.SparseMultiConv2D method}}

\begin{fulllineitems}
\phantomsection\label{\detokenize{beyondml.tflow.layers:beyondml.tflow.layers.SparseMultiConv2D.SparseMultiConv2D.get_config}}
\pysigstartsignatures
\pysiglinewithargsret{\sphinxbfcode{\sphinxupquote{get\_config}}}{}{}
\pysigstopsignatures
\sphinxAtStartPar
Returns the config of the layer.

\sphinxAtStartPar
A layer config is a Python dictionary (serializable)
containing the configuration of a layer.
The same layer can be reinstantiated later
(without its trained weights) from this configuration.

\sphinxAtStartPar
The config of a layer does not include connectivity
information, nor the layer class name. These are handled
by \sphinxtitleref{Network} (one layer of abstraction above).

\sphinxAtStartPar
Note that \sphinxtitleref{get\_config()} does not guarantee to return a fresh copy of
dict every time it is called. The callers should make a copy of the
returned dict if they want to modify it.
\begin{quote}\begin{description}
\sphinxlineitem{Returns}
\sphinxAtStartPar
Python dictionary.

\end{description}\end{quote}

\end{fulllineitems}


\end{fulllineitems}



\subparagraph{beyondml.tflow.layers.SparseMultiConv3D module}
\label{\detokenize{beyondml.tflow.layers:module-beyondml.tflow.layers.SparseMultiConv3D}}\label{\detokenize{beyondml.tflow.layers:beyondml-tflow-layers-sparsemulticonv3d-module}}\index{module@\spxentry{module}!beyondml.tflow.layers.SparseMultiConv3D@\spxentry{beyondml.tflow.layers.SparseMultiConv3D}}\index{beyondml.tflow.layers.SparseMultiConv3D@\spxentry{beyondml.tflow.layers.SparseMultiConv3D}!module@\spxentry{module}}\index{SparseMultiConv3D (class in beyondml.tflow.layers.SparseMultiConv3D)@\spxentry{SparseMultiConv3D}\spxextra{class in beyondml.tflow.layers.SparseMultiConv3D}}

\begin{fulllineitems}
\phantomsection\label{\detokenize{beyondml.tflow.layers:beyondml.tflow.layers.SparseMultiConv3D.SparseMultiConv3D}}
\pysigstartsignatures
\pysiglinewithargsret{\sphinxbfcode{\sphinxupquote{class\DUrole{w}{  }}}\sphinxcode{\sphinxupquote{beyondml.tflow.layers.SparseMultiConv3D.}}\sphinxbfcode{\sphinxupquote{SparseMultiConv3D}}}{\emph{\DUrole{o}{*}\DUrole{n}{args}}, \emph{\DUrole{o}{**}\DUrole{n}{kwargs}}}{}
\pysigstopsignatures
\sphinxAtStartPar
Bases: \sphinxcode{\sphinxupquote{Layer}}

\sphinxAtStartPar
Sparse implementation of the MultiConv layer. If used in a model, must be saved and loaded via pickle
\index{build() (beyondml.tflow.layers.SparseMultiConv3D.SparseMultiConv3D method)@\spxentry{build()}\spxextra{beyondml.tflow.layers.SparseMultiConv3D.SparseMultiConv3D method}}

\begin{fulllineitems}
\phantomsection\label{\detokenize{beyondml.tflow.layers:beyondml.tflow.layers.SparseMultiConv3D.SparseMultiConv3D.build}}
\pysigstartsignatures
\pysiglinewithargsret{\sphinxbfcode{\sphinxupquote{build}}}{\emph{\DUrole{n}{input\_shape}}}{}
\pysigstopsignatures
\sphinxAtStartPar
Build the layer in preparation to be trained or called. Should not be called directly,
but rather is called when the layer is added to a model

\end{fulllineitems}

\index{call() (beyondml.tflow.layers.SparseMultiConv3D.SparseMultiConv3D method)@\spxentry{call()}\spxextra{beyondml.tflow.layers.SparseMultiConv3D.SparseMultiConv3D method}}

\begin{fulllineitems}
\phantomsection\label{\detokenize{beyondml.tflow.layers:beyondml.tflow.layers.SparseMultiConv3D.SparseMultiConv3D.call}}
\pysigstartsignatures
\pysiglinewithargsret{\sphinxbfcode{\sphinxupquote{call}}}{\emph{\DUrole{n}{inputs}}}{}
\pysigstopsignatures
\sphinxAtStartPar
This is where the layer’s logic lives and is called upon inputs
\begin{quote}\begin{description}
\sphinxlineitem{Parameters}
\sphinxAtStartPar
\sphinxstyleliteralstrong{\sphinxupquote{inputs}} (\sphinxstyleliteralemphasis{\sphinxupquote{TensorFlow Tensor}}\sphinxstyleliteralemphasis{\sphinxupquote{ or }}\sphinxstyleliteralemphasis{\sphinxupquote{Tensor\sphinxhyphen{}like}}) \textendash{} The inputs to the layer

\sphinxlineitem{Returns}
\sphinxAtStartPar
\sphinxstylestrong{outputs} \textendash{} The outputs of the layer’s logic

\sphinxlineitem{Return type}
\sphinxAtStartPar
TensorFlow Tensor

\end{description}\end{quote}

\end{fulllineitems}

\index{from\_config() (beyondml.tflow.layers.SparseMultiConv3D.SparseMultiConv3D class method)@\spxentry{from\_config()}\spxextra{beyondml.tflow.layers.SparseMultiConv3D.SparseMultiConv3D class method}}

\begin{fulllineitems}
\phantomsection\label{\detokenize{beyondml.tflow.layers:beyondml.tflow.layers.SparseMultiConv3D.SparseMultiConv3D.from_config}}
\pysigstartsignatures
\pysiglinewithargsret{\sphinxbfcode{\sphinxupquote{classmethod\DUrole{w}{  }}}\sphinxbfcode{\sphinxupquote{from\_config}}}{\emph{\DUrole{n}{config}}}{}
\pysigstopsignatures
\sphinxAtStartPar
Creates a layer from its config.

\sphinxAtStartPar
This method is the reverse of \sphinxtitleref{get\_config},
capable of instantiating the same layer from the config
dictionary. It does not handle layer connectivity
(handled by Network), nor weights (handled by \sphinxtitleref{set\_weights}).
\begin{quote}\begin{description}
\sphinxlineitem{Parameters}
\sphinxAtStartPar
\sphinxstyleliteralstrong{\sphinxupquote{config}} \textendash{} A Python dictionary, typically the
output of get\_config.

\sphinxlineitem{Returns}
\sphinxAtStartPar
A layer instance.

\end{description}\end{quote}

\end{fulllineitems}

\index{from\_layer() (beyondml.tflow.layers.SparseMultiConv3D.SparseMultiConv3D class method)@\spxentry{from\_layer()}\spxextra{beyondml.tflow.layers.SparseMultiConv3D.SparseMultiConv3D class method}}

\begin{fulllineitems}
\phantomsection\label{\detokenize{beyondml.tflow.layers:beyondml.tflow.layers.SparseMultiConv3D.SparseMultiConv3D.from_layer}}
\pysigstartsignatures
\pysiglinewithargsret{\sphinxbfcode{\sphinxupquote{classmethod\DUrole{w}{  }}}\sphinxbfcode{\sphinxupquote{from\_layer}}}{\emph{\DUrole{n}{layer}}}{}
\pysigstopsignatures
\sphinxAtStartPar
Create a layer from an instance of another layer

\end{fulllineitems}

\index{get\_config() (beyondml.tflow.layers.SparseMultiConv3D.SparseMultiConv3D method)@\spxentry{get\_config()}\spxextra{beyondml.tflow.layers.SparseMultiConv3D.SparseMultiConv3D method}}

\begin{fulllineitems}
\phantomsection\label{\detokenize{beyondml.tflow.layers:beyondml.tflow.layers.SparseMultiConv3D.SparseMultiConv3D.get_config}}
\pysigstartsignatures
\pysiglinewithargsret{\sphinxbfcode{\sphinxupquote{get\_config}}}{}{}
\pysigstopsignatures
\sphinxAtStartPar
Returns the config of the layer.

\sphinxAtStartPar
A layer config is a Python dictionary (serializable)
containing the configuration of a layer.
The same layer can be reinstantiated later
(without its trained weights) from this configuration.

\sphinxAtStartPar
The config of a layer does not include connectivity
information, nor the layer class name. These are handled
by \sphinxtitleref{Network} (one layer of abstraction above).

\sphinxAtStartPar
Note that \sphinxtitleref{get\_config()} does not guarantee to return a fresh copy of
dict every time it is called. The callers should make a copy of the
returned dict if they want to modify it.
\begin{quote}\begin{description}
\sphinxlineitem{Returns}
\sphinxAtStartPar
Python dictionary.

\end{description}\end{quote}

\end{fulllineitems}


\end{fulllineitems}



\subparagraph{beyondml.tflow.layers.SparseMultiDense module}
\label{\detokenize{beyondml.tflow.layers:module-beyondml.tflow.layers.SparseMultiDense}}\label{\detokenize{beyondml.tflow.layers:beyondml-tflow-layers-sparsemultidense-module}}\index{module@\spxentry{module}!beyondml.tflow.layers.SparseMultiDense@\spxentry{beyondml.tflow.layers.SparseMultiDense}}\index{beyondml.tflow.layers.SparseMultiDense@\spxentry{beyondml.tflow.layers.SparseMultiDense}!module@\spxentry{module}}\index{SparseMultiDense (class in beyondml.tflow.layers.SparseMultiDense)@\spxentry{SparseMultiDense}\spxextra{class in beyondml.tflow.layers.SparseMultiDense}}

\begin{fulllineitems}
\phantomsection\label{\detokenize{beyondml.tflow.layers:beyondml.tflow.layers.SparseMultiDense.SparseMultiDense}}
\pysigstartsignatures
\pysiglinewithargsret{\sphinxbfcode{\sphinxupquote{class\DUrole{w}{  }}}\sphinxcode{\sphinxupquote{beyondml.tflow.layers.SparseMultiDense.}}\sphinxbfcode{\sphinxupquote{SparseMultiDense}}}{\emph{\DUrole{o}{*}\DUrole{n}{args}}, \emph{\DUrole{o}{**}\DUrole{n}{kwargs}}}{}
\pysigstopsignatures
\sphinxAtStartPar
Bases: \sphinxcode{\sphinxupquote{Layer}}

\sphinxAtStartPar
Sparse implementation of the MultiDense layer. If used in a model, must be saved and loaded via pickle
\index{build() (beyondml.tflow.layers.SparseMultiDense.SparseMultiDense method)@\spxentry{build()}\spxextra{beyondml.tflow.layers.SparseMultiDense.SparseMultiDense method}}

\begin{fulllineitems}
\phantomsection\label{\detokenize{beyondml.tflow.layers:beyondml.tflow.layers.SparseMultiDense.SparseMultiDense.build}}
\pysigstartsignatures
\pysiglinewithargsret{\sphinxbfcode{\sphinxupquote{build}}}{\emph{\DUrole{n}{input\_shape}}}{}
\pysigstopsignatures
\sphinxAtStartPar
Build the layer in preparation to be trained or called. Should not be called directly,
but rather is called when the layer is added to a model

\end{fulllineitems}

\index{call() (beyondml.tflow.layers.SparseMultiDense.SparseMultiDense method)@\spxentry{call()}\spxextra{beyondml.tflow.layers.SparseMultiDense.SparseMultiDense method}}

\begin{fulllineitems}
\phantomsection\label{\detokenize{beyondml.tflow.layers:beyondml.tflow.layers.SparseMultiDense.SparseMultiDense.call}}
\pysigstartsignatures
\pysiglinewithargsret{\sphinxbfcode{\sphinxupquote{call}}}{\emph{\DUrole{n}{inputs}}}{}
\pysigstopsignatures
\sphinxAtStartPar
This is where the layer’s logic lives and is called upon inputs
\begin{quote}\begin{description}
\sphinxlineitem{Parameters}
\sphinxAtStartPar
\sphinxstyleliteralstrong{\sphinxupquote{inputs}} (\sphinxstyleliteralemphasis{\sphinxupquote{TensorFlow Tensor}}\sphinxstyleliteralemphasis{\sphinxupquote{ or }}\sphinxstyleliteralemphasis{\sphinxupquote{Tensor\sphinxhyphen{}like}}) \textendash{} The inputs to the layer

\sphinxlineitem{Returns}
\sphinxAtStartPar
\sphinxstylestrong{outputs} \textendash{} The outputs of the layer’s logic

\sphinxlineitem{Return type}
\sphinxAtStartPar
TensorFlow Tensor

\end{description}\end{quote}

\end{fulllineitems}

\index{from\_config() (beyondml.tflow.layers.SparseMultiDense.SparseMultiDense class method)@\spxentry{from\_config()}\spxextra{beyondml.tflow.layers.SparseMultiDense.SparseMultiDense class method}}

\begin{fulllineitems}
\phantomsection\label{\detokenize{beyondml.tflow.layers:beyondml.tflow.layers.SparseMultiDense.SparseMultiDense.from_config}}
\pysigstartsignatures
\pysiglinewithargsret{\sphinxbfcode{\sphinxupquote{classmethod\DUrole{w}{  }}}\sphinxbfcode{\sphinxupquote{from\_config}}}{\emph{\DUrole{n}{config}}}{}
\pysigstopsignatures
\sphinxAtStartPar
Creates a layer from its config.

\sphinxAtStartPar
This method is the reverse of \sphinxtitleref{get\_config},
capable of instantiating the same layer from the config
dictionary. It does not handle layer connectivity
(handled by Network), nor weights (handled by \sphinxtitleref{set\_weights}).
\begin{quote}\begin{description}
\sphinxlineitem{Parameters}
\sphinxAtStartPar
\sphinxstyleliteralstrong{\sphinxupquote{config}} \textendash{} A Python dictionary, typically the
output of get\_config.

\sphinxlineitem{Returns}
\sphinxAtStartPar
A layer instance.

\end{description}\end{quote}

\end{fulllineitems}

\index{from\_layer() (beyondml.tflow.layers.SparseMultiDense.SparseMultiDense class method)@\spxentry{from\_layer()}\spxextra{beyondml.tflow.layers.SparseMultiDense.SparseMultiDense class method}}

\begin{fulllineitems}
\phantomsection\label{\detokenize{beyondml.tflow.layers:beyondml.tflow.layers.SparseMultiDense.SparseMultiDense.from_layer}}
\pysigstartsignatures
\pysiglinewithargsret{\sphinxbfcode{\sphinxupquote{classmethod\DUrole{w}{  }}}\sphinxbfcode{\sphinxupquote{from\_layer}}}{\emph{\DUrole{n}{layer}}}{}
\pysigstopsignatures
\sphinxAtStartPar
Create a layer from an instance of another layer

\end{fulllineitems}

\index{get\_config() (beyondml.tflow.layers.SparseMultiDense.SparseMultiDense method)@\spxentry{get\_config()}\spxextra{beyondml.tflow.layers.SparseMultiDense.SparseMultiDense method}}

\begin{fulllineitems}
\phantomsection\label{\detokenize{beyondml.tflow.layers:beyondml.tflow.layers.SparseMultiDense.SparseMultiDense.get_config}}
\pysigstartsignatures
\pysiglinewithargsret{\sphinxbfcode{\sphinxupquote{get\_config}}}{}{}
\pysigstopsignatures
\sphinxAtStartPar
Returns the config of the layer.

\sphinxAtStartPar
A layer config is a Python dictionary (serializable)
containing the configuration of a layer.
The same layer can be reinstantiated later
(without its trained weights) from this configuration.

\sphinxAtStartPar
The config of a layer does not include connectivity
information, nor the layer class name. These are handled
by \sphinxtitleref{Network} (one layer of abstraction above).

\sphinxAtStartPar
Note that \sphinxtitleref{get\_config()} does not guarantee to return a fresh copy of
dict every time it is called. The callers should make a copy of the
returned dict if they want to modify it.
\begin{quote}\begin{description}
\sphinxlineitem{Returns}
\sphinxAtStartPar
Python dictionary.

\end{description}\end{quote}

\end{fulllineitems}


\end{fulllineitems}



\subparagraph{beyondml.tflow.layers.SumLayer module}
\label{\detokenize{beyondml.tflow.layers:module-beyondml.tflow.layers.SumLayer}}\label{\detokenize{beyondml.tflow.layers:beyondml-tflow-layers-sumlayer-module}}\index{module@\spxentry{module}!beyondml.tflow.layers.SumLayer@\spxentry{beyondml.tflow.layers.SumLayer}}\index{beyondml.tflow.layers.SumLayer@\spxentry{beyondml.tflow.layers.SumLayer}!module@\spxentry{module}}\index{SumLayer (class in beyondml.tflow.layers.SumLayer)@\spxentry{SumLayer}\spxextra{class in beyondml.tflow.layers.SumLayer}}

\begin{fulllineitems}
\phantomsection\label{\detokenize{beyondml.tflow.layers:beyondml.tflow.layers.SumLayer.SumLayer}}
\pysigstartsignatures
\pysiglinewithargsret{\sphinxbfcode{\sphinxupquote{class\DUrole{w}{  }}}\sphinxcode{\sphinxupquote{beyondml.tflow.layers.SumLayer.}}\sphinxbfcode{\sphinxupquote{SumLayer}}}{\emph{\DUrole{o}{*}\DUrole{n}{args}}, \emph{\DUrole{o}{**}\DUrole{n}{kwargs}}}{}
\pysigstopsignatures
\sphinxAtStartPar
Bases: \sphinxcode{\sphinxupquote{Layer}}

\sphinxAtStartPar
Layer which adds all inputs together. All inputs must have compatible shapes

\sphinxAtStartPar
Example:

\begin{sphinxVerbatim}[commandchars=\\\{\}]
\PYG{g+gp}{\PYGZgt{}\PYGZgt{}\PYGZgt{} }\PYG{c+c1}{\PYGZsh{} Create a model with just a SumLayer and two inputs}
\PYG{g+gp}{\PYGZgt{}\PYGZgt{}\PYGZgt{} }\PYG{n}{input\PYGZus{}1} \PYG{o}{=} \PYG{n}{tf}\PYG{o}{.}\PYG{n}{keras}\PYG{o}{.}\PYG{n}{layers}\PYG{o}{.}\PYG{n}{Input}\PYG{p}{(}\PYG{l+m+mi}{10}\PYG{p}{)}
\PYG{g+gp}{\PYGZgt{}\PYGZgt{}\PYGZgt{} }\PYG{n}{input\PYGZus{}2} \PYG{o}{=} \PYG{n}{tf}\PYG{o}{.}\PYG{n}{keras}\PYG{o}{.}\PYG{n}{layers}\PYG{o}{.}\PYG{n}{Input}\PYG{p}{(}\PYG{l+m+mi}{10}\PYG{p}{)}
\PYG{g+gp}{\PYGZgt{}\PYGZgt{}\PYGZgt{} }\PYG{n}{sum\PYGZus{}layer} \PYG{o}{=} \PYG{n}{mann}\PYG{o}{.}\PYG{n}{layers}\PYG{o}{.}\PYG{n}{SumLayer}\PYG{p}{(}\PYG{p}{)}\PYG{p}{(}\PYG{p}{[}\PYG{n}{input\PYGZus{}1}\PYG{p}{,} \PYG{n}{input\PYGZus{}2}\PYG{p}{]}\PYG{p}{)}
\PYG{g+gp}{\PYGZgt{}\PYGZgt{}\PYGZgt{} }\PYG{n}{model} \PYG{o}{=} \PYG{n}{tf}\PYG{o}{.}\PYG{n}{keras}\PYG{o}{.}\PYG{n}{models}\PYG{o}{.}\PYG{n}{Model}\PYG{p}{(}\PYG{p}{[}\PYG{n}{input\PYGZus{}1}\PYG{p}{,} \PYG{n}{input\PYGZus{}2}\PYG{p}{]}\PYG{p}{,} \PYG{n}{sum\PYGZus{}layer}\PYG{p}{)}
\PYG{g+gp}{\PYGZgt{}\PYGZgt{}\PYGZgt{} }\PYG{n}{model}\PYG{o}{.}\PYG{n}{compile}\PYG{p}{(}\PYG{p}{)}
\PYG{g+gp}{\PYGZgt{}\PYGZgt{}\PYGZgt{} }\PYG{c+c1}{\PYGZsh{} Call the model}
\PYG{g+gp}{\PYGZgt{}\PYGZgt{}\PYGZgt{} }\PYG{n}{data} \PYG{o}{=} \PYG{n}{np}\PYG{o}{.}\PYG{n}{arange}\PYG{p}{(}\PYG{l+m+mi}{10}\PYG{p}{)}\PYG{o}{.}\PYG{n}{reshape}\PYG{p}{(}\PYG{p}{(}\PYG{l+m+mi}{1}\PYG{p}{,} \PYG{l+m+mi}{10}\PYG{p}{)}\PYG{p}{)}
\PYG{g+gp}{\PYGZgt{}\PYGZgt{}\PYGZgt{} }\PYG{n}{model}\PYG{o}{.}\PYG{n}{predict}\PYG{p}{(}\PYG{p}{[}\PYG{n}{data}\PYG{p}{,} \PYG{n}{data}\PYG{p}{]}\PYG{p}{)}
\PYG{g+go}{array([[ 0.,  2.,  4.,  6.,  8., 10., 12., 14., 16., 18.]], dtype=float32)}
\end{sphinxVerbatim}
\index{call() (beyondml.tflow.layers.SumLayer.SumLayer method)@\spxentry{call()}\spxextra{beyondml.tflow.layers.SumLayer.SumLayer method}}

\begin{fulllineitems}
\phantomsection\label{\detokenize{beyondml.tflow.layers:beyondml.tflow.layers.SumLayer.SumLayer.call}}
\pysigstartsignatures
\pysiglinewithargsret{\sphinxbfcode{\sphinxupquote{call}}}{\emph{\DUrole{n}{inputs}}}{}
\pysigstopsignatures
\sphinxAtStartPar
This is where the layer’s logic lives and is called upon inputs
\begin{quote}\begin{description}
\sphinxlineitem{Parameters}
\sphinxAtStartPar
\sphinxstyleliteralstrong{\sphinxupquote{inputs}} (\sphinxstyleliteralemphasis{\sphinxupquote{TensorFlow Tensor}}\sphinxstyleliteralemphasis{\sphinxupquote{ or }}\sphinxstyleliteralemphasis{\sphinxupquote{Tensor\sphinxhyphen{}like}}) \textendash{} The inputs to the layer

\sphinxlineitem{Returns}
\sphinxAtStartPar
\sphinxstylestrong{outputs} \textendash{} The outputs of the layer’s logic

\sphinxlineitem{Return type}
\sphinxAtStartPar
TensorFlow Tensor

\end{description}\end{quote}

\end{fulllineitems}

\index{from\_config() (beyondml.tflow.layers.SumLayer.SumLayer class method)@\spxentry{from\_config()}\spxextra{beyondml.tflow.layers.SumLayer.SumLayer class method}}

\begin{fulllineitems}
\phantomsection\label{\detokenize{beyondml.tflow.layers:beyondml.tflow.layers.SumLayer.SumLayer.from_config}}
\pysigstartsignatures
\pysiglinewithargsret{\sphinxbfcode{\sphinxupquote{classmethod\DUrole{w}{  }}}\sphinxbfcode{\sphinxupquote{from\_config}}}{\emph{\DUrole{n}{config}}}{}
\pysigstopsignatures
\sphinxAtStartPar
Creates a layer from its config.

\sphinxAtStartPar
This method is the reverse of \sphinxtitleref{get\_config},
capable of instantiating the same layer from the config
dictionary. It does not handle layer connectivity
(handled by Network), nor weights (handled by \sphinxtitleref{set\_weights}).
\begin{quote}\begin{description}
\sphinxlineitem{Parameters}
\sphinxAtStartPar
\sphinxstyleliteralstrong{\sphinxupquote{config}} \textendash{} A Python dictionary, typically the
output of get\_config.

\sphinxlineitem{Returns}
\sphinxAtStartPar
A layer instance.

\end{description}\end{quote}

\end{fulllineitems}

\index{get\_config() (beyondml.tflow.layers.SumLayer.SumLayer method)@\spxentry{get\_config()}\spxextra{beyondml.tflow.layers.SumLayer.SumLayer method}}

\begin{fulllineitems}
\phantomsection\label{\detokenize{beyondml.tflow.layers:beyondml.tflow.layers.SumLayer.SumLayer.get_config}}
\pysigstartsignatures
\pysiglinewithargsret{\sphinxbfcode{\sphinxupquote{get\_config}}}{}{}
\pysigstopsignatures
\sphinxAtStartPar
Returns the config of the layer.

\sphinxAtStartPar
A layer config is a Python dictionary (serializable)
containing the configuration of a layer.
The same layer can be reinstantiated later
(without its trained weights) from this configuration.

\sphinxAtStartPar
The config of a layer does not include connectivity
information, nor the layer class name. These are handled
by \sphinxtitleref{Network} (one layer of abstraction above).

\sphinxAtStartPar
Note that \sphinxtitleref{get\_config()} does not guarantee to return a fresh copy of
dict every time it is called. The callers should make a copy of the
returned dict if they want to modify it.
\begin{quote}\begin{description}
\sphinxlineitem{Returns}
\sphinxAtStartPar
Python dictionary.

\end{description}\end{quote}

\end{fulllineitems}


\end{fulllineitems}



\subparagraph{Module contents}
\label{\detokenize{beyondml.tflow.layers:module-beyondml.tflow.layers}}\label{\detokenize{beyondml.tflow.layers:module-contents}}\index{module@\spxentry{module}!beyondml.tflow.layers@\spxentry{beyondml.tflow.layers}}\index{beyondml.tflow.layers@\spxentry{beyondml.tflow.layers}!module@\spxentry{module}}
\sphinxAtStartPar
Custom layers to use when building MANN models

\sphinxstepscope


\subparagraph{beyondml.tflow.utils package}
\label{\detokenize{beyondml.tflow.utils:beyondml-tflow-utils-package}}\label{\detokenize{beyondml.tflow.utils::doc}}

\subparagraph{Submodules}
\label{\detokenize{beyondml.tflow.utils:submodules}}

\subparagraph{beyondml.tflow.utils.transformer module}
\label{\detokenize{beyondml.tflow.utils:module-beyondml.tflow.utils.transformer}}\label{\detokenize{beyondml.tflow.utils:beyondml-tflow-utils-transformer-module}}\index{module@\spxentry{module}!beyondml.tflow.utils.transformer@\spxentry{beyondml.tflow.utils.transformer}}\index{beyondml.tflow.utils.transformer@\spxentry{beyondml.tflow.utils.transformer}!module@\spxentry{module}}\index{build\_token\_position\_embedding\_block() (in module beyondml.tflow.utils.transformer)@\spxentry{build\_token\_position\_embedding\_block()}\spxextra{in module beyondml.tflow.utils.transformer}}

\begin{fulllineitems}
\phantomsection\label{\detokenize{beyondml.tflow.utils:beyondml.tflow.utils.transformer.build_token_position_embedding_block}}
\pysigstartsignatures
\pysiglinewithargsret{\sphinxcode{\sphinxupquote{beyondml.tflow.utils.transformer.}}\sphinxbfcode{\sphinxupquote{build\_token\_position\_embedding\_block}}}{\emph{\DUrole{n}{sequence\_length}}, \emph{\DUrole{n}{vocab\_size}}, \emph{\DUrole{n}{embed\_dim}}}{}
\pysigstopsignatures
\sphinxAtStartPar
Builds a token and position embedding block
\begin{quote}\begin{description}
\sphinxlineitem{Parameters}\begin{itemize}
\item {} 
\sphinxAtStartPar
\sphinxstyleliteralstrong{\sphinxupquote{sequence\_length}} (\sphinxstyleliteralemphasis{\sphinxupquote{int}}) \textendash{} The length of each sequence

\item {} 
\sphinxAtStartPar
\sphinxstyleliteralstrong{\sphinxupquote{vocab\_size}} (\sphinxstyleliteralemphasis{\sphinxupquote{int}}) \textendash{} The size of the vocabulary used

\item {} 
\sphinxAtStartPar
\sphinxstyleliteralstrong{\sphinxupquote{embed\_dim}} (\sphinxstyleliteralemphasis{\sphinxupquote{int}}) \textendash{} The desired embedding dimension

\end{itemize}

\sphinxlineitem{Returns}
\sphinxAtStartPar
\sphinxstylestrong{embedding\_block} \textendash{} The embedding block, which can be used alone or
as a layer in another model

\sphinxlineitem{Return type}
\sphinxAtStartPar
TensorFlow keras Functional model

\end{description}\end{quote}

\end{fulllineitems}

\index{build\_transformer\_block() (in module beyondml.tflow.utils.transformer)@\spxentry{build\_transformer\_block()}\spxextra{in module beyondml.tflow.utils.transformer}}

\begin{fulllineitems}
\phantomsection\label{\detokenize{beyondml.tflow.utils:beyondml.tflow.utils.transformer.build_transformer_block}}
\pysigstartsignatures
\pysiglinewithargsret{\sphinxcode{\sphinxupquote{beyondml.tflow.utils.transformer.}}\sphinxbfcode{\sphinxupquote{build\_transformer\_block}}}{\emph{\DUrole{n}{input\_shape}}, \emph{\DUrole{n}{embed\_dim}}, \emph{\DUrole{n}{num\_heads}}, \emph{\DUrole{n}{neurons}}, \emph{\DUrole{n}{dropout\_rate}\DUrole{o}{=}\DUrole{default_value}{0.1}}}{}
\pysigstopsignatures
\sphinxAtStartPar
Build a Transformer Block
\begin{quote}\begin{description}
\sphinxlineitem{Parameters}\begin{itemize}
\item {} 
\sphinxAtStartPar
\sphinxstyleliteralstrong{\sphinxupquote{input\_shape}} (\sphinxstyleliteralemphasis{\sphinxupquote{int}}\sphinxstyleliteralemphasis{\sphinxupquote{ or }}\sphinxstyleliteralemphasis{\sphinxupquote{tuple}}\sphinxstyleliteralemphasis{\sphinxupquote{ of }}\sphinxstyleliteralemphasis{\sphinxupquote{int}}) \textendash{} The input shape for the model to use

\item {} 
\sphinxAtStartPar
\sphinxstyleliteralstrong{\sphinxupquote{embed\_dim}} (\sphinxstyleliteralemphasis{\sphinxupquote{int}}) \textendash{} The dimension of the embedding

\item {} 
\sphinxAtStartPar
\sphinxstyleliteralstrong{\sphinxupquote{num\_heads}} (\sphinxstyleliteralemphasis{\sphinxupquote{int}}) \textendash{} The number of attention heads to use

\item {} 
\sphinxAtStartPar
\sphinxstyleliteralstrong{\sphinxupquote{neurons}} (\sphinxstyleliteralemphasis{\sphinxupquote{int}}) \textendash{} The number of hidden neurons to use in the hidden layer

\item {} 
\sphinxAtStartPar
\sphinxstyleliteralstrong{\sphinxupquote{dropout\_rate}} (\sphinxstyleliteralemphasis{\sphinxupquote{float}}\sphinxstyleliteralemphasis{\sphinxupquote{ (}}\sphinxstyleliteralemphasis{\sphinxupquote{default 0.1}}\sphinxstyleliteralemphasis{\sphinxupquote{)}}) \textendash{} Rate at which dropout is applied

\item {} 
\sphinxAtStartPar
\sphinxstyleliteralstrong{\sphinxupquote{value\_dim}} (\sphinxstyleliteralemphasis{\sphinxupquote{int}}\sphinxstyleliteralemphasis{\sphinxupquote{ or }}\sphinxstyleliteralemphasis{\sphinxupquote{None}}\sphinxstyleliteralemphasis{\sphinxupquote{ (}}\sphinxstyleliteralemphasis{\sphinxupquote{default None}}\sphinxstyleliteralemphasis{\sphinxupquote{)}}) \textendash{} The dimension to use for the \sphinxtitleref{value} matrix, if provided

\end{itemize}

\sphinxlineitem{Returns}
\sphinxAtStartPar
\sphinxstylestrong{transformer\_block} \textendash{} The transformer block, which can then be used alone or as
a layer in another model

\sphinxlineitem{Return type}
\sphinxAtStartPar
TensorFlow keras Functional model

\end{description}\end{quote}

\end{fulllineitems}



\subparagraph{beyondml.tflow.utils.utils module}
\label{\detokenize{beyondml.tflow.utils:module-beyondml.tflow.utils.utils}}\label{\detokenize{beyondml.tflow.utils:beyondml-tflow-utils-utils-module}}\index{module@\spxentry{module}!beyondml.tflow.utils.utils@\spxentry{beyondml.tflow.utils.utils}}\index{beyondml.tflow.utils.utils@\spxentry{beyondml.tflow.utils.utils}!module@\spxentry{module}}\index{ActiveSparsification (class in beyondml.tflow.utils.utils)@\spxentry{ActiveSparsification}\spxextra{class in beyondml.tflow.utils.utils}}

\begin{fulllineitems}
\phantomsection\label{\detokenize{beyondml.tflow.utils:beyondml.tflow.utils.utils.ActiveSparsification}}
\pysigstartsignatures
\pysiglinewithargsret{\sphinxbfcode{\sphinxupquote{class\DUrole{w}{  }}}\sphinxcode{\sphinxupquote{beyondml.tflow.utils.utils.}}\sphinxbfcode{\sphinxupquote{ActiveSparsification}}}{\emph{\DUrole{n}{performance\_cutoff}}, \emph{\DUrole{n}{performance\_measure}\DUrole{o}{=}\DUrole{default_value}{\textquotesingle{}auto\textquotesingle{}}}, \emph{\DUrole{n}{starting\_sparsification}\DUrole{o}{=}\DUrole{default_value}{None}}, \emph{\DUrole{n}{max\_sparsification}\DUrole{o}{=}\DUrole{default_value}{99}}, \emph{\DUrole{n}{sparsification\_rate}\DUrole{o}{=}\DUrole{default_value}{1}}, \emph{\DUrole{n}{sparsification\_patience}\DUrole{o}{=}\DUrole{default_value}{10}}, \emph{\DUrole{n}{stopping\_delta}\DUrole{o}{=}\DUrole{default_value}{0.01}}, \emph{\DUrole{n}{stopping\_patience}\DUrole{o}{=}\DUrole{default_value}{5}}, \emph{\DUrole{n}{restore\_best\_weights}\DUrole{o}{=}\DUrole{default_value}{True}}, \emph{\DUrole{n}{verbose}\DUrole{o}{=}\DUrole{default_value}{1}}}{}
\pysigstopsignatures
\sphinxAtStartPar
Bases: \sphinxcode{\sphinxupquote{Callback}}

\sphinxAtStartPar
Keras\sphinxhyphen{}compatible callback object which enables active sparsification, allowing for increased sparsification as models
train.
\index{on\_epoch\_end() (beyondml.tflow.utils.utils.ActiveSparsification method)@\spxentry{on\_epoch\_end()}\spxextra{beyondml.tflow.utils.utils.ActiveSparsification method}}

\begin{fulllineitems}
\phantomsection\label{\detokenize{beyondml.tflow.utils:beyondml.tflow.utils.utils.ActiveSparsification.on_epoch_end}}
\pysigstartsignatures
\pysiglinewithargsret{\sphinxbfcode{\sphinxupquote{on\_epoch\_end}}}{\emph{\DUrole{n}{epoch}}, \emph{\DUrole{n}{logs}\DUrole{o}{=}\DUrole{default_value}{None}}}{}
\pysigstopsignatures
\sphinxAtStartPar
Called at the end of an epoch.

\sphinxAtStartPar
Subclasses should override for any actions to run. This function should
only be called during TRAIN mode.
\begin{quote}\begin{description}
\sphinxlineitem{Parameters}\begin{itemize}
\item {} 
\sphinxAtStartPar
\sphinxstyleliteralstrong{\sphinxupquote{epoch}} \textendash{} Integer, index of epoch.

\item {} 
\sphinxAtStartPar
\sphinxstyleliteralstrong{\sphinxupquote{logs}} \textendash{} Dict, metric results for this training epoch, and for the
validation epoch if validation is performed. Validation result
keys are prefixed with \sphinxtitleref{val\_}. For training epoch, the values of
the \sphinxtitleref{Model}’s metrics are returned. Example:
\sphinxtitleref{\{‘loss’: 0.2, ‘accuracy’: 0.7\}}.

\end{itemize}

\end{description}\end{quote}

\end{fulllineitems}

\index{on\_train\_begin() (beyondml.tflow.utils.utils.ActiveSparsification method)@\spxentry{on\_train\_begin()}\spxextra{beyondml.tflow.utils.utils.ActiveSparsification method}}

\begin{fulllineitems}
\phantomsection\label{\detokenize{beyondml.tflow.utils:beyondml.tflow.utils.utils.ActiveSparsification.on_train_begin}}
\pysigstartsignatures
\pysiglinewithargsret{\sphinxbfcode{\sphinxupquote{on\_train\_begin}}}{\emph{\DUrole{n}{logs}\DUrole{o}{=}\DUrole{default_value}{None}}}{}
\pysigstopsignatures
\sphinxAtStartPar
Called at the beginning of training.

\sphinxAtStartPar
Subclasses should override for any actions to run.
\begin{quote}\begin{description}
\sphinxlineitem{Parameters}
\sphinxAtStartPar
\sphinxstyleliteralstrong{\sphinxupquote{logs}} \textendash{} Dict. Currently no data is passed to this argument for this
method but that may change in the future.

\end{description}\end{quote}

\end{fulllineitems}


\end{fulllineitems}

\index{add\_layer\_masks() (in module beyondml.tflow.utils.utils)@\spxentry{add\_layer\_masks()}\spxextra{in module beyondml.tflow.utils.utils}}

\begin{fulllineitems}
\phantomsection\label{\detokenize{beyondml.tflow.utils:beyondml.tflow.utils.utils.add_layer_masks}}
\pysigstartsignatures
\pysiglinewithargsret{\sphinxcode{\sphinxupquote{beyondml.tflow.utils.utils.}}\sphinxbfcode{\sphinxupquote{add\_layer\_masks}}}{\emph{\DUrole{n}{model}}, \emph{\DUrole{n}{additional\_custom\_objects}\DUrole{o}{=}\DUrole{default_value}{None}}}{}
\pysigstopsignatures
\sphinxAtStartPar
Convert a trained model from one that does not have masking weights to one that does have
masking weights
\begin{quote}\begin{description}
\sphinxlineitem{Parameters}\begin{itemize}
\item {} 
\sphinxAtStartPar
\sphinxstyleliteralstrong{\sphinxupquote{model}} (\sphinxstyleliteralemphasis{\sphinxupquote{TensorFlow Keras model}}) \textendash{} The model to be converted

\item {} 
\sphinxAtStartPar
\sphinxstyleliteralstrong{\sphinxupquote{additional\_custom\_objects}} (\sphinxstyleliteralemphasis{\sphinxupquote{dict}}\sphinxstyleliteralemphasis{\sphinxupquote{ or }}\sphinxstyleliteralemphasis{\sphinxupquote{None}}\sphinxstyleliteralemphasis{\sphinxupquote{ (}}\sphinxstyleliteralemphasis{\sphinxupquote{default None}}\sphinxstyleliteralemphasis{\sphinxupquote{)}}) \textendash{} Additional custom layers to use

\end{itemize}

\sphinxlineitem{Returns}
\sphinxAtStartPar
\sphinxstylestrong{new\_model} \textendash{} The converted model

\sphinxlineitem{Return type}
\sphinxAtStartPar
TensorFlow Keras model

\end{description}\end{quote}

\end{fulllineitems}

\index{get\_custom\_objects() (in module beyondml.tflow.utils.utils)@\spxentry{get\_custom\_objects()}\spxextra{in module beyondml.tflow.utils.utils}}

\begin{fulllineitems}
\phantomsection\label{\detokenize{beyondml.tflow.utils:beyondml.tflow.utils.utils.get_custom_objects}}
\pysigstartsignatures
\pysiglinewithargsret{\sphinxcode{\sphinxupquote{beyondml.tflow.utils.utils.}}\sphinxbfcode{\sphinxupquote{get\_custom\_objects}}}{}{}
\pysigstopsignatures
\sphinxAtStartPar
Return a dictionary of custom objects (layers) to use when loading models trained using this package

\end{fulllineitems}

\index{get\_task\_masking\_gradients() (in module beyondml.tflow.utils.utils)@\spxentry{get\_task\_masking\_gradients()}\spxextra{in module beyondml.tflow.utils.utils}}

\begin{fulllineitems}
\phantomsection\label{\detokenize{beyondml.tflow.utils:beyondml.tflow.utils.utils.get_task_masking_gradients}}
\pysigstartsignatures
\pysiglinewithargsret{\sphinxcode{\sphinxupquote{beyondml.tflow.utils.utils.}}\sphinxbfcode{\sphinxupquote{get\_task\_masking\_gradients}}}{\emph{\DUrole{n}{model}}, \emph{\DUrole{n}{task\_num}}}{}
\pysigstopsignatures
\sphinxAtStartPar
Get the gradients of masking weights within a model
\begin{quote}\begin{description}
\sphinxlineitem{Parameters}
\sphinxAtStartPar
\sphinxstyleliteralstrong{\sphinxupquote{model}} (\sphinxstyleliteralemphasis{\sphinxupquote{TensorFlow Keras model}}) \textendash{} The model to retrieve the gradients of

\end{description}\end{quote}
\subsubsection*{Notes}
\begin{itemize}
\item {} \begin{description}
\sphinxlineitem{This function should only be run \sphinxstyleemphasis{before} the model has been trained}
\sphinxAtStartPar
or used to predict.  There is an unknown bug related to TensorFlow which
is leading to incorrect results after initial training

\end{description}

\item {} \begin{description}
\sphinxlineitem{When running this function, randomized input and output data is sent}
\sphinxAtStartPar
through the model to retrieve gradients respective to each task. If
the model is compiled using \sphinxtitleref{sparse\_categorical\_crossentropy’ loss,
this will break this function’s functionality. As a result, please
use \textasciigrave{}categorical\_crossentropy} (or even better, \sphinxtitleref{mse}) before running this function. After
retrieving gradients, the model can be recompiled with whatever parameters are desired.

\end{description}

\end{itemize}
\begin{quote}\begin{description}
\sphinxlineitem{Returns}
\sphinxAtStartPar
\sphinxstylestrong{gradients} \textendash{} The gradients of the masking weights of the model

\sphinxlineitem{Return type}
\sphinxAtStartPar
list of TensorFlow tensors

\end{description}\end{quote}

\end{fulllineitems}

\index{mask\_model() (in module beyondml.tflow.utils.utils)@\spxentry{mask\_model()}\spxextra{in module beyondml.tflow.utils.utils}}

\begin{fulllineitems}
\phantomsection\label{\detokenize{beyondml.tflow.utils:beyondml.tflow.utils.utils.mask_model}}
\pysigstartsignatures
\pysiglinewithargsret{\sphinxcode{\sphinxupquote{beyondml.tflow.utils.utils.}}\sphinxbfcode{\sphinxupquote{mask\_model}}}{\emph{\DUrole{n}{model}}, \emph{\DUrole{n}{percentile}}, \emph{\DUrole{n}{method}\DUrole{o}{=}\DUrole{default_value}{\textquotesingle{}gradients\textquotesingle{}}}, \emph{\DUrole{n}{exclusive}\DUrole{o}{=}\DUrole{default_value}{True}}, \emph{\DUrole{n}{x}\DUrole{o}{=}\DUrole{default_value}{None}}, \emph{\DUrole{n}{y}\DUrole{o}{=}\DUrole{default_value}{None}}}{}
\pysigstopsignatures
\sphinxAtStartPar
Mask the multitask model for training respective using the gradients for the tasks at hand
\begin{quote}\begin{description}
\sphinxlineitem{Parameters}\begin{itemize}
\item {} 
\sphinxAtStartPar
\sphinxstyleliteralstrong{\sphinxupquote{model}} (\sphinxstyleliteralemphasis{\sphinxupquote{keras model with MANN masking layers}}) \textendash{} The model to be masked

\item {} 
\sphinxAtStartPar
\sphinxstyleliteralstrong{\sphinxupquote{percentile}} (\sphinxstyleliteralemphasis{\sphinxupquote{int}}) \textendash{} Percentile to use in masking. Any weights less than the \sphinxtitleref{percentile} value will be made zero

\item {} 
\sphinxAtStartPar
\sphinxstyleliteralstrong{\sphinxupquote{method}} (\sphinxstyleliteralemphasis{\sphinxupquote{str}}\sphinxstyleliteralemphasis{\sphinxupquote{ (}}\sphinxstyleliteralemphasis{\sphinxupquote{default \textquotesingle{}gradients\textquotesingle{}}}\sphinxstyleliteralemphasis{\sphinxupquote{)}}) \textendash{} One of either ‘gradients’ or ‘magnitude’ \sphinxhyphen{} the method for how to identify weights to mask
If method is ‘gradients’, utilizes the gradients with respect to the passed x and y variables
to identify the subnetwork to activate for each task
If method is ‘magnitude’, uses the magnitude of the weights to identify the subnetwork to activate for each task

\item {} 
\sphinxAtStartPar
\sphinxstyleliteralstrong{\sphinxupquote{exclusive}} (\sphinxstyleliteralemphasis{\sphinxupquote{bool}}\sphinxstyleliteralemphasis{\sphinxupquote{ (}}\sphinxstyleliteralemphasis{\sphinxupquote{default True}}\sphinxstyleliteralemphasis{\sphinxupquote{)}}) \textendash{} Whether to restrict previously\sphinxhyphen{}used weight indices for each task. If \sphinxtitleref{True}, this identifies disjoint subsets of
weights within the layer which perform the tasks requested.

\item {} 
\sphinxAtStartPar
\sphinxstyleliteralstrong{\sphinxupquote{x}} (\sphinxstyleliteralemphasis{\sphinxupquote{list}}\sphinxstyleliteralemphasis{\sphinxupquote{ of }}\sphinxstyleliteralemphasis{\sphinxupquote{np.ndarray}}\sphinxstyleliteralemphasis{\sphinxupquote{ or }}\sphinxstyleliteralemphasis{\sphinxupquote{array\sphinxhyphen{}like}}) \textendash{} The training data input values, ignored if “method” is ‘magnitude’

\item {} 
\sphinxAtStartPar
\sphinxstyleliteralstrong{\sphinxupquote{y}} (\sphinxstyleliteralemphasis{\sphinxupquote{list}}\sphinxstyleliteralemphasis{\sphinxupquote{ of }}\sphinxstyleliteralemphasis{\sphinxupquote{np.ndarray}}\sphinxstyleliteralemphasis{\sphinxupquote{ or }}\sphinxstyleliteralemphasis{\sphinxupquote{array\sphinxhyphen{}like}}) \textendash{} The training data output values, ignored if “method” is ‘magnitude’

\end{itemize}

\end{description}\end{quote}

\end{fulllineitems}

\index{mask\_task\_weights() (in module beyondml.tflow.utils.utils)@\spxentry{mask\_task\_weights()}\spxextra{in module beyondml.tflow.utils.utils}}

\begin{fulllineitems}
\phantomsection\label{\detokenize{beyondml.tflow.utils:beyondml.tflow.utils.utils.mask_task_weights}}
\pysigstartsignatures
\pysiglinewithargsret{\sphinxcode{\sphinxupquote{beyondml.tflow.utils.utils.}}\sphinxbfcode{\sphinxupquote{mask\_task\_weights}}}{\emph{\DUrole{n}{model}}, \emph{\DUrole{n}{task\_masking\_gradients}}, \emph{\DUrole{n}{percentile}}, \emph{\DUrole{n}{respect\_previous\_tasks}\DUrole{o}{=}\DUrole{default_value}{True}}}{}
\pysigstopsignatures\begin{quote}\begin{description}
\sphinxlineitem{Parameters}\begin{itemize}
\item {} 
\sphinxAtStartPar
\sphinxstyleliteralstrong{\sphinxupquote{model}} (\sphinxstyleliteralemphasis{\sphinxupquote{TensorFlow Keras model}}) \textendash{} The model to be masked

\item {} 
\sphinxAtStartPar
\sphinxstyleliteralstrong{\sphinxupquote{task\_masking\_gradients}} (\sphinxstyleliteralemphasis{\sphinxupquote{list}}\sphinxstyleliteralemphasis{\sphinxupquote{ of }}\sphinxstyleliteralemphasis{\sphinxupquote{TensorFlow tensors}}) \textendash{} The gradients for the specific task requested

\item {} 
\sphinxAtStartPar
\sphinxstyleliteralstrong{\sphinxupquote{percentile}} (\sphinxstyleliteralemphasis{\sphinxupquote{int}}) \textendash{} The percentile to mask/prune

\item {} 
\sphinxAtStartPar
\sphinxstyleliteralstrong{\sphinxupquote{respect\_previous\_tasks}} (\sphinxstyleliteralemphasis{\sphinxupquote{bool}}\sphinxstyleliteralemphasis{\sphinxupquote{ (}}\sphinxstyleliteralemphasis{\sphinxupquote{default True}}\sphinxstyleliteralemphasis{\sphinxupquote{)}}) \textendash{} Whether to respect the weights used for previous tasks and not use them
for subsequent tasks

\end{itemize}

\sphinxlineitem{Returns}
\sphinxAtStartPar
\sphinxstylestrong{masked\_model} \textendash{} The masked model

\sphinxlineitem{Return type}
\sphinxAtStartPar
TensorFlow Keras model

\end{description}\end{quote}

\end{fulllineitems}

\index{quantize\_model() (in module beyondml.tflow.utils.utils)@\spxentry{quantize\_model()}\spxextra{in module beyondml.tflow.utils.utils}}

\begin{fulllineitems}
\phantomsection\label{\detokenize{beyondml.tflow.utils:beyondml.tflow.utils.utils.quantize_model}}
\pysigstartsignatures
\pysiglinewithargsret{\sphinxcode{\sphinxupquote{beyondml.tflow.utils.utils.}}\sphinxbfcode{\sphinxupquote{quantize\_model}}}{\emph{\DUrole{n}{model}}, \emph{\DUrole{n}{dtype}\DUrole{o}{=}\DUrole{default_value}{\textquotesingle{}float16\textquotesingle{}}}, \emph{\DUrole{n}{additional\_custom\_objects}\DUrole{o}{=}\DUrole{default_value}{None}}}{}
\pysigstopsignatures
\sphinxAtStartPar
Apply model quantization
\begin{quote}\begin{description}
\sphinxlineitem{Parameters}\begin{itemize}
\item {} 
\sphinxAtStartPar
\sphinxstyleliteralstrong{\sphinxupquote{model}} (\sphinxstyleliteralemphasis{\sphinxupquote{TensorFlow Keras Model}}) \textendash{} The model to quantize

\item {} 
\sphinxAtStartPar
\sphinxstyleliteralstrong{\sphinxupquote{dtype}} (\sphinxstyleliteralemphasis{\sphinxupquote{str}}\sphinxstyleliteralemphasis{\sphinxupquote{ or }}\sphinxstyleliteralemphasis{\sphinxupquote{TensorFlow datatype}}\sphinxstyleliteralemphasis{\sphinxupquote{ (}}\sphinxstyleliteralemphasis{\sphinxupquote{default \textquotesingle{}float16\textquotesingle{}}}\sphinxstyleliteralemphasis{\sphinxupquote{)}}) \textendash{} The datatype to quantize to

\item {} 
\sphinxAtStartPar
\sphinxstyleliteralstrong{\sphinxupquote{additional\_custom\_objects}} (\sphinxstyleliteralemphasis{\sphinxupquote{None}}\sphinxstyleliteralemphasis{\sphinxupquote{ or }}\sphinxstyleliteralemphasis{\sphinxupquote{dict}}\sphinxstyleliteralemphasis{\sphinxupquote{ (}}\sphinxstyleliteralemphasis{\sphinxupquote{default None}}\sphinxstyleliteralemphasis{\sphinxupquote{)}}) \textendash{} Additional custom  objects to use to instantiate the model

\end{itemize}

\sphinxlineitem{Returns}
\sphinxAtStartPar
\sphinxstylestrong{new\_model} \textendash{} The quantized model

\sphinxlineitem{Return type}
\sphinxAtStartPar
TensorFlow Keras Model

\end{description}\end{quote}

\end{fulllineitems}

\index{remove\_layer\_masks() (in module beyondml.tflow.utils.utils)@\spxentry{remove\_layer\_masks()}\spxextra{in module beyondml.tflow.utils.utils}}

\begin{fulllineitems}
\phantomsection\label{\detokenize{beyondml.tflow.utils:beyondml.tflow.utils.utils.remove_layer_masks}}
\pysigstartsignatures
\pysiglinewithargsret{\sphinxcode{\sphinxupquote{beyondml.tflow.utils.utils.}}\sphinxbfcode{\sphinxupquote{remove\_layer\_masks}}}{\emph{\DUrole{n}{model}}, \emph{\DUrole{n}{additional\_custom\_objects}\DUrole{o}{=}\DUrole{default_value}{None}}}{}
\pysigstopsignatures
\sphinxAtStartPar
Convert a trained model from using Masking layers to using non\sphinxhyphen{}masking layers
\begin{quote}\begin{description}
\sphinxlineitem{Parameters}\begin{itemize}
\item {} 
\sphinxAtStartPar
\sphinxstyleliteralstrong{\sphinxupquote{model}} (\sphinxstyleliteralemphasis{\sphinxupquote{TensorFlow Keras model}}) \textendash{} The model to be converted

\item {} 
\sphinxAtStartPar
\sphinxstyleliteralstrong{\sphinxupquote{additional\_custom\_objects}} (\sphinxstyleliteralemphasis{\sphinxupquote{dict}}\sphinxstyleliteralemphasis{\sphinxupquote{ or }}\sphinxstyleliteralemphasis{\sphinxupquote{None}}\sphinxstyleliteralemphasis{\sphinxupquote{ (}}\sphinxstyleliteralemphasis{\sphinxupquote{default None}}\sphinxstyleliteralemphasis{\sphinxupquote{)}}) \textendash{} Additional custom layers to use

\end{itemize}

\sphinxlineitem{Returns}
\sphinxAtStartPar
\sphinxstylestrong{new\_model} \textendash{} The converted model

\sphinxlineitem{Return type}
\sphinxAtStartPar
TensorFlow Keras model

\end{description}\end{quote}

\end{fulllineitems}

\index{train\_model() (in module beyondml.tflow.utils.utils)@\spxentry{train\_model()}\spxextra{in module beyondml.tflow.utils.utils}}

\begin{fulllineitems}
\phantomsection\label{\detokenize{beyondml.tflow.utils:beyondml.tflow.utils.utils.train_model}}
\pysigstartsignatures
\pysiglinewithargsret{\sphinxcode{\sphinxupquote{beyondml.tflow.utils.utils.}}\sphinxbfcode{\sphinxupquote{train\_model}}}{\emph{\DUrole{n}{model}}, \emph{\DUrole{n}{train\_x}}, \emph{\DUrole{n}{train\_y}}, \emph{\DUrole{n}{loss}}, \emph{\DUrole{n}{metrics}}, \emph{\DUrole{n}{optimizer}}, \emph{\DUrole{n}{cutoff}}, \emph{\DUrole{n}{batch\_size}\DUrole{o}{=}\DUrole{default_value}{32}}, \emph{\DUrole{n}{epochs}\DUrole{o}{=}\DUrole{default_value}{100}}, \emph{\DUrole{n}{starting\_sparsification}\DUrole{o}{=}\DUrole{default_value}{0}}, \emph{\DUrole{n}{max\_sparsification}\DUrole{o}{=}\DUrole{default_value}{99}}, \emph{\DUrole{n}{sparsification\_rate}\DUrole{o}{=}\DUrole{default_value}{5}}, \emph{\DUrole{n}{sparsification\_patience}\DUrole{o}{=}\DUrole{default_value}{10}}, \emph{\DUrole{n}{stopping\_patience}\DUrole{o}{=}\DUrole{default_value}{5}}}{}
\pysigstopsignatures
\end{fulllineitems}

\index{train\_model\_iteratively() (in module beyondml.tflow.utils.utils)@\spxentry{train\_model\_iteratively()}\spxextra{in module beyondml.tflow.utils.utils}}

\begin{fulllineitems}
\phantomsection\label{\detokenize{beyondml.tflow.utils:beyondml.tflow.utils.utils.train_model_iteratively}}
\pysigstartsignatures
\pysiglinewithargsret{\sphinxcode{\sphinxupquote{beyondml.tflow.utils.utils.}}\sphinxbfcode{\sphinxupquote{train\_model\_iteratively}}}{\emph{\DUrole{n}{model}}, \emph{\DUrole{n}{task\_gradients}}, \emph{\DUrole{n}{train\_x}}, \emph{\DUrole{n}{train\_y}}, \emph{\DUrole{n}{validation\_split}}, \emph{\DUrole{n}{delta}}, \emph{\DUrole{n}{batch\_size}}, \emph{\DUrole{n}{losses}}, \emph{\DUrole{n}{optimizer}\DUrole{o}{=}\DUrole{default_value}{\textquotesingle{}adam\textquotesingle{}}}, \emph{\DUrole{n}{metrics}\DUrole{o}{=}\DUrole{default_value}{None}}, \emph{\DUrole{n}{starting\_pruning}\DUrole{o}{=}\DUrole{default_value}{0}}, \emph{\DUrole{n}{pruning\_rate}\DUrole{o}{=}\DUrole{default_value}{10}}, \emph{\DUrole{n}{patience}\DUrole{o}{=}\DUrole{default_value}{5}}, \emph{\DUrole{n}{max\_epochs}\DUrole{o}{=}\DUrole{default_value}{100}}}{}
\pysigstopsignatures
\sphinxAtStartPar
Train a model iteratively on each task, first obtaining
baseline performance on each task and then iteratively
training and pruning each task as far back as possible while
maintaining acceptable performance on each task
\begin{quote}\begin{description}
\sphinxlineitem{Parameters}\begin{itemize}
\item {} 
\sphinxAtStartPar
\sphinxstyleliteralstrong{\sphinxupquote{model}} (\sphinxstyleliteralemphasis{\sphinxupquote{TensorFlow Keras model}}) \textendash{} The model to be trained

\item {} 
\sphinxAtStartPar
\sphinxstyleliteralstrong{\sphinxupquote{task\_gradients}} (\sphinxstyleliteralemphasis{\sphinxupquote{list}}\sphinxstyleliteralemphasis{\sphinxupquote{ of }}\sphinxstyleliteralemphasis{\sphinxupquote{TensorFlow tensors}}) \textendash{} Gradients for each task, output from the \sphinxtitleref{get\_task\_masking\_gradients} function

\item {} 
\sphinxAtStartPar
\sphinxstyleliteralstrong{\sphinxupquote{train\_x}} (\sphinxstyleliteralemphasis{\sphinxupquote{list}}\sphinxstyleliteralemphasis{\sphinxupquote{ of }}\sphinxstyleliteralemphasis{\sphinxupquote{numpy arrays}}\sphinxstyleliteralemphasis{\sphinxupquote{, }}\sphinxstyleliteralemphasis{\sphinxupquote{TensorFlow Datasets}}\sphinxstyleliteralemphasis{\sphinxupquote{, or }}\sphinxstyleliteralemphasis{\sphinxupquote{other}}) \textendash{} data types models can train with
The input data to use to train on

\item {} 
\sphinxAtStartPar
\sphinxstyleliteralstrong{\sphinxupquote{train\_y}} (\sphinxstyleliteralemphasis{\sphinxupquote{list}}\sphinxstyleliteralemphasis{\sphinxupquote{ of }}\sphinxstyleliteralemphasis{\sphinxupquote{numpy arrays}}\sphinxstyleliteralemphasis{\sphinxupquote{, }}\sphinxstyleliteralemphasis{\sphinxupquote{TensorFlow Datasets}}\sphinxstyleliteralemphasis{\sphinxupquote{, or }}\sphinxstyleliteralemphasis{\sphinxupquote{other}}) \textendash{} data types model can train with
The output data to use to train on

\item {} 
\sphinxAtStartPar
\sphinxstyleliteralstrong{\sphinxupquote{validation\_split}} (\sphinxstyleliteralemphasis{\sphinxupquote{float}}\sphinxstyleliteralemphasis{\sphinxupquote{, or }}\sphinxstyleliteralemphasis{\sphinxupquote{list}}\sphinxstyleliteralemphasis{\sphinxupquote{ of }}\sphinxstyleliteralemphasis{\sphinxupquote{float}}) \textendash{} The proportion of data to use for validation

\item {} 
\sphinxAtStartPar
\sphinxstyleliteralstrong{\sphinxupquote{delta}} (\sphinxstyleliteralemphasis{\sphinxupquote{float}}) \textendash{} The tolerance between validation losses to be considered “acceptable”
performance to continue

\item {} 
\sphinxAtStartPar
\sphinxstyleliteralstrong{\sphinxupquote{batch\_size}} (\sphinxstyleliteralemphasis{\sphinxupquote{int}}) \textendash{} The batch size to train with

\item {} 
\sphinxAtStartPar
\sphinxstyleliteralstrong{\sphinxupquote{losses}} (\sphinxstyleliteralemphasis{\sphinxupquote{str}}\sphinxstyleliteralemphasis{\sphinxupquote{, }}\sphinxstyleliteralemphasis{\sphinxupquote{list}}\sphinxstyleliteralemphasis{\sphinxupquote{, or }}\sphinxstyleliteralemphasis{\sphinxupquote{Keras loss function}}) \textendash{} The loss or losses to use when training

\item {} 
\sphinxAtStartPar
\sphinxstyleliteralstrong{\sphinxupquote{optimizer}} (\sphinxstyleliteralemphasis{\sphinxupquote{str}}\sphinxstyleliteralemphasis{\sphinxupquote{, }}\sphinxstyleliteralemphasis{\sphinxupquote{list}}\sphinxstyleliteralemphasis{\sphinxupquote{, or }}\sphinxstyleliteralemphasis{\sphinxupquote{Keras optimizer}}) \textendash{} The optimizer to use when training (default ‘adam’)

\item {} 
\sphinxAtStartPar
\sphinxstyleliteralstrong{\sphinxupquote{starting\_pruning}} (\sphinxstyleliteralemphasis{\sphinxupquote{int}}\sphinxstyleliteralemphasis{\sphinxupquote{ or }}\sphinxstyleliteralemphasis{\sphinxupquote{list}}\sphinxstyleliteralemphasis{\sphinxupquote{ of }}\sphinxstyleliteralemphasis{\sphinxupquote{int}}\sphinxstyleliteralemphasis{\sphinxupquote{ (}}\sphinxstyleliteralemphasis{\sphinxupquote{default 0}}\sphinxstyleliteralemphasis{\sphinxupquote{)}}) \textendash{} The starting pruning rate to use for each task

\item {} 
\sphinxAtStartPar
\sphinxstyleliteralstrong{\sphinxupquote{pruning\_rate}} (\sphinxstyleliteralemphasis{\sphinxupquote{int}}\sphinxstyleliteralemphasis{\sphinxupquote{ or }}\sphinxstyleliteralemphasis{\sphinxupquote{list}}\sphinxstyleliteralemphasis{\sphinxupquote{ of }}\sphinxstyleliteralemphasis{\sphinxupquote{int}}\sphinxstyleliteralemphasis{\sphinxupquote{ (}}\sphinxstyleliteralemphasis{\sphinxupquote{default}}\sphinxstyleliteralemphasis{\sphinxupquote{ {[}}}\sphinxstyleliteralemphasis{\sphinxupquote{10}}\sphinxstyleliteralemphasis{\sphinxupquote{, }}\sphinxstyleliteralemphasis{\sphinxupquote{5}}\sphinxstyleliteralemphasis{\sphinxupquote{, }}\sphinxstyleliteralemphasis{\sphinxupquote{2}}\sphinxstyleliteralemphasis{\sphinxupquote{, }}\sphinxstyleliteralemphasis{\sphinxupquote{1}}\sphinxstyleliteralemphasis{\sphinxupquote{{]}}}\sphinxstyleliteralemphasis{\sphinxupquote{)}}) \textendash{} The pruning rate to use

\item {} 
\sphinxAtStartPar
\sphinxstyleliteralstrong{\sphinxupquote{patience}} (\sphinxstyleliteralemphasis{\sphinxupquote{int}}\sphinxstyleliteralemphasis{\sphinxupquote{ (}}\sphinxstyleliteralemphasis{\sphinxupquote{default 5}}\sphinxstyleliteralemphasis{\sphinxupquote{)}}) \textendash{} The patience for number of epochs to wait for performance to improve sufficiently

\item {} 
\sphinxAtStartPar
\sphinxstyleliteralstrong{\sphinxupquote{max\_epochs}} (\sphinxstyleliteralemphasis{\sphinxupquote{int}}\sphinxstyleliteralemphasis{\sphinxupquote{ or }}\sphinxstyleliteralemphasis{\sphinxupquote{list}}\sphinxstyleliteralemphasis{\sphinxupquote{ of }}\sphinxstyleliteralemphasis{\sphinxupquote{int}}\sphinxstyleliteralemphasis{\sphinxupquote{ (}}\sphinxstyleliteralemphasis{\sphinxupquote{default 100}}\sphinxstyleliteralemphasis{\sphinxupquote{)}}) \textendash{} The maximum number of epochs to use for training each task

\end{itemize}

\end{description}\end{quote}

\end{fulllineitems}



\subparagraph{Module contents}
\label{\detokenize{beyondml.tflow.utils:module-beyondml.tflow.utils}}\label{\detokenize{beyondml.tflow.utils:module-contents}}\index{module@\spxentry{module}!beyondml.tflow.utils@\spxentry{beyondml.tflow.utils}}\index{beyondml.tflow.utils@\spxentry{beyondml.tflow.utils}!module@\spxentry{module}}
\sphinxAtStartPar
Some utilities to use when building, loading, and training MANN models


\subparagraph{Module contents}
\label{\detokenize{beyondml.tflow:module-beyondml.tflow}}\label{\detokenize{beyondml.tflow:module-contents}}\index{module@\spxentry{module}!beyondml.tflow@\spxentry{beyondml.tflow}}\index{beyondml.tflow@\spxentry{beyondml.tflow}!module@\spxentry{module}}
\sphinxAtStartPar
\#\# TensorFlow compatibility for building MANN models.

\sphinxAtStartPar
The \sphinxtitleref{beyondml.tflow} package contains two subpackages, \sphinxtitleref{beyondml.tflow.layers} and \sphinxtitleref{beyondml.tflow.utils}, which contain
the functionality to create and train MANN layers within TensorFlow. For individuals who are
familiar with the former name of this package, \sphinxtitleref{mann}, backwards compatibility can be achieved
(assuming only TensorFlow support is needed), by replacing the following line of code:

\begin{sphinxVerbatim}[commandchars=\\\{\}]
\PYG{g+gp}{\PYGZgt{}\PYGZgt{}\PYGZgt{} }\PYG{k+kn}{import} \PYG{n+nn}{mann}
\end{sphinxVerbatim}

\sphinxAtStartPar
with the following line:

\begin{sphinxVerbatim}[commandchars=\\\{\}]
\PYG{g+gp}{\PYGZgt{}\PYGZgt{}\PYGZgt{} }\PYG{k+kn}{import} \PYG{n+nn}{beyondml}\PYG{n+nn}{.}\PYG{n+nn}{tflow} \PYG{k}{as} \PYG{n+nn}{mann}
\end{sphinxVerbatim}

\sphinxAtStartPar
in all existing scripts.

\sphinxAtStartPar
Within the \sphinxtitleref{layers} package, there is current functionality for the the following layers:
\sphinxhyphen{} \sphinxtitleref{beyondml.tflow.layers.FilterLayer}
\sphinxhyphen{} \sphinxtitleref{beyondml.tflow.layers.MaskedConv2D}
\sphinxhyphen{} \sphinxtitleref{beyondml.tflow.layers.MaskedDense}
\sphinxhyphen{} \sphinxtitleref{beyondml.tflow.layers.MultiConv2D}
\sphinxhyphen{} \sphinxtitleref{beyondml.tflow.layers.MultiDense}
\sphinxhyphen{} \sphinxtitleref{beyondml.tflow.layers.MultiMaskedConv2D}
\sphinxhyphen{} \sphinxtitleref{beyondml.tflow.layers.MultiMaskedDense}
\sphinxhyphen{} \sphinxtitleref{beyondml.tflow.layers.MultiMaxPool2D}
\sphinxhyphen{} \sphinxtitleref{beyondml.tflow.layers.SelectorLayer}
\sphinxhyphen{} \sphinxtitleref{beyondml.tflow.layers.SumLayer}
\sphinxhyphen{} \sphinxtitleref{beyondml.tflow.layers.SparseDense}
\sphinxhyphen{} \sphinxtitleref{beyondml.tflow.layers.SparseConv}
\sphinxhyphen{} \sphinxtitleref{beyondml.tflow.layers.SparseMultiDense}
\sphinxhyphen{} \sphinxtitleref{beyondml.tflow.layers.SparseMultiConv}

\sphinxAtStartPar
\sphinxstylestrong{Note that with any of the sparse layers (such as the \textasciigrave{}SparseDense\textasciigrave{} layer), any model which
utilizes these layers will not be loadable using the traditional \textasciigrave{}load\_model\textasciigrave{} functions available
in TensorFlow. Instead, the model should be saved using either joblib or pickle.}

\sphinxAtStartPar
Within the \sphinxtitleref{utils} package, there are the current functions and classes:
\sphinxhyphen{} \sphinxtitleref{ActiveSparsification}
\sphinxhyphen{} \sphinxtitleref{build\_transformer\_block}
\sphinxhyphen{} \sphinxtitleref{build\_token\_position\_embedding\_block}
\sphinxhyphen{} \sphinxtitleref{get\_custom\_objects}
\sphinxhyphen{} \sphinxtitleref{mask\_model}
\sphinxhyphen{} \sphinxtitleref{remove\_layer\_masks}
\sphinxhyphen{} \sphinxtitleref{add\_layer\_masks}
\sphinxhyphen{} \sphinxtitleref{quantize\_model}
\sphinxhyphen{} \sphinxtitleref{get\_task\_masking\_gradients}
\sphinxhyphen{} \sphinxtitleref{mask\_task\_weights}
\sphinxhyphen{} \sphinxtitleref{train\_model\_iteratively}


\subsubsection{Module contents}
\label{\detokenize{beyondml:module-beyondml}}\label{\detokenize{beyondml:module-contents}}\index{module@\spxentry{module}!beyondml@\spxentry{beyondml}}\index{beyondml@\spxentry{beyondml}!module@\spxentry{module}}
\sphinxAtStartPar
BeyondML (formerly MANN) is a Python package which enables creating sparse multitask artificial neural networks (MANNs)
compatible with {[}TensorFlow{]}(\sphinxurl{https://tensorflow.org}) and {[}PyTorch{]}(\sphinxurl{https://pytorch.org}). This package
contains custom layers and utilities to facilitate the training and optimization of models using the
Reduction of Sub\sphinxhyphen{}Network Neuroplasticity (RSN2) training procedure developed by {[}AI Squared, Inc{]}(\sphinxurl{https://squared.ai}).

\sphinxAtStartPar
\#\#\# Installation

\sphinxAtStartPar
This package is available through {[}PyPi{]}(\sphinxurl{https://pypi.org}) and can be installed via the following command:

\sphinxAtStartPar
\sphinxcode{\sphinxupquote{\textasciigrave{}bash
pip install beyondml
\textasciigrave{}}}

\sphinxAtStartPar
\#\#\# Capabilities

\sphinxAtStartPar
There are two major subpackages within the BeyondML package, the \sphinxtitleref{beyondml.tflow} and the \sphinxtitleref{beyondml.pt} packages.
The \sphinxtitleref{beyondml.tflow} package contains functionality for building multitask models using TensorFlow, and the
\sphinxtitleref{beyondml.pt} package contains functionality for building multitask models using PyTorch.


\chapter{Changelog}
\label{\detokenize{index:changelog}}\begin{itemize}
\item {} \begin{description}
\sphinxlineitem{Version 0.1.0}\begin{itemize}
\item {} 
\sphinxAtStartPar
Refactored existing MANN repository to rename to BeyondML

\end{itemize}

\end{description}

\item {} \begin{description}
\sphinxlineitem{Version 0.1.1}\begin{itemize}
\item {} \begin{description}
\sphinxlineitem{Added the \sphinxtitleref{SparseDense}, \sphinxtitleref{SparseConv}, \sphinxtitleref{SparseMultiDense}, and \sphinxtitleref{SparseMultiConv} layers to}
\sphinxAtStartPar
\sphinxtitleref{beyondml.tflow.layers}, giving users the functionality to utilize sparse tensors during
inference

\end{description}

\end{itemize}

\end{description}

\item {} \begin{description}
\sphinxlineitem{Version 0.1.2}\begin{itemize}
\item {} 
\sphinxAtStartPar
Added the \sphinxtitleref{MaskedMultiHeadAttention}, \sphinxtitleref{MaskedTransformerEncoderLayer}, and \sphinxtitleref{MaskedTransformerDecoderLayer} layers to \sphinxtitleref{beyondml.pt.layers} to add pruning to the transformer architecture

\item {} 
\sphinxAtStartPar
Added \sphinxtitleref{MaskedConv3D}, \sphinxtitleref{MultiMaskedConv3D}, \sphinxtitleref{MultiConv3D}, \sphinxtitleref{MultiMaxPool3D}, \sphinxtitleref{SparseConv3D}, and \sphinxtitleref{SparseMultiConv3D} layers to \sphinxtitleref{beyondml.tflow.layers}

\item {} 
\sphinxAtStartPar
Added \sphinxtitleref{MaskedConv3D}, \sphinxtitleref{MultiMaskedConv3D}, \sphinxtitleref{MultiConv3D}, \sphinxtitleref{MultiMaxPool3D}, \sphinxtitleref{SparseConv3D}, \sphinxtitleref{SparseMultiConv3D}, and \sphinxtitleref{MultiMaxPool2D} layers to \sphinxtitleref{beyondml.pt.layers}

\end{itemize}

\end{description}

\item {} \begin{description}
\sphinxlineitem{Version 0.1.3}\begin{itemize}
\item {} 
\sphinxAtStartPar
Added \sphinxtitleref{beyondml.pt} compatibility with more native PyTorch functionality for using models on different devices and datatypes

\item {} 
\sphinxAtStartPar
Added \sphinxtitleref{train\_model} function to \sphinxtitleref{beyondml.tflow.utils}

\item {} 
\sphinxAtStartPar
Added \sphinxtitleref{MultitaskNormalization} layer to \sphinxtitleref{beyondml.tflow.layers} and \sphinxtitleref{beyondml.pt.layers}

\end{itemize}

\end{description}

\item {} \begin{description}
\sphinxlineitem{Version 0.1.4}\begin{itemize}
\item {} 
\sphinxAtStartPar
Updated documentation to use Sphinx

\end{itemize}

\end{description}

\end{itemize}


\renewcommand{\indexname}{Python Module Index}
\begin{sphinxtheindex}
\let\bigletter\sphinxstyleindexlettergroup
\bigletter{b}
\item\relax\sphinxstyleindexentry{beyondml}\sphinxstyleindexpageref{beyondml:\detokenize{module-beyondml}}
\item\relax\sphinxstyleindexentry{beyondml.pt}\sphinxstyleindexpageref{beyondml.pt:\detokenize{module-beyondml.pt}}
\item\relax\sphinxstyleindexentry{beyondml.pt.layers}\sphinxstyleindexpageref{beyondml.pt.layers:\detokenize{module-beyondml.pt.layers}}
\item\relax\sphinxstyleindexentry{beyondml.pt.layers.Conv2D}\sphinxstyleindexpageref{beyondml.pt.layers:\detokenize{module-beyondml.pt.layers.Conv2D}}
\item\relax\sphinxstyleindexentry{beyondml.pt.layers.Conv3D}\sphinxstyleindexpageref{beyondml.pt.layers:\detokenize{module-beyondml.pt.layers.Conv3D}}
\item\relax\sphinxstyleindexentry{beyondml.pt.layers.Dense}\sphinxstyleindexpageref{beyondml.pt.layers:\detokenize{module-beyondml.pt.layers.Dense}}
\item\relax\sphinxstyleindexentry{beyondml.pt.layers.FilterLayer}\sphinxstyleindexpageref{beyondml.pt.layers:\detokenize{module-beyondml.pt.layers.FilterLayer}}
\item\relax\sphinxstyleindexentry{beyondml.pt.layers.MaskedConv2D}\sphinxstyleindexpageref{beyondml.pt.layers:\detokenize{module-beyondml.pt.layers.MaskedConv2D}}
\item\relax\sphinxstyleindexentry{beyondml.pt.layers.MaskedConv3D}\sphinxstyleindexpageref{beyondml.pt.layers:\detokenize{module-beyondml.pt.layers.MaskedConv3D}}
\item\relax\sphinxstyleindexentry{beyondml.pt.layers.MaskedDense}\sphinxstyleindexpageref{beyondml.pt.layers:\detokenize{module-beyondml.pt.layers.MaskedDense}}
\item\relax\sphinxstyleindexentry{beyondml.pt.layers.MaskedMultiHeadAttention}\sphinxstyleindexpageref{beyondml.pt.layers:\detokenize{module-beyondml.pt.layers.MaskedMultiHeadAttention}}
\item\relax\sphinxstyleindexentry{beyondml.pt.layers.MaskedTransformerDecoderLayer}\sphinxstyleindexpageref{beyondml.pt.layers:\detokenize{module-beyondml.pt.layers.MaskedTransformerDecoderLayer}}
\item\relax\sphinxstyleindexentry{beyondml.pt.layers.MaskedTransformerEncoderLayer}\sphinxstyleindexpageref{beyondml.pt.layers:\detokenize{module-beyondml.pt.layers.MaskedTransformerEncoderLayer}}
\item\relax\sphinxstyleindexentry{beyondml.pt.layers.MultiConv2D}\sphinxstyleindexpageref{beyondml.pt.layers:\detokenize{module-beyondml.pt.layers.MultiConv2D}}
\item\relax\sphinxstyleindexentry{beyondml.pt.layers.MultiConv3D}\sphinxstyleindexpageref{beyondml.pt.layers:\detokenize{module-beyondml.pt.layers.MultiConv3D}}
\item\relax\sphinxstyleindexentry{beyondml.pt.layers.MultiDense}\sphinxstyleindexpageref{beyondml.pt.layers:\detokenize{module-beyondml.pt.layers.MultiDense}}
\item\relax\sphinxstyleindexentry{beyondml.pt.layers.MultiMaskedConv2D}\sphinxstyleindexpageref{beyondml.pt.layers:\detokenize{module-beyondml.pt.layers.MultiMaskedConv2D}}
\item\relax\sphinxstyleindexentry{beyondml.pt.layers.MultiMaskedConv3D}\sphinxstyleindexpageref{beyondml.pt.layers:\detokenize{module-beyondml.pt.layers.MultiMaskedConv3D}}
\item\relax\sphinxstyleindexentry{beyondml.pt.layers.MultiMaskedDense}\sphinxstyleindexpageref{beyondml.pt.layers:\detokenize{module-beyondml.pt.layers.MultiMaskedDense}}
\item\relax\sphinxstyleindexentry{beyondml.pt.layers.MultiMaxPool2D}\sphinxstyleindexpageref{beyondml.pt.layers:\detokenize{module-beyondml.pt.layers.MultiMaxPool2D}}
\item\relax\sphinxstyleindexentry{beyondml.pt.layers.MultiMaxPool3D}\sphinxstyleindexpageref{beyondml.pt.layers:\detokenize{module-beyondml.pt.layers.MultiMaxPool3D}}
\item\relax\sphinxstyleindexentry{beyondml.pt.layers.MultitaskNormalization}\sphinxstyleindexpageref{beyondml.pt.layers:\detokenize{module-beyondml.pt.layers.MultitaskNormalization}}
\item\relax\sphinxstyleindexentry{beyondml.pt.layers.SelectorLayer}\sphinxstyleindexpageref{beyondml.pt.layers:\detokenize{module-beyondml.pt.layers.SelectorLayer}}
\item\relax\sphinxstyleindexentry{beyondml.pt.layers.SparseConv2D}\sphinxstyleindexpageref{beyondml.pt.layers:\detokenize{module-beyondml.pt.layers.SparseConv2D}}
\item\relax\sphinxstyleindexentry{beyondml.pt.layers.SparseConv3D}\sphinxstyleindexpageref{beyondml.pt.layers:\detokenize{module-beyondml.pt.layers.SparseConv3D}}
\item\relax\sphinxstyleindexentry{beyondml.pt.layers.SparseDense}\sphinxstyleindexpageref{beyondml.pt.layers:\detokenize{module-beyondml.pt.layers.SparseDense}}
\item\relax\sphinxstyleindexentry{beyondml.pt.layers.SparseMultiConv2D}\sphinxstyleindexpageref{beyondml.pt.layers:\detokenize{module-beyondml.pt.layers.SparseMultiConv2D}}
\item\relax\sphinxstyleindexentry{beyondml.pt.layers.SparseMultiConv3D}\sphinxstyleindexpageref{beyondml.pt.layers:\detokenize{module-beyondml.pt.layers.SparseMultiConv3D}}
\item\relax\sphinxstyleindexentry{beyondml.pt.layers.SparseMultiDense}\sphinxstyleindexpageref{beyondml.pt.layers:\detokenize{module-beyondml.pt.layers.SparseMultiDense}}
\item\relax\sphinxstyleindexentry{beyondml.pt.utils}\sphinxstyleindexpageref{beyondml.pt.utils:\detokenize{module-beyondml.pt.utils}}
\item\relax\sphinxstyleindexentry{beyondml.pt.utils.utils}\sphinxstyleindexpageref{beyondml.pt.utils:\detokenize{module-beyondml.pt.utils.utils}}
\item\relax\sphinxstyleindexentry{beyondml.tflow}\sphinxstyleindexpageref{beyondml.tflow:\detokenize{module-beyondml.tflow}}
\item\relax\sphinxstyleindexentry{beyondml.tflow.layers}\sphinxstyleindexpageref{beyondml.tflow.layers:\detokenize{module-beyondml.tflow.layers}}
\item\relax\sphinxstyleindexentry{beyondml.tflow.layers.FilterLayer}\sphinxstyleindexpageref{beyondml.tflow.layers:\detokenize{module-beyondml.tflow.layers.FilterLayer}}
\item\relax\sphinxstyleindexentry{beyondml.tflow.layers.MaskedConv2D}\sphinxstyleindexpageref{beyondml.tflow.layers:\detokenize{module-beyondml.tflow.layers.MaskedConv2D}}
\item\relax\sphinxstyleindexentry{beyondml.tflow.layers.MaskedConv3D}\sphinxstyleindexpageref{beyondml.tflow.layers:\detokenize{module-beyondml.tflow.layers.MaskedConv3D}}
\item\relax\sphinxstyleindexentry{beyondml.tflow.layers.MaskedDense}\sphinxstyleindexpageref{beyondml.tflow.layers:\detokenize{module-beyondml.tflow.layers.MaskedDense}}
\item\relax\sphinxstyleindexentry{beyondml.tflow.layers.MultiConv2D}\sphinxstyleindexpageref{beyondml.tflow.layers:\detokenize{module-beyondml.tflow.layers.MultiConv2D}}
\item\relax\sphinxstyleindexentry{beyondml.tflow.layers.MultiConv3D}\sphinxstyleindexpageref{beyondml.tflow.layers:\detokenize{module-beyondml.tflow.layers.MultiConv3D}}
\item\relax\sphinxstyleindexentry{beyondml.tflow.layers.MultiDense}\sphinxstyleindexpageref{beyondml.tflow.layers:\detokenize{module-beyondml.tflow.layers.MultiDense}}
\item\relax\sphinxstyleindexentry{beyondml.tflow.layers.MultiMaskedConv2D}\sphinxstyleindexpageref{beyondml.tflow.layers:\detokenize{module-beyondml.tflow.layers.MultiMaskedConv2D}}
\item\relax\sphinxstyleindexentry{beyondml.tflow.layers.MultiMaskedConv3D}\sphinxstyleindexpageref{beyondml.tflow.layers:\detokenize{module-beyondml.tflow.layers.MultiMaskedConv3D}}
\item\relax\sphinxstyleindexentry{beyondml.tflow.layers.MultiMaskedDense}\sphinxstyleindexpageref{beyondml.tflow.layers:\detokenize{module-beyondml.tflow.layers.MultiMaskedDense}}
\item\relax\sphinxstyleindexentry{beyondml.tflow.layers.MultiMaxPool2D}\sphinxstyleindexpageref{beyondml.tflow.layers:\detokenize{module-beyondml.tflow.layers.MultiMaxPool2D}}
\item\relax\sphinxstyleindexentry{beyondml.tflow.layers.MultiMaxPool3D}\sphinxstyleindexpageref{beyondml.tflow.layers:\detokenize{module-beyondml.tflow.layers.MultiMaxPool3D}}
\item\relax\sphinxstyleindexentry{beyondml.tflow.layers.MultitaskNormalization}\sphinxstyleindexpageref{beyondml.tflow.layers:\detokenize{module-beyondml.tflow.layers.MultitaskNormalization}}
\item\relax\sphinxstyleindexentry{beyondml.tflow.layers.SelectorLayer}\sphinxstyleindexpageref{beyondml.tflow.layers:\detokenize{module-beyondml.tflow.layers.SelectorLayer}}
\item\relax\sphinxstyleindexentry{beyondml.tflow.layers.SparseConv2D}\sphinxstyleindexpageref{beyondml.tflow.layers:\detokenize{module-beyondml.tflow.layers.SparseConv2D}}
\item\relax\sphinxstyleindexentry{beyondml.tflow.layers.SparseConv3D}\sphinxstyleindexpageref{beyondml.tflow.layers:\detokenize{module-beyondml.tflow.layers.SparseConv3D}}
\item\relax\sphinxstyleindexentry{beyondml.tflow.layers.SparseDense}\sphinxstyleindexpageref{beyondml.tflow.layers:\detokenize{module-beyondml.tflow.layers.SparseDense}}
\item\relax\sphinxstyleindexentry{beyondml.tflow.layers.SparseMultiConv2D}\sphinxstyleindexpageref{beyondml.tflow.layers:\detokenize{module-beyondml.tflow.layers.SparseMultiConv2D}}
\item\relax\sphinxstyleindexentry{beyondml.tflow.layers.SparseMultiConv3D}\sphinxstyleindexpageref{beyondml.tflow.layers:\detokenize{module-beyondml.tflow.layers.SparseMultiConv3D}}
\item\relax\sphinxstyleindexentry{beyondml.tflow.layers.SparseMultiDense}\sphinxstyleindexpageref{beyondml.tflow.layers:\detokenize{module-beyondml.tflow.layers.SparseMultiDense}}
\item\relax\sphinxstyleindexentry{beyondml.tflow.layers.SumLayer}\sphinxstyleindexpageref{beyondml.tflow.layers:\detokenize{module-beyondml.tflow.layers.SumLayer}}
\item\relax\sphinxstyleindexentry{beyondml.tflow.utils}\sphinxstyleindexpageref{beyondml.tflow.utils:\detokenize{module-beyondml.tflow.utils}}
\item\relax\sphinxstyleindexentry{beyondml.tflow.utils.transformer}\sphinxstyleindexpageref{beyondml.tflow.utils:\detokenize{module-beyondml.tflow.utils.transformer}}
\item\relax\sphinxstyleindexentry{beyondml.tflow.utils.utils}\sphinxstyleindexpageref{beyondml.tflow.utils:\detokenize{module-beyondml.tflow.utils.utils}}
\end{sphinxtheindex}

\renewcommand{\indexname}{Index}
\printindex
\end{document}